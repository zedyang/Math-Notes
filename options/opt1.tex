\documentclass[a4paper, 10pt]{article}    
\usepackage{geometry}       
\geometry{a4paper}
\geometry{margin=1in} 
\usepackage{paralist}
  \let\itemize\compactitem
  \let\enditemize\endcompactitem
  \let\enumerate\compactenum
  \let\endenumerate\endcompactenum
  \let\description\compactdesc
  \let\enddescription\endcompactdesc
  \pltopsep=\medskipamount
  \plitemsep=1pt
  \plparsep=1pt
\usepackage[english]{babel}
\usepackage[utf8]{inputenc}

\usepackage{bbm, bm}
\usepackage{amsmath, amssymb, amsthm, mathrsfs}
\usepackage{booktabs, tikz, array, eurosym}
\usepackage{float}
\renewcommand{\arraystretch}{1.4}
\newcolumntype{L}{>{\arraybackslash}m{10cm}}
\newcommand\indep{\protect\mathpalette{\protect\indeP}{\perp}}
\def\indeP#1#2{\mathrel{\rlap{$#1#2$}\mkern2mu{#1#2}}}
\DeclareMathOperator*{\argmin}{argmin}
\DeclareMathOperator*{\argmax}{argmax}

\pagestyle{headings}
\newcommand{\boxwidth}{430pt}

\theoremstyle{definition}
\newtheorem{problem}{Problem}

\newtheoremstyle{hSol}
  {1.0pt}% Space above
  {1.0pt}% Space below
  {}% bodyfont
  {}% indent
  {\bfseries}% thm head font
  {.}% punctuation after thm head
  { }% Space after thm head
  {}% thm head spec

\theoremstyle{hSol}
\newtheorem*{solution}{Solution}



\title{\textbf{Options Assignment I}}
\author{Ze Yang (zey@andrew.cmu.edu)}

\begin{document}
\maketitle



\noindent\rule{16cm}{0.4pt}
%///////////////////////////////////////////////////////////////////////

\begin{problem}
\end{problem}
\begin{solution} \textbf{(Part-a)}

% Table generated by Excel2LaTeX from sheet 'Sheet2'
\begin{table}[htbp]
  \centering
  \caption{State Prices and RN Probabilities}
  \label{rn}
  \vspace{4pt}
  \def\arraystretch{1.15}
    \begin{tabular}{rrrrrr}
    \toprule
    Macro State & Objective Prob & State Price & RN Prob & $\mathbb{E}\left[\text{Ret}\right]-R_f$& Pricing Kernel\\
    \hline
    Strong & 0.1000  & \$~0.0950 & 0.0964 & 0.0374 & 0.9500 \\
    Weak  & 0.3000  & \$~0.2800 & 0.2843 & 0.0562 & 0.9333 \\
    Flat  & 0.2500  & \$~0.2500 & 0.2538 & -0.0152 & 1.0000 \\
    Continued Recession & 0.3500  & \$~0.3600 & 0.3655 & -0.0430 & 1.0286 \\
    \hline
    Sums & 1.0000 & 0.9850 & 1.0000 \\
    \hline
    \hline
    ~\\

    $T-t$ & 0.5 \\
    $r_{t,T}(a_t)$ & 0.03023 \\
    \bottomrule
    \end{tabular}%
  \label{tab:addlabel}%
\end{table}%
~\\
Let $a_T \in A_T = \{strong, weak, flat, continued~recession\}$ be the four future states. Our inputs are the objective probabilities $p(a_T|a_t)$ in the second column of table \ref{rn}, and the state prices $\pi(a_T|a_t)$ in the third column of table \ref{rn}. \\
The risk neutral probabilities are listed in the fourth column:
\begin{equation}
  q^{RN}(a_T|a_t) = \frac{\pi(a_T|a_t)}{\sum_{a_T\in A_T}\pi(a_T|a_t)} = \pi(a_T|a_t)e^{-r_{t,T}(a_t)(T-t)}
\end{equation}
Therefore we can calculate the continuously compounded annualized risk-free interest rate:uo
\begin{equation}
  \begin{split}
     r_{t,T}(a_t) &= -\frac{1}{T-t}\log\left(\sum_{a_T\in A_T}\pi(a_T|a_t)\right)\\
     &= - \frac{1}{0.5}\log(0.985) \approx 0.03023
  \end{split}
\end{equation}
The pricing kernels are listed in the sixth column:
\begin{equation}
  m(a_T|a_t) = \frac{\pi(a_T|a_t)}{p(a_T|a_t)} 
\end{equation}
The expected risk premium are in the fifth column:
\begin{equation}
  \mathbb{E}_t\left[R_i\right] - R_f = \left(\frac{p(a_T|a_t)}{q^{RN}(a_T|a_t)}-1\right) e^{r_{t,T}(a_t)(T-t)}
\end{equation}
Now we calculate the option price via risk neutral valuation, see the third column of table-2. We have:
\begin{equation}
  c_{jt} = \sum_{a_T\in A_T} \pi(a_T|a_t) c_j^*(a_T)
\end{equation}
\begin{itemize}
  \item[$\cdot$] Price of reverse macro straddle that pays \$1 million if $a_T \in \{weak, flat\}$: $c_{1t} = 0.53$ million \$.
  \item[$\cdot$] Price of reverse macro straddle that pays \$25 million if $a_T \in \{weak, flat\}$: $c_{2t} = 13.25$ million \$.
  \item[$\cdot$] Price of long macro straddle that pays \$1 million if $a_T \in \{strong, recession\}$: $c_{3t} = 0.46$ million \$.
\end{itemize}
By the DCF valuation:
\begin{equation}
  c_{jt} = \frac{\sum_{a_T\in A_T} p(a_T|a_t)c^*_j(a_T)}{1+RAD_{jt}}
\end{equation}
\begin{itemize}
  \item[$\cdot$] RAD of reverse macro straddle that pays \$1 million if $a_T \in \{weak, flat\}$: $RAD_{1t} = 0.0377$.
  \item[$\cdot$] RAD of reverse macro straddle that pays \$25 million if $a_T \in \{weak, flat\}$: $RAD_{2t} = 0.0377$.
  \item[$\cdot$] RAD of long macro straddle that pays \$1 million if $a_T \in \{strong, recession\}$: $RAD_{3t} = -0.0110$.
\end{itemize}
~\\
\textbf{Explaination}: 
\begin{itemize}
  \item[$\cdot$] As we have seen in the table, the reverse macro straddle is a portfolio of state claim for $flat$ and $weak$. Hence $c_{1t} = \pi(weak|a_t)+\pi(flat|a_t)$. By our calculation, the risk premium of state claim $weak$ is positive (0.0562), and the risk premium of state claim $flat$ is negative (-0.0152), but a lot smaller than that of $weak$ in absolute value. The positive risk premium dominates. As a result, the overall impact on risk-adjusted discount rate is a positive risk premium. Hence $RAD_{1j}$ is greater than the risk-free rate.
  \item[$\cdot$] The Second case is just 25 shares of the reverse macro straddle in the First case. $c_{2t} = 25c_{1t}$. Holding more shares of options does \textbf{not} change the payoff probabilites, preferences or timing. So the $RAD$ does not change for the second one.
  \item[$\cdot$] For the third case, the long macro straddle is a portfolio of state claim for $strong$ and $recession$. Hence $c_{3t} = \pi(strong|a_t)+\pi(recession|a_t)$. By our calculation, the risk premium of state claim $strong$ is positive (0.0374), and the risk premium of state claim $recession$ is negative (-0.0430), but greater than that of $strong$ in absolute value. The negative risk premium dominates. As a result, the overall impact on risk-adjusted discount rate is a negative risk premium. The negative risk-premium is so deep that it wiped out risk-free rate. As a result, the RAD ends up to be negative.
\end{itemize}



\begin{table}[H]
  \centering
  \caption{Option Valuations}
  \vspace{4pt}
  \def\arraystretch{1.15}
    \begin{tabular}{rrrrr}
    \toprule
    Reverse Macro Straddle &   \\
    Macro State & Payoffs $c^*(a_T)$ & RN Valuations & $p(a_T|a_t)c^*(a_T)$ & RAD \\
    \midrule
    Strong & 0     & 0.00  & 0.00  &  \\
    Weak  & 1     & 0.28  & 0.30  &  \\
    Flat  & 1     & 0.25  & 0.25  &  \\
    Continued Recession & 0     & 0.00  & 0.00  &  \\
    \midrule
    value &       & \$0.53 & \$0.55 & 0.0377358 \\
          &       &       &       &  \\
    Reverse Macro Straddle (25)   \\
    Macro State & Payoffs $c^*(a_T)$ & RN Valuations & $p(a_T|a_t)c^*(a_T)$ & RAD \\
    \midrule
    Strong & 0     & 0.00  & 0.00  &  \\
    Weak  & 25    & 7.00  & 7.50  &  \\
    Flat  & 25    & 6.25  & 6.25  &  \\
    Continued Recession & 0     & 0.00  & 0.00  &  \\
    \midrule
    option value &       & \$13.25 & \$13.75 & 0.0377358 \\
          &       &       &       &  \\
    Long straddle  \\
    Macro State & Payoffs $c^*(a_T)$ & RN Valuations & $p(a_T|a_t)c^*(a_T)$ & RAD \\
    \midrule
    Strong & 1     & 0.10  & 0.10  &  \\
    Weak  & 0     & 0.00  & 0.00  &  \\
    Flat  & 0     & 0.00  & 0.00  &  \\
    Continued Recession & 1     & 0.36  & 0.35  &  \\
    \midrule
    option value &       & \$0.46 & \$0.45 & -0.010989 \\
    \bottomrule
    \end{tabular}%
  \label{tab:addlabel}%
\end{table}%
~\\
\textbf{(Part-b)}
% Table generated by Excel2LaTeX from sheet 'Sheet2'
\begin{equation}
  F_{t,T} = \mathbb{E}_t^{RN}\left[S^*_T\right] = 1074.7462
\end{equation}
\begin{table}[H]
  \centering
  \caption{Future Price}
  \vspace{4pt}
  \def\arraystretch{1.15}
  \begin{tabular}{rrrr}
    \toprule
    Macro State & Payoffs & RN Probs & $S^*_Tq^{RN}(a_T|a_t)$ \\
    \hline
    Strong & 1195  & 0.0964 & 115.25381 \\
    Weak  & 1150  & 0.2843 & 326.90355 \\
    Flat  & 1110  & 0.2538 & 281.72589 \\
    Continued Recession & 960   & 0.3655 & 350.86294 \\
    \hline
    Future Price &       &       & 1074.7462 \\
    \bottomrule
  \end{tabular}%
  \label{tab:addlabel}%
\end{table}%
~\\
\textbf{(Part-c)}
State price $\pi(a_T|a_t)$ summarizes combined valuation impact of timing, probabilities and preferences. \\
Since the objective probabilities and the timing of the states are unchanged, the \textbf{preferences} would need to have changed to account for the changes in state prices.

\end{solution}

\noindent\rule{16cm}{0.4pt}
%///////////////////////////////////////////////////////////////////////



\end{document}
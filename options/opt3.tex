\documentclass[10 pt]{hwtemplate} % Use the custom resume.cls style

%-------------------------------------------------------------------------------
% Document begins
%-------------------------------------------------------------------------------

\title{\textbf{Options Assignment 3}}
\author{Ze Yang (zey@andrew.cmu.edu)}

\begin{document}
\maketitle


\noindent\rule{16cm}{0.4pt}
%///////////////////////////////////////////////////////////////////////
\textbf{Problem. 2} (\textit{Futures and forward cash flows})\\
\begin{solution}
\textbf{(a)} The variation margin cash flow of future contract is given by
$$
C_t = F_{t,T} - F_{t-\Delta t, T}
$$
Since future contracts are linear in there cash flows, the cash flow for 20 longs is just $20C_t$, and that for 2 shorts is $-2C_t$. 
\begin{figure}[H]
  \centering
  \captionsetup{justification=centering}
  \caption{\label{fig:q21}Variation Margin Cashflow of Future Contracts}
  %\setlength{\abovecaptionskip}{-10pt}
  %\setlength{\belowcaptionskip}{2pt}
  \includegraphics[scale=0.75]{figures/fig1.png}
\end{figure}

\textbf{(b)} The variation margin cash flow of forward contract is:
$$
C_t = \begin{cases}S_T - F_0 & t=T\\
0 & \text{otherwise}
\end{cases}
$$

\begin{figure}[H]
  \centering
  \captionsetup{justification=centering}
  \caption{\label{fig:q22}Variation Margin Cashflow of Forward Contracts}
  %\setlength{\abovecaptionskip}{-10pt}
  %\setlength{\belowcaptionskip}{2pt}
  \includegraphics[scale=0.75]{figures/fig2.png}
\end{figure}
\end{solution}

\noindent\rule{16cm}{0.4pt}
%///////////////////////////////////////////////////////////////////////
\textbf{Problem. 18} (\textit{Binomial})\\
\begin{solution} \textbf{(a)} We have:
\begin{equation}
  \pi_u = \frac{R-d}{R(u-d)} = 0.3127;~~~~  \pi_d = \frac{u-R}{R(u-d)} = 0.6774
\end{equation}
Let $c_t(S_t)$ be the value of american call at time $t$, we have:
\begin{equation}
  c_t(S_t, D_t) = \max\left\{\max(0, S_t^{[cum]}-K), \pi_u c_{t+1}(uS_t^{[ex]})+\pi_d c_{t+1}(dS_t^{[ex]})\right\}
\end{equation}
Where $S_t^{[cum]}$ is the cum dividend stock price, $S_t^{[ex]}$ is the ex dividend stock price. $S_t^{[ex]} = S_t^{[cum]} - D_t$. The binomial tree calulation is presented in the table below.\\
We obtain:
$$c_0 = 1.2569$$
I.e. the 1-year American call option worth 1.2569 dollars today.\\
~\\
\textbf{(b)} Denote the value of asian call option along path $w$ as $Ac_n(w)$, the number of periods is $N$. We have:
\begin{equation}
  Ac_n = \begin{cases}
  \max(0, \frac{1}{N}\sum_{i=0}^N S_i^{[cum]}(w)) & n=N\\
  \pi_u Ac_{n+1}((w, u)) + \pi_d Ac_{n+1}((w, d)) & 0\leq n<N
  \end{cases}
\end{equation}
The binomial tree calulation is presented in the table below.\\
We obtain:
$$Ac_0 = 0.2830$$
I.e. the 1-year Asian call option worth 0.2830 dollars today.\\


\begin{figure}[H]
  \centering
  \captionsetup{justification=centering}
  \caption{\label{fig:q18pa}American Option}
  %\setlength{\abovecaptionskip}{-10pt}
  %\setlength{\belowcaptionskip}{2pt}
  \includegraphics[scale=0.7]{figures/fig3.png}
\end{figure}


\begin{figure}[H]
  \centering
  \captionsetup{justification=centering}
  \caption{\label{fig:q18pb}Asian option}
  %\setlength{\abovecaptionskip}{-10pt}
  %\setlength{\belowcaptionskip}{2pt}
  \includegraphics[scale=0.75]{figures/fig4.png}
\end{figure}

\end{solution}


\noindent\rule{16cm}{0.4pt}
%///////////////////////////////////////////////////////////////////////
\textbf{Problem. 19} (\textit{Binomial(2)})\\
\begin{solution} \textbf{(a)} For the dynamic replication strategy, the stock and bond positions are given by:
\begin{equation}
  \Delta_t = \frac{p_{t+1,u} - p_{t+1, d}}{S_t(u-d)}
\end{equation}
\begin{equation}
  B_t = \frac{up_{t+1,d} - dp_{t+1, u}}{R(u-d)}
\end{equation}
where the value of American put option price at time $t$ is:
\begin{equation}
  p_t = \max\left\{\max(K-S_t, 0), S_t\Delta_t + B_t\right\}
\end{equation}
i.e. the maximum between replicating portfolio and the early exercise payoff. The trees of option price and dynamic replication portfolio are presented below. \\
The dynamic strategy is described by the right blue tree, with
\begin{equation}
  \begin{split}
    &\Delta_0 = -0.47502,~~~~B_0 = 25.4844\\
    &\Delta_{1,u} = 0,~~~~B_{1,u} = 0,~~~~\Delta_{1,d} = -1,~~~~B_{1,d} = 45.5437
  \end{split}
\end{equation}
And the price at time 0 is $c_0 = 1.7334$.\\

\begin{figure}[H]
  \centering
  \captionsetup{justification=centering}
  \caption{\label{fig:q18pa}American option (Dynamic Replication)}
  %\setlength{\abovecaptionskip}{-10pt}
  %\setlength{\belowcaptionskip}{2pt}
  \includegraphics[scale=0.8]{figures/fig5.png}
\end{figure}
~\\
\textbf{(b)} We first calculate the state prices:
\begin{equation}
  \pi_u = \frac{R-d}{R(u-d)} = 0.6066;~~~~  \pi_d = \frac{u-R}{R(u-d)} = 0.3643
\end{equation}
And the value of American put option price at time $t$ is:
\begin{equation}
  p_t = \max\left\{\max(K-S_t, 0), ~~p_{t+1, u}\pi_u + p_{t+1, d}\pi_d\right\}
\end{equation}
Plug the equations into the binomal tree, we obtain: $p_0 = 1.7334$.\\
~\\
\textbf{(c)} We first calculate the risk-neutral probabilities:
\begin{equation}
  q_u^{RN} = \frac{R-d}{u-d} = 0.6248;~~~~  q^{RN}_d = \frac{u-R}{u-d} = 0.3752
\end{equation}
And the value of American put option price at time $t$ is:
\begin{equation}
  p_t = \max\left\{\max(K-S_t, 0), ~~\tfrac{1}{R}\left(p_{t+1, u}q^{RN}_u + p_{t+1, d}q^{RN}_d\right)\right\}
\end{equation}
Plug the equations into the binomal tree, we obtain: $p_0 = 1.7334$.\\
~\\
\textbf{(d)} Now with the dividends, the value of American put option price at time $t$ is:
\begin{equation}
  p_t = \max\left\{\max(K-S_t^{[ex]}, 0), ~~p_{t+1, u}\pi_u + p_{t+1, d}\pi_d\right\}
\end{equation}
For the call:
\begin{equation}
  c_t = \max\left\{\max(S_t^{[cum]}-K, 0), ~~c_{t+1, u}\pi_u + c_{t+1, d}\pi_d\right\}
\end{equation}
See the trees below.
\begin{figure}[H]
  \centering
  \captionsetup{justification=centering}
  \caption{\label{fig:q19pb}American option (Dividend)}
  %\setlength{\abovecaptionskip}{-10pt}
  %\setlength{\belowcaptionskip}{2pt}
  \includegraphics[scale=0.8]{figures/fig6.png}
\end{figure}
We got $c_0 = 3.2598$, $p_0=2.6797$. The put call parity for European option is
$$
c_t-p_t = S_t-D_t-Ke^{-r(T-t)}
$$
Where $D_t$ is the cash dividend paid at time $t$. Clearly this equality does not hold for American option. Because with early exericise, this relation becomes an inequality:
$$
S_t - D_t - K \leq c_t - p_t \leq S_t - Ke^{-r(T-t)}
$$
\end{solution}

\noindent\rule{16cm}{0.4pt}
%///////////////////////////////////////////////////////////////////////
\textbf{Problem. 21} (\textit{Binomial/Capital Structure})\\
\begin{solution}
We first calcuate $u,d$, with CRR calibration:
\begin{equation}
  u = e^{\sigma \sqrt{t}}=1.2969;~~~~d = e^{-\sigma \sqrt{t}}=0.7711
\end{equation}
The state prices are
\begin{equation}
  \pi_u = \frac{R-d}{R(u-d)} = 0.4822;~~~~  \pi_d = \frac{u-R}{R(u-d)} = 0.4859
\end{equation}
Denote value of company, bond (prior to coupon payment), and equity as $V, B, E$ respectively. Let the face value and coupon rate of bond be $F$ and $c$. We have:
$$
B_t = \begin{cases}\min(V_T, F(1+c)) & t=T\\
\pi_u B_{t+1, u}+\pi_d B_{t+1, d} & 0\leq t< T
\end{cases}
$$
Hence
$$
E_t =\begin{cases}\max(0, V_T - F(1+c)) & t=T \\
V_t -B_t & 0\leq t< T
\end{cases}
$$
We assume that $\{V_t\}$ evolves by the binomial tree model. Denote $V_t^{[cum]}$ as cum dividend cum interest value, $V^{[ex]}$ as ex dividend ex interest value, we have:
\begin{equation}
  \begin{split}
     &V_t^{[ex]} = V_t^{[cum]} - D_t - Fc\\
     &V_{t,u}^{[cum]} = uV_{t}^{[ex]}\\
     &V_{t,d}^{[cum]} = dV_{t}^{[ex]}
  \end{split}
\end{equation}
See the binomial tree below for calculation details. We set $E_0=250$ as target and solve for desired coupon rate $c^*$. And we obtain:
$$
c^* = 6.435\%
$$

\begin{figure}[H]
  \centering
  \captionsetup{justification=centering}
  \caption{\label{fig:q21pa}Capital Structure Problem}
  %\setlength{\abovecaptionskip}{-10pt}
  %\setlength{\belowcaptionskip}{2pt}
  \includegraphics[scale=0.8]{figures/fig7.png}
\end{figure}

I.e. the company should sell the par-500 bond for annual coupon rate 6.435\%.\\
~\\
~\\
\begin{figure}[H]
  \centering
  \captionsetup{justification=centering}
  \caption{\label{fig:q21pb}Capital Structure Problem(b)}
  %\setlength{\abovecaptionskip}{-10pt}
  %\setlength{\belowcaptionskip}{2pt}
  \includegraphics[scale=0.8]{figures/fig8.png}
\end{figure}
~\\
\textbf{(b)} With same analysis, we now set $E_0 = 275$, $F=475$ and solve for $c$. For this one,
$$
c^* = 5.7158\%
$$
I.e. the company should sell the par-475 bond for annual coupon rate 5.7158\%.\\
~\\
\begin{figure}[H]
  \centering
  \captionsetup{justification=centering}
  \caption{\label{fig:q21pc}Capital Structure Problem(Putable Bond)}
  %\setlength{\abovecaptionskip}{-10pt}
  %\setlength{\belowcaptionskip}{2pt}
  \includegraphics[scale=0.8]{figures/fig9.png}
\end{figure}
~\\
\textbf{(c)} With same analysis, we now set $E_0 = 250$, $F=500$ and solve for $c$. For this one,
$$
B_t = \begin{cases}\min(V_T, F(1+c)) & t=T\\
\max\{K, ~~\pi_u B_{t+1, u}+\pi_d B_{t+1, d}\} & 0\leq t< T
\end{cases}
$$
where $K$ is the put price, $K=485$.\\
We obtain
$$
c^* = 4.3158\%
$$
I.e. the company should sell the par-500 putable bond for annual coupon rate 4.3158\%.\\
\end{solution}

\noindent\rule{16cm}{0.4pt}
%///////////////////////////////////////////////////////////////////////
\textbf{Problem. 31} (\textit{Black Scholes Hedging})\\
\begin{solution} Denote $\Delta_t, B_t$ the stock, bond position in replicating portfolio. $c_t$ the Black-Scholes option price, $\Pi_t$ the value of replicating portfolio, $\epsilon_t$ the tracing error, and $\delta t$ the hedging interval width. At time $t$, the incoming portfolio value:
$$
\Pi_t = \Delta_{t-\delta t} S_t + B_{t-\delta t} e^{r \delta t}
$$
$$
h_t := h(t, S_t) = \frac{\log(S_t/K)+(r+\tfrac{1}{2}\sigma^2)(T-t)}{\sigma \sqrt{T-t}}
$$
$$
\Delta_t = N(h_t);~~~~B_t = \Pi_t - S_t \Delta_t
$$
$$
c_t = S_tN(h_t) - Ke^{-t(T-t)}N(h_t-\sigma(T-t))
$$
where $N(z)$ is the standard normal cdf.
$$
\epsilon_t = \Pi_t - c_t
$$
The dynamic hedging process is presented in the table below. 

\begin{figure}[H]
  \centering
  \captionsetup{justification=centering}
  \caption{\label{fig:q31pa}Dynamic Delta Hedge}
  %\setlength{\abovecaptionskip}{-10pt}
  %\setlength{\belowcaptionskip}{2pt}
  \includegraphics[scale=0.68]{figures/fig10.png}
\end{figure}

In the final date $T$, we have
\begin{equation}
  \begin{split}
    &\Pi_T = 5.5963\\
    &c_T = 5.5010\\
    &\epsilon_T = 0.0953
  \end{split}
\end{equation}
\end{solution}




\end{document}
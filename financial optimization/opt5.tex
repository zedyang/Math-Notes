\documentclass[a4paper, 8pt]{article}    
\usepackage{geometry}       
\geometry{a4paper}
\geometry{margin=1in} 
\usepackage{paralist}
  \let\itemize\compactitem
  \let\enditemize\endcompactitem
  \let\enumerate\compactenum
  \let\endenumerate\endcompactenum
  \let\description\compactdesc
  \let\enddescription\endcompactdesc
  \pltopsep=\medskipamount
  \plitemsep=1pt
  \plparsep=1pt
\usepackage[english]{babel}
\usepackage[utf8]{inputenc}

\usepackage{bbm, bm}
\usepackage{amsmath, amssymb, amsthm, mathrsfs}
\usepackage{booktabs, tikz, array, eurosym}
\usepackage{float}
\renewcommand{\arraystretch}{1.4}
\newcolumntype{L}{>{\arraybackslash}m{10cm}}
\newcommand\indep{\protect\mathpalette{\protect\indeP}{\perp}}
\def\indeP#1#2{\mathrel{\rlap{$#1#2$}\mkern2mu{#1#2}}}

\pagestyle{headings}
\newcommand{\boxwidth}{430pt}

\theoremstyle{definition}
\newtheorem{problem}{Problem}

\newtheoremstyle{hSol}
  {1.0pt}% Space above
  {1.0pt}% Space below
  {}% bodyfont
  {}% indent
  {\bfseries}% thm head font
  {.}% punctuation after thm head
  { }% Space after thm head
  {}% thm head spec

\theoremstyle{hSol}
\newtheorem*{solution}{Solution}


\title{\textbf{Optimization Assignment 5}}
\author{Ze Yang~~~~(zey@andrew.cmu.edu)}

\begin{document}
\maketitle

\noindent\rule{16cm}{0.4pt}
%///////////////////////////////////////////////////////////////////////
\begin{problem} \textbf{(a)} Note that $Y$ and $\mu + \sigma Z$ are identically distributed, where $Z$ is standard normal.
\begin{equation}
	\begin{split}
		&\text{VaR}_{\alpha}(Y) = \text{inf}\{y: \mathbb{P}\left(Y\geq y\right) = 1- \alpha\} =\text{inf}\{y: \mathbb{P}\left(\frac{Y-\mu}{\sigma}< \frac{y-\mu}{\sigma}\right) = \alpha\}\\
		&\Rightarrow\quad \frac{\text{VaR}_{\alpha}(Y)-\mu}{\sigma}=\Phi^{-1}(\alpha) \quad \Rightarrow\quad  \text{VaR}_{\alpha}(Y)=\mu + \sigma \Phi^{-1}(\alpha) 
	\end{split}
\end{equation}
~\\
\begin{equation}
	\begin{split}
		\text{CVaR}_{\alpha}(Y) &= \mathbb{E}\left[Y | Y\geq\mu + \sigma \Phi^{-1}(\alpha) \right] =\mathbb{E}\left[\mu + \sigma Z | Z \geq\Phi^{-1}(\alpha) \right] = \frac{\mathbb{E}\left[\mu + \sigma Z ; Z \geq\Phi^{-1}(\alpha) \right]}{\mathbb{P}\left(Z \geq\Phi^{-1}(\alpha)\right)}\\
		&= \frac{1}{1- \alpha }\left(\mu(1- \alpha) + \sigma\int_{\Phi^{-1}(\alpha)}^{\infty} \frac{z}{\sqrt{2\pi}} e^{-z^2/2}dz\right)= \mu + \sigma\frac{\phi(\Phi^{-1}(\alpha))}{1- \alpha}
	\end{split}
\end{equation}
\textbf{(b)} Note that $Y$ and $e^{\mu + \sigma Z}$ are identically distributed, where $Z$ is standard normal.
\begin{equation}
	\begin{split}
		&\text{VaR}_{\alpha}(Y) = \text{inf}\{y: \mathbb{P}\left(Y\geq y\right) = 1- \alpha\} =\text{inf}\{y: \mathbb{P}\left(\frac{\log Y-\mu}{\sigma}< \frac{\log y-\mu}{\sigma}\right) = \alpha\}\\
		&\Rightarrow~~~\frac{\log \text{VaR}_{\alpha}(Y)-\mu}{\sigma}=\Phi^{-1}(\alpha) \quad \Rightarrow \quad\text{VaR}_{\alpha}(Y)= \exp\left(\mu + \sigma \Phi^{-1}(\alpha) \right)
	\end{split}
\end{equation}
~\\
\begin{equation}
	\begin{split}
		\text{CVaR}_{\alpha}(Y) &= \mathbb{E}\left[Y | Y\geq e^{\mu + \sigma \Phi^{-1}(\alpha)} \right] =\mathbb{E}\left[e^{\mu+\sigma Z} | Z \geq\Phi^{-1}(\alpha) \right] = \frac{\mathbb{E}\left[e^{\mu+\sigma Z} ; Z \geq\Phi^{-1}(\alpha) \right]}{\mathbb{P}\left(Z \geq\Phi^{-1}(\alpha)\right)}\\
		&= \frac{1}{1- \alpha }\left(\int_{\Phi^{-1}(\alpha)}^{\infty} e^{\mu + \frac{\sigma^2}{2}}\frac{1}{\sqrt{2\pi}} e^{\frac{-z^2+2\sigma z - \sigma^2}{2}}dz\right)= \frac{1}{1- \alpha}e^{\mu+\frac{\sigma^2}{2}}\Phi(\sigma-\Phi^{-1}(\alpha))
	\end{split}
\end{equation}
\end{problem}

\noindent\rule{16cm}{0.4pt}
%///////////////////////////////////////////////////////////////////////
\begin{problem} \textbf{(a)} We setup a linear program to solve VaR and cVaR, the results are:
\begin{equation}
	\text{VaR}_{0.9}(Y) = 40\qquad \text{cVaR}_{0.9}(Y) = 44.0693
\end{equation}
\textbf{(b)} We achieve the cVaR optimization by adding one more free variable to the program above: the portfolio weights $\bm{x}$. The optimal portfolio is
\begin{equation}
	\bm{x}^* = \begin{pmatrix}
		0.875 & 0.125 & 0
	\end{pmatrix}^{\top}
\end{equation}
And the corresponding portfolio VaR and cVaR are:
\begin{equation}
	\text{VaR}_{0.9}(Y) = 8.75\qquad \text{cVaR}_{0.9}(Y) = 24.5853
\end{equation}
\end{problem} 
\newpage
\noindent\rule{16cm}{0.4pt}
%///////////////////////////////////////////////////////////////////////
\begin{problem} 
\end{problem}
\begin{equation}
	\begin{split}
		W_{t+1} = \begin{cases}
		W_t(1+f_1+f_2) & \text{If win both gambles at time }t\\
		W_t(1+f_1 - f_2) & \text{If win the first and lose the second gamble at time }t\\
		W_t(1-f_1 + f_2) & \text{If win the second and lose the first gamble at time }t\\
		W_t(1-f_1 - f_2) & \text{If lose both gambles }t\\
		\end{cases}
	\end{split}
\end{equation}
Therefore
\begin{equation}
	W_n = W_0 (1+f_1+f_2)^{\frac{k_{3}}{n}} (1-f_1+f_2)^{\frac{k_{2}}{n}} (1+f_1-f_2)^{\frac{k_{1}}{n}}(1-f_1-f_2)^{\frac{n-k_{1}-k_2-k_3}{n}}
\end{equation}
Let $g = \log(W_n/W_0)^{1/n}$, $x=f_1+f_2, y=f_1-f_2$ we have:
\begin{equation}
	\begin{split}
		\mathbb{E}\left[ g\right] &= p_1p_2 \log(1+f_1+f_2) + (1-p_1)p_2 \log(1-f_1+f_2) + p_1(1-p_2) \log(1+f_1-f_2) \\
		&~~~+ (1-p_1)(1-p_2)\log(1-f_1 - f_2) \\
		&=p_1p_2 \log(1+x) + (1-p_1)p_2 \log(1-y) + p_1(1-p_2) \log(1+y) \\
		&~~~+ (1-p_1)(1-p_2)\log(1-x) \\
	\end{split}
\end{equation}
First order conditions:
\begin{equation}
	\begin{split}
		&\nabla_{x} \mathbb{E}\left[g\right] = \frac{p_1p_2}{1+x} - \frac{(1-p_1)(1-p_2)}{1-x}=0\quad \Rightarrow\quad x=\frac{p_1+p_2 - 1}{2p_1p_2 -p_1 - p_2 + 1} \\
		&\nabla_y \mathbb{E}\left[g\right] = \frac{p_1(1-p_2)}{1+y} - \frac{(1-p_1)p_2}{1-y} =0\quad \Rightarrow\quad y = \frac{p_1 - p_2}{-2p_1p_2 + p_1 + p_2}
	\end{split}
\end{equation}
\begin{equation}
	\begin{split}
		f_1 = \frac{x+y}{2} = \frac{1}{2}\left(\frac{p_1+p_2 - 1}{2p_1p_2 -p_1 - p_2 + 1}+\frac{p_1 - p_2}{-2p_1p_2 + p_1 + p_2}\right) \\
		f_2 = \frac{x-y}{2} = \frac{1}{2}\left(\frac{p_1+p_2 - 1}{2p_1p_2 -p_1 - p_2 + 1}-\frac{p_1 - p_2}{-2p_1p_2 + p_1 + p_2}\right)
	\end{split}
\end{equation}
In this problem, we have $p_1 = p_2 = p$, therefore
\begin{equation}
	f_1 = f_2 = \frac{2p - 1}{2(2p^2-2p + 1)} = \frac{p^2 - (1-p)^2}{2(p^2 + (1-p)^2)}
\end{equation}

\end{document}
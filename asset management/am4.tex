
% Default to the notebook output style

    


% Inherit from the specified cell style.




    
\documentclass[11pt]{article}

    
    
    \usepackage[T1]{fontenc}
    % Nicer default font (+ math font) than Computer Modern for most use cases
    %\usepackage{mathpazo}

    % Basic figure setup, for now with no caption control since it's done
    % automatically by Pandoc (which extracts ![](path) syntax from Markdown).
    \usepackage{graphicx,bm}
    % We will generate all images so they have a width \maxwidth. This means
    % that they will get their normal width if they fit onto the page, but
    % are scaled down if they would overflow the margins.
    \makeatletter
    \def\maxwidth{\ifdim\Gin@nat@width>\linewidth\linewidth
    \else\Gin@nat@width\fi}
    \makeatother
    \let\Oldincludegraphics\includegraphics
    % Set max figure width to be 80% of text width, for now hardcoded.
    \renewcommand{\includegraphics}[1]{\Oldincludegraphics[width=.8\maxwidth]{#1}}
    % Ensure that by default, figures have no caption (until we provide a
    % proper Figure object with a Caption API and a way to capture that
    % in the conversion process - todo).
    \usepackage{caption}
    \DeclareCaptionLabelFormat{nolabel}{}
    \captionsetup{labelformat=nolabel}

    \usepackage{adjustbox} % Used to constrain images to a maximum size 
    \usepackage{xcolor} % Allow colors to be defined
    \usepackage{enumerate} % Needed for markdown enumerations to work
    \usepackage{geometry} % Used to adjust the document margins
    \usepackage{amsmath} % Equations
    \usepackage{amssymb} % Equations
    \usepackage{textcomp} % defines textquotesingle
    % Hack from http://tex.stackexchange.com/a/47451/13684:
    \AtBeginDocument{%
        \def\PYZsq{\textquotesingle}% Upright quotes in Pygmentized code
    }
    \usepackage{upquote} % Upright quotes for verbatim code
    \usepackage{eurosym} % defines \euro
    \usepackage[mathletters]{ucs} % Extended unicode (utf-8) support
    \usepackage[utf8x]{inputenc} % Allow utf-8 characters in the tex document
    \usepackage{fancyvrb} % verbatim replacement that allows latex
    \usepackage{grffile} % extends the file name processing of package graphics 
                         % to support a larger range 
    % The hyperref package gives us a pdf with properly built
    % internal navigation ('pdf bookmarks' for the table of contents,
    % internal cross-reference links, web links for URLs, etc.)
    \usepackage{hyperref}
    \usepackage{longtable} % longtable support required by pandoc >1.10
    \usepackage{booktabs}  % table support for pandoc > 1.12.2
    \usepackage[inline]{enumitem} % IRkernel/repr support (it uses the enumerate* environment)
    \usepackage[normalem]{ulem} % ulem is needed to support strikethroughs (\sout)
                                % normalem makes italics be italics, not underlines
    

    
    
    % Colors for the hyperref package
    \definecolor{urlcolor}{rgb}{0,.145,.698}
    \definecolor{linkcolor}{rgb}{.71,0.21,0.01}
    \definecolor{citecolor}{rgb}{.12,.54,.11}

    % ANSI colors
    \definecolor{ansi-black}{HTML}{3E424D}
    \definecolor{ansi-black-intense}{HTML}{282C36}
    \definecolor{ansi-red}{HTML}{E75C58}
    \definecolor{ansi-red-intense}{HTML}{B22B31}
    \definecolor{ansi-green}{HTML}{00A250}
    \definecolor{ansi-green-intense}{HTML}{007427}
    \definecolor{ansi-yellow}{HTML}{DDB62B}
    \definecolor{ansi-yellow-intense}{HTML}{B27D12}
    \definecolor{ansi-blue}{HTML}{208FFB}
    \definecolor{ansi-blue-intense}{HTML}{0065CA}
    \definecolor{ansi-magenta}{HTML}{D160C4}
    \definecolor{ansi-magenta-intense}{HTML}{A03196}
    \definecolor{ansi-cyan}{HTML}{60C6C8}
    \definecolor{ansi-cyan-intense}{HTML}{258F8F}
    \definecolor{ansi-white}{HTML}{C5C1B4}
    \definecolor{ansi-white-intense}{HTML}{A1A6B2}

    % commands and environments needed by pandoc snippets
    % extracted from the output of `pandoc -s`
    \providecommand{\tightlist}{%
      \setlength{\itemsep}{0pt}\setlength{\parskip}{0pt}}
    \DefineVerbatimEnvironment{Highlighting}{Verbatim}{commandchars=\\\{\}}
    % Add ',fontsize=\small' for more characters per line
    \newenvironment{Shaded}{}{}
    \newcommand{\KeywordTok}[1]{\textcolor[rgb]{0.00,0.44,0.13}{\textbf{{#1}}}}
    \newcommand{\DataTypeTok}[1]{\textcolor[rgb]{0.56,0.13,0.00}{{#1}}}
    \newcommand{\DecValTok}[1]{\textcolor[rgb]{0.25,0.63,0.44}{{#1}}}
    \newcommand{\BaseNTok}[1]{\textcolor[rgb]{0.25,0.63,0.44}{{#1}}}
    \newcommand{\FloatTok}[1]{\textcolor[rgb]{0.25,0.63,0.44}{{#1}}}
    \newcommand{\CharTok}[1]{\textcolor[rgb]{0.25,0.44,0.63}{{#1}}}
    \newcommand{\StringTok}[1]{\textcolor[rgb]{0.25,0.44,0.63}{{#1}}}
    \newcommand{\CommentTok}[1]{\textcolor[rgb]{0.38,0.63,0.69}{\textit{{#1}}}}
    \newcommand{\OtherTok}[1]{\textcolor[rgb]{0.00,0.44,0.13}{{#1}}}
    \newcommand{\AlertTok}[1]{\textcolor[rgb]{1.00,0.00,0.00}{\textbf{{#1}}}}
    \newcommand{\FunctionTok}[1]{\textcolor[rgb]{0.02,0.16,0.49}{{#1}}}
    \newcommand{\RegionMarkerTok}[1]{{#1}}
    \newcommand{\ErrorTok}[1]{\textcolor[rgb]{1.00,0.00,0.00}{\textbf{{#1}}}}
    \newcommand{\NormalTok}[1]{{#1}}
    
    % Additional commands for more recent versions of Pandoc
    \newcommand{\ConstantTok}[1]{\textcolor[rgb]{0.53,0.00,0.00}{{#1}}}
    \newcommand{\SpecialCharTok}[1]{\textcolor[rgb]{0.25,0.44,0.63}{{#1}}}
    \newcommand{\VerbatimStringTok}[1]{\textcolor[rgb]{0.25,0.44,0.63}{{#1}}}
    \newcommand{\SpecialStringTok}[1]{\textcolor[rgb]{0.73,0.40,0.53}{{#1}}}
    \newcommand{\ImportTok}[1]{{#1}}
    \newcommand{\DocumentationTok}[1]{\textcolor[rgb]{0.73,0.13,0.13}{\textit{{#1}}}}
    \newcommand{\AnnotationTok}[1]{\textcolor[rgb]{0.38,0.63,0.69}{\textbf{\textit{{#1}}}}}
    \newcommand{\CommentVarTok}[1]{\textcolor[rgb]{0.38,0.63,0.69}{\textbf{\textit{{#1}}}}}
    \newcommand{\VariableTok}[1]{\textcolor[rgb]{0.10,0.09,0.49}{{#1}}}
    \newcommand{\ControlFlowTok}[1]{\textcolor[rgb]{0.00,0.44,0.13}{\textbf{{#1}}}}
    \newcommand{\OperatorTok}[1]{\textcolor[rgb]{0.40,0.40,0.40}{{#1}}}
    \newcommand{\BuiltInTok}[1]{{#1}}
    \newcommand{\ExtensionTok}[1]{{#1}}
    \newcommand{\PreprocessorTok}[1]{\textcolor[rgb]{0.74,0.48,0.00}{{#1}}}
    \newcommand{\AttributeTok}[1]{\textcolor[rgb]{0.49,0.56,0.16}{{#1}}}
    \newcommand{\InformationTok}[1]{\textcolor[rgb]{0.38,0.63,0.69}{\textbf{\textit{{#1}}}}}
    \newcommand{\WarningTok}[1]{\textcolor[rgb]{0.38,0.63,0.69}{\textbf{\textit{{#1}}}}}
    
    
    % Define a nice break command that doesn't care if a line doesn't already
    % exist.
    \def\br{\hspace*{\fill} \\* }
    % Math Jax compatability definitions
    \def\gt{>}
    \def\lt{<}
    % Document parameters
    \title{Asset Management HW4}
    \author{Ze Yang}
    
    
    

    % Pygments definitions
    
\makeatletter
\def\PY@reset{\let\PY@it=\relax \let\PY@bf=\relax%
    \let\PY@ul=\relax \let\PY@tc=\relax%
    \let\PY@bc=\relax \let\PY@ff=\relax}
\def\PY@tok#1{\csname PY@tok@#1\endcsname}
\def\PY@toks#1+{\ifx\relax#1\empty\else%
    \PY@tok{#1}\expandafter\PY@toks\fi}
\def\PY@do#1{\PY@bc{\PY@tc{\PY@ul{%
    \PY@it{\PY@bf{\PY@ff{#1}}}}}}}
\def\PY#1#2{\PY@reset\PY@toks#1+\relax+\PY@do{#2}}

\expandafter\def\csname PY@tok@w\endcsname{\def\PY@tc##1{\textcolor[rgb]{0.73,0.73,0.73}{##1}}}
\expandafter\def\csname PY@tok@c\endcsname{\let\PY@it=\textit\def\PY@tc##1{\textcolor[rgb]{0.25,0.50,0.50}{##1}}}
\expandafter\def\csname PY@tok@cp\endcsname{\def\PY@tc##1{\textcolor[rgb]{0.74,0.48,0.00}{##1}}}
\expandafter\def\csname PY@tok@k\endcsname{\let\PY@bf=\textbf\def\PY@tc##1{\textcolor[rgb]{0.00,0.50,0.00}{##1}}}
\expandafter\def\csname PY@tok@kp\endcsname{\def\PY@tc##1{\textcolor[rgb]{0.00,0.50,0.00}{##1}}}
\expandafter\def\csname PY@tok@kt\endcsname{\def\PY@tc##1{\textcolor[rgb]{0.69,0.00,0.25}{##1}}}
\expandafter\def\csname PY@tok@o\endcsname{\def\PY@tc##1{\textcolor[rgb]{0.40,0.40,0.40}{##1}}}
\expandafter\def\csname PY@tok@ow\endcsname{\let\PY@bf=\textbf\def\PY@tc##1{\textcolor[rgb]{0.67,0.13,1.00}{##1}}}
\expandafter\def\csname PY@tok@nb\endcsname{\def\PY@tc##1{\textcolor[rgb]{0.00,0.50,0.00}{##1}}}
\expandafter\def\csname PY@tok@nf\endcsname{\def\PY@tc##1{\textcolor[rgb]{0.00,0.00,1.00}{##1}}}
\expandafter\def\csname PY@tok@nc\endcsname{\let\PY@bf=\textbf\def\PY@tc##1{\textcolor[rgb]{0.00,0.00,1.00}{##1}}}
\expandafter\def\csname PY@tok@nn\endcsname{\let\PY@bf=\textbf\def\PY@tc##1{\textcolor[rgb]{0.00,0.00,1.00}{##1}}}
\expandafter\def\csname PY@tok@ne\endcsname{\let\PY@bf=\textbf\def\PY@tc##1{\textcolor[rgb]{0.82,0.25,0.23}{##1}}}
\expandafter\def\csname PY@tok@nv\endcsname{\def\PY@tc##1{\textcolor[rgb]{0.10,0.09,0.49}{##1}}}
\expandafter\def\csname PY@tok@no\endcsname{\def\PY@tc##1{\textcolor[rgb]{0.53,0.00,0.00}{##1}}}
\expandafter\def\csname PY@tok@nl\endcsname{\def\PY@tc##1{\textcolor[rgb]{0.63,0.63,0.00}{##1}}}
\expandafter\def\csname PY@tok@ni\endcsname{\let\PY@bf=\textbf\def\PY@tc##1{\textcolor[rgb]{0.60,0.60,0.60}{##1}}}
\expandafter\def\csname PY@tok@na\endcsname{\def\PY@tc##1{\textcolor[rgb]{0.49,0.56,0.16}{##1}}}
\expandafter\def\csname PY@tok@nt\endcsname{\let\PY@bf=\textbf\def\PY@tc##1{\textcolor[rgb]{0.00,0.50,0.00}{##1}}}
\expandafter\def\csname PY@tok@nd\endcsname{\def\PY@tc##1{\textcolor[rgb]{0.67,0.13,1.00}{##1}}}
\expandafter\def\csname PY@tok@s\endcsname{\def\PY@tc##1{\textcolor[rgb]{0.73,0.13,0.13}{##1}}}
\expandafter\def\csname PY@tok@sd\endcsname{\let\PY@it=\textit\def\PY@tc##1{\textcolor[rgb]{0.73,0.13,0.13}{##1}}}
\expandafter\def\csname PY@tok@si\endcsname{\let\PY@bf=\textbf\def\PY@tc##1{\textcolor[rgb]{0.73,0.40,0.53}{##1}}}
\expandafter\def\csname PY@tok@se\endcsname{\let\PY@bf=\textbf\def\PY@tc##1{\textcolor[rgb]{0.73,0.40,0.13}{##1}}}
\expandafter\def\csname PY@tok@sr\endcsname{\def\PY@tc##1{\textcolor[rgb]{0.73,0.40,0.53}{##1}}}
\expandafter\def\csname PY@tok@ss\endcsname{\def\PY@tc##1{\textcolor[rgb]{0.10,0.09,0.49}{##1}}}
\expandafter\def\csname PY@tok@sx\endcsname{\def\PY@tc##1{\textcolor[rgb]{0.00,0.50,0.00}{##1}}}
\expandafter\def\csname PY@tok@m\endcsname{\def\PY@tc##1{\textcolor[rgb]{0.40,0.40,0.40}{##1}}}
\expandafter\def\csname PY@tok@gh\endcsname{\let\PY@bf=\textbf\def\PY@tc##1{\textcolor[rgb]{0.00,0.00,0.50}{##1}}}
\expandafter\def\csname PY@tok@gu\endcsname{\let\PY@bf=\textbf\def\PY@tc##1{\textcolor[rgb]{0.50,0.00,0.50}{##1}}}
\expandafter\def\csname PY@tok@gd\endcsname{\def\PY@tc##1{\textcolor[rgb]{0.63,0.00,0.00}{##1}}}
\expandafter\def\csname PY@tok@gi\endcsname{\def\PY@tc##1{\textcolor[rgb]{0.00,0.63,0.00}{##1}}}
\expandafter\def\csname PY@tok@gr\endcsname{\def\PY@tc##1{\textcolor[rgb]{1.00,0.00,0.00}{##1}}}
\expandafter\def\csname PY@tok@ge\endcsname{\let\PY@it=\textit}
\expandafter\def\csname PY@tok@gs\endcsname{\let\PY@bf=\textbf}
\expandafter\def\csname PY@tok@gp\endcsname{\let\PY@bf=\textbf\def\PY@tc##1{\textcolor[rgb]{0.00,0.00,0.50}{##1}}}
\expandafter\def\csname PY@tok@go\endcsname{\def\PY@tc##1{\textcolor[rgb]{0.53,0.53,0.53}{##1}}}
\expandafter\def\csname PY@tok@gt\endcsname{\def\PY@tc##1{\textcolor[rgb]{0.00,0.27,0.87}{##1}}}
\expandafter\def\csname PY@tok@err\endcsname{\def\PY@bc##1{\setlength{\fboxsep}{0pt}\fcolorbox[rgb]{1.00,0.00,0.00}{1,1,1}{\strut ##1}}}
\expandafter\def\csname PY@tok@kc\endcsname{\let\PY@bf=\textbf\def\PY@tc##1{\textcolor[rgb]{0.00,0.50,0.00}{##1}}}
\expandafter\def\csname PY@tok@kd\endcsname{\let\PY@bf=\textbf\def\PY@tc##1{\textcolor[rgb]{0.00,0.50,0.00}{##1}}}
\expandafter\def\csname PY@tok@kn\endcsname{\let\PY@bf=\textbf\def\PY@tc##1{\textcolor[rgb]{0.00,0.50,0.00}{##1}}}
\expandafter\def\csname PY@tok@kr\endcsname{\let\PY@bf=\textbf\def\PY@tc##1{\textcolor[rgb]{0.00,0.50,0.00}{##1}}}
\expandafter\def\csname PY@tok@bp\endcsname{\def\PY@tc##1{\textcolor[rgb]{0.00,0.50,0.00}{##1}}}
\expandafter\def\csname PY@tok@fm\endcsname{\def\PY@tc##1{\textcolor[rgb]{0.00,0.00,1.00}{##1}}}
\expandafter\def\csname PY@tok@vc\endcsname{\def\PY@tc##1{\textcolor[rgb]{0.10,0.09,0.49}{##1}}}
\expandafter\def\csname PY@tok@vg\endcsname{\def\PY@tc##1{\textcolor[rgb]{0.10,0.09,0.49}{##1}}}
\expandafter\def\csname PY@tok@vi\endcsname{\def\PY@tc##1{\textcolor[rgb]{0.10,0.09,0.49}{##1}}}
\expandafter\def\csname PY@tok@vm\endcsname{\def\PY@tc##1{\textcolor[rgb]{0.10,0.09,0.49}{##1}}}
\expandafter\def\csname PY@tok@sa\endcsname{\def\PY@tc##1{\textcolor[rgb]{0.73,0.13,0.13}{##1}}}
\expandafter\def\csname PY@tok@sb\endcsname{\def\PY@tc##1{\textcolor[rgb]{0.73,0.13,0.13}{##1}}}
\expandafter\def\csname PY@tok@sc\endcsname{\def\PY@tc##1{\textcolor[rgb]{0.73,0.13,0.13}{##1}}}
\expandafter\def\csname PY@tok@dl\endcsname{\def\PY@tc##1{\textcolor[rgb]{0.73,0.13,0.13}{##1}}}
\expandafter\def\csname PY@tok@s2\endcsname{\def\PY@tc##1{\textcolor[rgb]{0.73,0.13,0.13}{##1}}}
\expandafter\def\csname PY@tok@sh\endcsname{\def\PY@tc##1{\textcolor[rgb]{0.73,0.13,0.13}{##1}}}
\expandafter\def\csname PY@tok@s1\endcsname{\def\PY@tc##1{\textcolor[rgb]{0.73,0.13,0.13}{##1}}}
\expandafter\def\csname PY@tok@mb\endcsname{\def\PY@tc##1{\textcolor[rgb]{0.40,0.40,0.40}{##1}}}
\expandafter\def\csname PY@tok@mf\endcsname{\def\PY@tc##1{\textcolor[rgb]{0.40,0.40,0.40}{##1}}}
\expandafter\def\csname PY@tok@mh\endcsname{\def\PY@tc##1{\textcolor[rgb]{0.40,0.40,0.40}{##1}}}
\expandafter\def\csname PY@tok@mi\endcsname{\def\PY@tc##1{\textcolor[rgb]{0.40,0.40,0.40}{##1}}}
\expandafter\def\csname PY@tok@il\endcsname{\def\PY@tc##1{\textcolor[rgb]{0.40,0.40,0.40}{##1}}}
\expandafter\def\csname PY@tok@mo\endcsname{\def\PY@tc##1{\textcolor[rgb]{0.40,0.40,0.40}{##1}}}
\expandafter\def\csname PY@tok@ch\endcsname{\let\PY@it=\textit\def\PY@tc##1{\textcolor[rgb]{0.25,0.50,0.50}{##1}}}
\expandafter\def\csname PY@tok@cm\endcsname{\let\PY@it=\textit\def\PY@tc##1{\textcolor[rgb]{0.25,0.50,0.50}{##1}}}
\expandafter\def\csname PY@tok@cpf\endcsname{\let\PY@it=\textit\def\PY@tc##1{\textcolor[rgb]{0.25,0.50,0.50}{##1}}}
\expandafter\def\csname PY@tok@c1\endcsname{\let\PY@it=\textit\def\PY@tc##1{\textcolor[rgb]{0.25,0.50,0.50}{##1}}}
\expandafter\def\csname PY@tok@cs\endcsname{\let\PY@it=\textit\def\PY@tc##1{\textcolor[rgb]{0.25,0.50,0.50}{##1}}}

\def\PYZbs{\char`\\}
\def\PYZus{\char`\_}
\def\PYZob{\char`\{}
\def\PYZcb{\char`\}}
\def\PYZca{\char`\^}
\def\PYZam{\char`\&}
\def\PYZlt{\char`\<}
\def\PYZgt{\char`\>}
\def\PYZsh{\char`\#}
\def\PYZpc{\char`\%}
\def\PYZdl{\char`\$}
\def\PYZhy{\char`\-}
\def\PYZsq{\char`\'}
\def\PYZdq{\char`\"}
\def\PYZti{\char`\~}
% for compatibility with earlier versions
\def\PYZat{@}
\def\PYZlb{[}
\def\PYZrb{]}
\makeatother


    % Exact colors from NB
    \definecolor{incolor}{rgb}{0.0, 0.0, 0.5}
    \definecolor{outcolor}{rgb}{0.545, 0.0, 0.0}



    
    % Prevent overflowing lines due to hard-to-break entities
    \sloppy 
    % Setup hyperref package
    \hypersetup{
      breaklinks=true,  % so long urls are correctly broken across lines
      colorlinks=true,
      urlcolor=urlcolor,
      linkcolor=linkcolor,
      citecolor=citecolor,
      }
    % Slightly bigger margins than the latex defaults
    
    \geometry{verbose,tmargin=1in,bmargin=1in,lmargin=1in,rmargin=1in}
    
    

    \begin{document}
    
    
    \maketitle
    
    \section*{Problem 2}
\subsection*{2-(a)}
\begin{equation}
 \begin{split}
   &\mathcal{L}(\bm{h}, \bm{\lambda}) = \bm{h}^{\top} \bm{h} + \lambda_1(\bm{a}^{\top} \bm{h} - 1) + \lambda_2 \bm{1}^{\top} \bm{h} \\
   \Rightarrow \quad &
   \nabla_{\bm{h}} \mathcal{L} = 2 \bm{h} + \lambda_1 \bm{a} + \lambda_2 \bm{1} = 0 \\
   \Rightarrow \quad &
   \begin{cases}
   \bm{h} = -\frac{1}{2} (\lambda_1 \bm{a} + \lambda_2 \bm{1}) \\
   \lambda_1 \bm{a}^{\top} \bm{a} +  \lambda_2 \bm{a}^{\top} \bm{1} = -2\\
   \lambda_1 \bm{1}^{\top} \bm{a} + \lambda_2 n = 0
   \end{cases} ~~\Rightarrow~~\begin{cases}
   \lambda_1 = \frac{2}{(\bm{1}^{\top} \bm{a})^2/n - \bm{a}^{\top} \bm{a}}\\
   \lambda_2 = \frac{2\cdot\bm{1}^{\top} \bm{a}/n}{\bm{a}^{\top} \bm{a} - (\bm{1}^{\top} \bm{a})^2/n}
   \end{cases}\\
   \Rightarrow\quad& \bm{h}^* = \frac{1}{\bm{a}^{\top} \bm{a} - (\bm{1}^{\top} \bm{a})^2/n} \bm{a} - \frac{\bm{1}^{\top} \bm{a}/n}{\bm{a}^{\top} \bm{a} - (\bm{1}^{\top} \bm{a})^2/n} \bm{1} \\
   &~~~=  \frac{1}{n \sigma_{CS}^2(\bm{a})} \left(\bm{a} - \mu_{CS}(\bm{a}) \bm{1}\right) \\
   &~~~= \frac{1}{n \sigma_{CS}(\bm{a})} \bm{z}
 \end{split}
 \end{equation} 
 Hence $C=\frac{1}{n \sigma_{CS}(\bm{a})}$ for general solution. $C=1/n$ for zero cross-sectional mean and unit cross-sectional variance. In this special case: $\bm{h}^* = \frac{1}{n} \bm{a}$.

 \subsection*{2-(b)}
Let $\bm{b}_1^\top = (\bm{1}_k^{\top}~~ \bm{0}_{n-k}^{\top})$, $\bm{b}_2^\top = (\bm{0}_{k}^{\top}~~\bm{1}_{n-k}^{\top})$, i.e. $\bm{B} = (\bm{b}_1 ~~\bm{b}_2)$. The neutrality constraints are: $\bm{b}^{\top}_1 \bm{h} = \bm{b}^{\top}_2 \bm{h} = 0$. One of them is redundant since we already have constraint $\bm{1}^{\top} \bm{h}=0$, and $\bm{b}_1 + \bm{b}_2 = \bm{1}$. Let dual variable $\bm{\lambda} = (\lambda_1~~\lambda_2~~\lambda_3)^{\top}$. The KKT condition changes to:
\begin{equation}
  \begin{split}
    &\nabla_{\bm{h}} \mathcal{L} = 2 \bm{h} + \lambda_1 \bm{a} + \lambda_2 \bm{1} + \lambda_3 \bm{b}_1 = 0 \\
   \Rightarrow \quad &
   \begin{cases}
   \bm{h} = -\frac{1}{2} \begin{pmatrix}
     \bm{a} & \bm{1} & \bm{b}_1 
   \end{pmatrix} \bm{\lambda}\\
   \begin{pmatrix}
     \bm{a}^{\top} \bm{a} & \bm{a}^{\top} \bm{1} &  \bm{a}^{\top}\bm{b}_1  \\[1pt]
     \bm{1}^{\top} \bm{a} & n & k  \\[1pt]
     \bm{b}_1^{\top} \bm{a} & k & k  \\[1pt]
   \end{pmatrix} \bm{\lambda} = \begin{pmatrix}
     -2 \\
     0 \\
    0
   \end{pmatrix}
   \end{cases}\\
   \text{(Gaussian Elimination)}~~\Rightarrow \quad &\begin{pmatrix}
     \bm{a}^{\top} \bm{a} - \frac{(\bm{a}^{\top} \bm{b}_1)^2}{k}-\frac{(\bm{a}^{\top} \bm{b}_2)^2}{n-k} & 0 &  0  \\[1pt]
     \bm{b}_2^{\top} \bm{a} & n-k & 0  \\[1pt]
     \bm{b}_1^{\top} \bm{a} & k & k  \\[1pt]
   \end{pmatrix} \bm{\lambda} = \begin{pmatrix}
     -2 \\
     0 \\
    0
   \end{pmatrix}\\
   \Rightarrow \quad & \begin{pmatrix}
     \lambda_1 \\
     \lambda_2 \\
     \lambda_3
   \end{pmatrix} = \begin{pmatrix}
     \frac{-2}{\bm{a}^{\top} \bm{a} - (\bm{a}^{\top} \bm{b}_1)^2/k-(\bm{a}^{\top} \bm{b}_2)^2/(n-k)} \\[3pt]
     -\frac{\bm{a}^{\top} \bm{b}_2}{n-k} \lambda_1\\[3pt]
     \left[\frac{\bm{a}^{\top} \bm{b}_2}{n-k} - \frac{\bm{a}^{\top} \bm{b}_1}{k}\right] \lambda_1\\[3pt]
   \end{pmatrix}
  \end{split}
\end{equation}
Therefore, 
\begin{equation}
  \begin{split}
    \bm{h}^* &= -\frac{1}{2}\lambda_1 \left(\bm{a} -\frac{\bm{a}^{\top} \bm{b}_2}{n-k}\bm{1} + \left(\frac{\bm{a}^{\top} \bm{b}_2}{n-k} - \frac{\bm{a}^{\top} \bm{b}_1}{k}\right) \bm{b}_1\right) \\
    &=-\frac{1}{2}\lambda_1 \left(\bm{a}  - \frac{\bm{a}^{\top} \bm{b}_1}{k} \bm{b}_1  -\frac{\bm{a}^{\top} \bm{b}_2}{n-k}\bm{b}_2\right) \\
    &=-\frac{1}{2}\lambda_1 \left(\bm{a}  - \frac{\sum_{j=1}^k a_j}{k} \bm{b}_1  -\frac{\sum_{j=k+1}^n a_j}{n-k}\bm{b}_2\right) \\
    &= -\frac{1}{2}\lambda_1 \widehat{\bm{a}} = C \widehat{\bm{a}}
  \end{split}
\end{equation}
Where $C = -\frac{1}{2}\lambda_1 = 1\Big/\left[\bm{a}^{\top} \bm{a} - \frac{(\bm{a}^{\top} \bm{b}_1)^2}{k}-\frac{(\bm{a}^{\top} \bm{b}_2)^2}{n-k}\right]=1\Big/\left[\sum_{j=1}^n a_j^2 - \frac{(\sum_{j=1}^k a_j)^2}{k}-\frac{(\sum_{j=k+1}^n a_j)^2}{n-k}\right]$

 \subsection*{2-(c)}
 %By the higher level assumption we knew that $\bm{1}^{\top} \bm{a} = \bm{1}^{\top} \bm{b} = 0$.
\begin{equation}
  \bm{a}^{\top} \bm{h}^{SMB} = \sum_{k=1}^4\frac{1}{2|G_k|}\sum_{i\in G_k}a_i \text{sgn}(a_i)  = \sum_{k=1}^4\frac{1}{2|G_k|}\sum_{i\in G_k}|a_i|  = \sum_{k=1}^4 \frac{1}{2} = 2
\end{equation}
\begin{equation}
  \bm{1}^{\top} \bm{h}^{SMB} = \sum_{k=1}^4\frac{1}{2|G_k|}\sum_{i\in G_k} \text{sgn}(a_i)  = \frac{|S|}{2|S|} - \frac{|B|}{2|B|} = 0
\end{equation}
\begin{equation}
  \bm{b}^{\top} \bm{h}^{SMB} = \sum_{k=1}^4\frac{1}{2|G_k|}\sum_{i\in G_k}b_i \text{sgn}(a_i)= \sum_{k=1}^4\frac{1}{2|G_k|}\sum_{i\in G_k} \text{sgn}(a_ib_i) = \frac{|SH|}{2|SH|} + \frac{|BL|}{2|BL|} - \frac{|SL|}{2|SL|} - \frac{|BH|}{2|BH|} = 0
\end{equation}
\subsection*{2-(d)}
We know from (c) that $\bm{a}^{\top} \bm{a}=\bm{b}^{\top} \bm{b}=n$, $\bm{1}^{\top} \bm{a}=|S|-|B|$, $\bm{1}^{\top} \bm{b} = |H|-|L|$, and $\bm{a}^{\top} \bm{b} = |SH|+|BL|-|SL|-|BH|$. Denote $A:=|SL|,B:=|SH|,C:=|BL|,D:=|BH|$. The KKT condition changes to:
\begin{equation}
  \begin{split}
    &\nabla_{\bm{h}} \mathcal{L} = 2 \bm{h} + \lambda_1 \bm{a} + \lambda_2 \bm{1} + \lambda_3 \bm{b} = 0 \\
   \Rightarrow \quad &
   \begin{cases}
   \bm{h} = -\frac{1}{2} \begin{pmatrix}
     \bm{a} & \bm{1} & \bm{b} 
   \end{pmatrix} \bm{\lambda}\\
   \begin{pmatrix}
     \bm{a}^{\top} \bm{a} & \bm{a}^{\top} \bm{1} &  \bm{a}^{\top}\bm{b}  \\[1pt]
     \bm{1}^{\top} \bm{a} & n & \bm{1}^{\top} \bm{b}  \\[1pt]
     \bm{b}^{\top} \bm{a} & \bm{b}^{\top} \bm{1} & \bm{b}^{\top} \bm{b}  \\[1pt]
   \end{pmatrix} \bm{\lambda} = \begin{pmatrix}
     -2 \\
     0 \\
    0
   \end{pmatrix}
   \end{cases}\\
   \Rightarrow\quad & \begin{pmatrix}
     A+B+C+D & A+B-C-D &  B+C-A-D  \\[1pt]
     A+B-C-D & A+B+C+D &  B+D-A-C  \\[1pt]
     B+C-A-D & B+D-A-C & A+B+C+D  \\[1pt]
   \end{pmatrix} \bm{\lambda} = \begin{pmatrix}
     -2 \\
     0 \\
    0
   \end{pmatrix}\\
   \Rightarrow \quad & \begin{pmatrix}
     \lambda_1 \\
     \lambda_2 \\
     \lambda_3
   \end{pmatrix} = \begin{pmatrix}
     -\frac{AB+BC+AD+CD}{2(ABC+ABD+ACD+BCD)} \\[2pt]
     -\frac{AB-CD}{2(ABC+ABD+ACD+BCD)}\\[2pt]
      -\frac{AD-BC}{2(ABC+ABD+ACD+BCD)}
   \end{pmatrix}
   \end{split}
   \end{equation}
   Therefore
   \begin{equation}
     \begin{split}
       \bm{h}^* &= \frac{(AB+BC+AD+CD)\bm{a} + (AB-CD)\bm{1} + (AD-BC) \bm{b}}{4(ABC+ABD+ACD+BCD)} \\
       h_i^* &= \begin{cases}
       \frac{ABC + ACD}{ABC+ABD+ACD+BCD} \cdot \frac{1}{2A} & i\in SL \\
       \frac{ABD + BCD}{ABC+ABD+ACD+BCD} \cdot \frac{1}{2B} & i\in SH \\
       \frac{ABC + ACD}{ABC+ABD+ACD+BCD} \cdot -\frac{1}{2C} & i\in BL \\
       \frac{ABD + BCD}{ABC+ABD+ACD+BCD} \cdot -\frac{1}{2D} & i\in BH \\
       \end{cases}
     \end{split}
   \end{equation}
   The coefficient before the required quantity is not a constant unless
   \begin{equation}
     ABC + ACD = ABD + BCD
   \end{equation}
   That is, $\frac{1}{D} + \frac{1}{B} = \frac{1}{C} + \frac{1}{A}$ $\iff$ $\frac{1}{|BH|} + \frac{1}{|SH|} = \frac{1}{|BL|} + \frac{1}{|SL|}$. This is not necessarily the case. Hence, the proposition is disproved.

    


    \section*{Problem 3}\label{problem-3}

\subsection*{3-(a)}\label{a}


            \begin{Verbatim}[commandchars=\\\{\}]
{\color{outcolor}Out[{\color{outcolor}4}]:}           const Mkt-RF    SMB    RMW     CMA
        est      -0.042  0.018  0.011  0.129   1.025
        t-stat   -0.486  0.849  0.381  3.169  22.795
        p<0.05                           (*)     (*)
        Adj.R\^{}2   0.500                             
\end{Verbatim}
        
    \subsection*{3-(b)}\label{b}


            \begin{Verbatim}[commandchars=\\\{\}]
{\color{outcolor}Out[{\color{outcolor}5}]:}          const  Mkt-RF    SMB    RMW     CMA     MOM
        est      0.050  -0.000  0.013  0.149   1.008  -0.129
        t-stat   0.587  -0.009  0.448  3.787  23.191  -6.822
        p<0.05                           (*)     (*)     (*)
        Adj.R\^{}2  0.535                                      
\end{Verbatim}
        
    \subsection*{3-(c)}\label{c}


            \begin{Verbatim}[commandchars=\\\{\}]
{\color{outcolor}Out[{\color{outcolor}6}]:}           const  Mkt-RF    SMB     RMW     CMA      MOM
        est      -0.014   0.066  0.014  -0.054   0.972         
        t-stat   -0.110   2.136  0.324  -0.913  14.783         
        p<0.05              (*)                    (*)         
        Adj.R\^{}2   0.279                                        
        est       0.363  -0.008  0.020   0.028   0.900   -0.528
        t-stat    4.289  -0.413  0.706   0.712  20.699  -28.009
        p<0.05      (*)                            (*)      (*)
        Adj.R\^{}2   0.686                                        
\end{Verbatim}
        
    \subsection*{3-(d)}\label{d}


            \begin{Verbatim}[commandchars=\\\{\}]
{\color{outcolor}Out[{\color{outcolor}7}]:}                    const  Mkt-RF    SMB     RMW     CMA      MOM
        HML      est      -0.042   0.018  0.011   0.129   1.025         
                 t-stat   -0.486   0.849  0.381   3.169  22.795         
                 p<0.05                             (*)     (*)         
                 Adj.R\^{}2   0.500                                        
        HML      est       0.050  -0.000  0.013   0.149   1.008   -0.129
                 t-stat    0.587  -0.009  0.448   3.787  23.191   -6.822
                 p<0.05                             (*)     (*)      (*)
                 Adj.R\^{}2   0.535                                        
        HML-DEV  est      -0.014   0.066  0.014  -0.054   0.972         
                 t-stat   -0.110   2.136  0.324  -0.913  14.783         
                 p<0.05              (*)                    (*)         
                 Adj.R\^{}2   0.279                                        
        HML-DEV  est       0.363  -0.008  0.020   0.028   0.900   -0.528
                 t-stat    4.289  -0.413  0.706   0.712  20.699  -28.009
                 p<0.05      (*)                            (*)      (*)
                 Adj.R\^{}2   0.686                                        
\end{Verbatim}
        
    \textbf{Comments}

\begin{itemize}
\tightlist
\item
  The results here are consistent with those in the article.
\item
  For HML, we observe its large and statistically significant exposures
  on RMW, CMA, and MOM. The negative exposure to MOM indicates that the
  return of HML and the return of MOM is negatively related.
\item
  HML has low, insignificant alpha when regressed on other factors. It
  bespeaks that the HML factor is kind of "redundant", in the sense that
  its risk premium can be almost fully expained by the risk premium of
  some return series living in the space spanned by other factors
  excluding HML.
\item
  HML-DEV can "resurrect" HML, in the sense that it still has large,
  significant alpha (0.363\%, with t-statistic = 4.289) when regressed
  to other factors. This indicates that HML-DEV is not "redundent" in
  the above sense.
\end{itemize}

    \subsection*{3-(e)}\label{e}


    \subsubsection*{1998 - 2018 Results}\label{results}

    \begin{Verbatim}[commandchars=\\\{\}]
{\color{incolor}In [{\color{incolor}9}]:} \PY{n}{prd1} \PY{o}{=} \PY{n}{df}\PY{p}{[}\PY{p}{(}\PY{n}{df}\PY{o}{.}\PY{n}{Date}\PY{o}{\PYZgt{}}\PY{o}{=}\PY{l+m+mi}{199800}\PY{p}{)} \PY{o}{\PYZam{}} \PY{p}{(}\PY{n}{df}\PY{o}{.}\PY{n}{Date}\PY{o}{\PYZlt{}}\PY{o}{=}\PY{l+m+mi}{201812}\PY{p}{)}\PY{p}{]}
        \PY{n}{test\PYZus{}hml}\PY{p}{(}\PY{n}{prd1}\PY{p}{)}
\end{Verbatim}

            \begin{Verbatim}[commandchars=\\\{\}]
{\color{outcolor}Out[{\color{outcolor}9}]:}           const Mkt-RF    SMB    RMW     CMA      MOM
        est      -0.352  0.185  0.063  0.454   0.863         
        t-stat   -2.408  4.898  1.209  7.109  12.248         
        p<0.05      (*)    (*)           (*)     (*)         
        Adj.R\^{}2   0.515                                      
        est      -0.271  0.128  0.095  0.456   0.828   -0.128
        t-stat   -1.917  3.366  1.887  7.445  12.181   -4.759
        p<0.05             (*)           (*)     (*)      (*)
        Adj.R\^{}2   0.554                                      
        est      -0.377  0.359  0.002  0.322   0.863         
        t-stat   -1.636  6.046  0.022  3.198   7.771         
        p<0.05             (*)           (*)     (*)         
        Adj.R\^{}2   0.265                                      
        est      -0.015  0.108  0.144  0.330   0.706   -0.570
        t-stat   -0.112  3.023  3.061  5.749  11.090  -22.576
        p<0.05             (*)    (*)    (*)     (*)      (*)
        Adj.R\^{}2   0.762                                      
\end{Verbatim}
        
    \subsubsection*{1978 - 2008 Results}\label{results}

    \begin{Verbatim}[commandchars=\\\{\}]
{\color{incolor}In [{\color{incolor}10}]:} \PY{n}{prd1} \PY{o}{=} \PY{n}{df}\PY{p}{[}\PY{p}{(}\PY{n}{df}\PY{o}{.}\PY{n}{Date}\PY{o}{\PYZgt{}}\PY{o}{=}\PY{l+m+mi}{197800}\PY{p}{)} \PY{o}{\PYZam{}} \PY{p}{(}\PY{n}{df}\PY{o}{.}\PY{n}{Date}\PY{o}{\PYZlt{}}\PY{o}{=}\PY{l+m+mi}{200812}\PY{p}{)}\PY{p}{]}
         \PY{n}{test\PYZus{}hml}\PY{p}{(}\PY{n}{prd1}\PY{p}{)}
\end{Verbatim}

            \begin{Verbatim}[commandchars=\\\{\}]
{\color{outcolor}Out[{\color{outcolor}10}]:}           const  Mkt-RF     SMB     RMW     CMA      MOM
         est       0.023  -0.044  -0.087   0.173   0.911         
         t-stat    0.208  -1.659  -2.339   3.754  16.527         
         p<0.05                      (*)     (*)     (*)         
         Adj.R\^{}2   0.553                                         
         est       0.118  -0.047  -0.053   0.212   0.904   -0.127
         t-stat    1.113  -1.847  -1.469   4.717  17.022   -5.450
         p<0.05                              (*)     (*)      (*)
         Adj.R\^{}2   0.586                                         
         est      -0.008  -0.031  -0.192  -0.078   0.820         
         t-stat   -0.051  -0.809  -3.551  -1.168  10.192         
         p<0.05                      (*)             (*)         
         Adj.R\^{}2   0.303                                         
         est       0.378  -0.044  -0.056   0.080   0.791   -0.516
         t-stat    3.622  -1.757  -1.582   1.801  15.124  -22.422
         p<0.05      (*)                             (*)      (*)
         Adj.R\^{}2   0.705                                         
\end{Verbatim}
        
    \subsubsection*{1968 - 1988 Results}\label{results}

    \begin{Verbatim}[commandchars=\\\{\}]
{\color{incolor}In [{\color{incolor}11}]:} \PY{n}{prd1} \PY{o}{=} \PY{n}{df}\PY{p}{[}\PY{p}{(}\PY{n}{df}\PY{o}{.}\PY{n}{Date}\PY{o}{\PYZgt{}}\PY{o}{=}\PY{l+m+mi}{196800}\PY{p}{)} \PY{o}{\PYZam{}} \PY{p}{(}\PY{n}{df}\PY{o}{.}\PY{n}{Date}\PY{o}{\PYZlt{}}\PY{o}{=}\PY{l+m+mi}{198812}\PY{p}{)}\PY{p}{]}
         \PY{n}{test\PYZus{}hml}\PY{p}{(}\PY{n}{prd1}\PY{p}{)}
\end{Verbatim}

            \begin{Verbatim}[commandchars=\\\{\}]
{\color{outcolor}Out[{\color{outcolor}11}]:}          const  Mkt-RF    SMB     RMW     CMA      MOM
         est      0.236  -0.082  0.049  -0.319   0.874         
         t-stat   1.982  -3.127  1.250  -3.850  11.732         
         p<0.05     (*)     (*)            (*)     (*)         
         Adj.R\^{}2  0.621                                        
         est      0.276  -0.079  0.039  -0.293   0.877   -0.060
         t-stat   2.294  -3.040  0.986  -3.521  11.841   -1.997
         p<0.05     (*)     (*)            (*)     (*)      (*)
         Adj.R\^{}2  0.626                                        
         est      0.403  -0.112  0.124  -0.518   0.761         
         t-stat   2.470  -3.139  2.324  -4.565   7.464         
         p<0.05     (*)     (*)    (*)     (*)     (*)         
         Adj.R\^{}2  0.487                                        
         est      0.671  -0.094  0.055  -0.344   0.782   -0.412
         t-stat   5.201  -3.372  1.304  -3.845   9.836  -12.677
         p<0.05     (*)     (*)            (*)     (*)      (*)
         Adj.R\^{}2  0.689                                        
\end{Verbatim}
        
    \textbf{Comments}

\begin{itemize}
\tightlist
\item
  The results in (d) is not always the case over different sample
  period.
\item
  Over 1998 - 2018, HML-DEV does not have significant alpha, but HML has
  significant negative alpha when regressed to the other FF 4 factors.
\item
  Over 1978 - 2008, the results in (d) hold in a similar way.
\item
  Over 1968 - 1988, Both HML and HML-DEV are not redundent (have
  significant positive alphas). Most pricing factors turn out to be more
  effective in early times. Nevertheless, HML-DEV still has better
  (larger, and more significant) alphas compared with HML. So it's
  legitimate to say that the paper's main idea still holds in this
  period.
\end{itemize}

    \subsubsection*{3-(f)}\label{f}


    \begin{Verbatim}[commandchars=\\\{\}]
{\color{incolor}In [{\color{incolor}13}]:} \PY{n}{rep5}
\end{Verbatim}

            \begin{Verbatim}[commandchars=\\\{\}]
{\color{outcolor}Out[{\color{outcolor}13}]:}           const Mkt-RF     SMB    RMW    CMA     MOM HML-DEV
         est      -0.231  0.006  -0.003  0.128  0.312   0.280   0.773
         t-stat   -4.236  0.488  -0.155  5.096  8.657  15.407  29.913
         p<0.05      (*)                   (*)    (*)     (*)     (*)
         Adj.R\^{}2   0.813                                             
\end{Verbatim}
        
    \begin{Verbatim}[commandchars=\\\{\}]
{\color{incolor}In [{\color{incolor}14}]:} \PY{n}{rep6}
\end{Verbatim}

            \begin{Verbatim}[commandchars=\\\{\}]
{\color{outcolor}Out[{\color{outcolor}14}]:}          const  Mkt-RF    SMB     RMW    CMA      MOM     HML
         est      0.325  -0.008  0.010  -0.087  0.120   -0.429   0.773
         t-stat   6.041  -0.639  0.567  -3.450  3.171  -34.512  29.913
         p<0.05     (*)                    (*)    (*)      (*)     (*)
         Adj.R\^{}2  0.874                                               
\end{Verbatim}
        
    \textbf{Comments}

\begin{itemize}
\tightlist
\item
  HML-DEV has positive and significant alpha, while HML has negative
  significant alpha.
\item
  Therefore, HML-DEC is more profitable when other factors are being
  controled.
\end{itemize}


    % Add a bibliography block to the postdoc
    
    
    
    \end{document}

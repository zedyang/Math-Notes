
% Default to the notebook output style

    


% Inherit from the specified cell style.




    
\documentclass[10pt]{article}

    
    
    \usepackage[T1]{fontenc}
    % Nicer default font (+ math font) than Computer Modern for most use cases
    %\usepackage{mathpazo}

    % Basic figure setup, for now with no caption control since it's done
    % automatically by Pandoc (which extracts ![](path) syntax from Markdown).
    \usepackage{graphicx}
    % We will generate all images so they have a width \maxwidth. This means
    % that they will get their normal width if they fit onto the page, but
    % are scaled down if they would overflow the margins.
    \makeatletter
    \def\maxwidth{\ifdim\Gin@nat@width>\linewidth\linewidth
    \else\Gin@nat@width\fi}
    \makeatother
    \let\Oldincludegraphics\includegraphics
    % Set max figure width to be 80% of text width, for now hardcoded.
    \renewcommand{\includegraphics}[1]{\Oldincludegraphics[width=.8\maxwidth]{#1}}
    % Ensure that by default, figures have no caption (until we provide a
    % proper Figure object with a Caption API and a way to capture that
    % in the conversion process - todo).
    \usepackage{caption}
    \DeclareCaptionLabelFormat{nolabel}{}
    \captionsetup{labelformat=nolabel}

    \usepackage{adjustbox} % Used to constrain images to a maximum size 
    \usepackage{xcolor} % Allow colors to be defined
    \usepackage{enumerate} % Needed for markdown enumerations to work
    \usepackage{geometry} % Used to adjust the document margins
    \usepackage{amsmath,bm,bbm} % Equations
    \usepackage{amssymb} % Equations
    \usepackage{textcomp} % defines textquotesingle
    % Hack from http://tex.stackexchange.com/a/47451/13684:
    \AtBeginDocument{%
        \def\PYZsq{\textquotesingle}% Upright quotes in Pygmentized code
    }
    \usepackage{upquote} % Upright quotes for verbatim code
    \usepackage{eurosym} % defines \euro
    \usepackage[mathletters]{ucs} % Extended unicode (utf-8) support
    \usepackage[utf8x]{inputenc} % Allow utf-8 characters in the tex document
    \usepackage{fancyvrb} % verbatim replacement that allows latex
    \usepackage{grffile} % extends the file name processing of package graphics 
                         % to support a larger range 
    % The hyperref package gives us a pdf with properly built
    % internal navigation ('pdf bookmarks' for the table of contents,
    % internal cross-reference links, web links for URLs, etc.)
    \usepackage{hyperref}
    \usepackage{longtable} % longtable support required by pandoc >1.10
    \usepackage{booktabs}  % table support for pandoc > 1.12.2
    \usepackage[inline]{enumitem} % IRkernel/repr support (it uses the enumerate* environment)
    \usepackage[normalem]{ulem} % ulem is needed to support strikethroughs (\sout)
                                % normalem makes italics be italics, not underlines
    

    
    
    % Colors for the hyperref package
    \definecolor{urlcolor}{rgb}{0,.145,.698}
    \definecolor{linkcolor}{rgb}{.71,0.21,0.01}
    \definecolor{citecolor}{rgb}{.12,.54,.11}

    % ANSI colors
    \definecolor{ansi-black}{HTML}{3E424D}
    \definecolor{ansi-black-intense}{HTML}{282C36}
    \definecolor{ansi-red}{HTML}{E75C58}
    \definecolor{ansi-red-intense}{HTML}{B22B31}
    \definecolor{ansi-green}{HTML}{00A250}
    \definecolor{ansi-green-intense}{HTML}{007427}
    \definecolor{ansi-yellow}{HTML}{DDB62B}
    \definecolor{ansi-yellow-intense}{HTML}{B27D12}
    \definecolor{ansi-blue}{HTML}{208FFB}
    \definecolor{ansi-blue-intense}{HTML}{0065CA}
    \definecolor{ansi-magenta}{HTML}{D160C4}
    \definecolor{ansi-magenta-intense}{HTML}{A03196}
    \definecolor{ansi-cyan}{HTML}{60C6C8}
    \definecolor{ansi-cyan-intense}{HTML}{258F8F}
    \definecolor{ansi-white}{HTML}{C5C1B4}
    \definecolor{ansi-white-intense}{HTML}{A1A6B2}

    % commands and environments needed by pandoc snippets
    % extracted from the output of `pandoc -s`
    \providecommand{\tightlist}{%
      \setlength{\itemsep}{0pt}\setlength{\parskip}{0pt}}
    \DefineVerbatimEnvironment{Highlighting}{Verbatim}{commandchars=\\\{\}}
    % Add ',fontsize=\small' for more characters per line
    \newenvironment{Shaded}{}{}
    \newcommand{\KeywordTok}[1]{\textcolor[rgb]{0.00,0.44,0.13}{\textbf{{#1}}}}
    \newcommand{\DataTypeTok}[1]{\textcolor[rgb]{0.56,0.13,0.00}{{#1}}}
    \newcommand{\DecValTok}[1]{\textcolor[rgb]{0.25,0.63,0.44}{{#1}}}
    \newcommand{\BaseNTok}[1]{\textcolor[rgb]{0.25,0.63,0.44}{{#1}}}
    \newcommand{\FloatTok}[1]{\textcolor[rgb]{0.25,0.63,0.44}{{#1}}}
    \newcommand{\CharTok}[1]{\textcolor[rgb]{0.25,0.44,0.63}{{#1}}}
    \newcommand{\StringTok}[1]{\textcolor[rgb]{0.25,0.44,0.63}{{#1}}}
    \newcommand{\CommentTok}[1]{\textcolor[rgb]{0.38,0.63,0.69}{\textit{{#1}}}}
    \newcommand{\OtherTok}[1]{\textcolor[rgb]{0.00,0.44,0.13}{{#1}}}
    \newcommand{\AlertTok}[1]{\textcolor[rgb]{1.00,0.00,0.00}{\textbf{{#1}}}}
    \newcommand{\FunctionTok}[1]{\textcolor[rgb]{0.02,0.16,0.49}{{#1}}}
    \newcommand{\RegionMarkerTok}[1]{{#1}}
    \newcommand{\ErrorTok}[1]{\textcolor[rgb]{1.00,0.00,0.00}{\textbf{{#1}}}}
    \newcommand{\NormalTok}[1]{{#1}}
    
    % Additional commands for more recent versions of Pandoc
    \newcommand{\ConstantTok}[1]{\textcolor[rgb]{0.53,0.00,0.00}{{#1}}}
    \newcommand{\SpecialCharTok}[1]{\textcolor[rgb]{0.25,0.44,0.63}{{#1}}}
    \newcommand{\VerbatimStringTok}[1]{\textcolor[rgb]{0.25,0.44,0.63}{{#1}}}
    \newcommand{\SpecialStringTok}[1]{\textcolor[rgb]{0.73,0.40,0.53}{{#1}}}
    \newcommand{\ImportTok}[1]{{#1}}
    \newcommand{\DocumentationTok}[1]{\textcolor[rgb]{0.73,0.13,0.13}{\textit{{#1}}}}
    \newcommand{\AnnotationTok}[1]{\textcolor[rgb]{0.38,0.63,0.69}{\textbf{\textit{{#1}}}}}
    \newcommand{\CommentVarTok}[1]{\textcolor[rgb]{0.38,0.63,0.69}{\textbf{\textit{{#1}}}}}
    \newcommand{\VariableTok}[1]{\textcolor[rgb]{0.10,0.09,0.49}{{#1}}}
    \newcommand{\ControlFlowTok}[1]{\textcolor[rgb]{0.00,0.44,0.13}{\textbf{{#1}}}}
    \newcommand{\OperatorTok}[1]{\textcolor[rgb]{0.40,0.40,0.40}{{#1}}}
    \newcommand{\BuiltInTok}[1]{{#1}}
    \newcommand{\ExtensionTok}[1]{{#1}}
    \newcommand{\PreprocessorTok}[1]{\textcolor[rgb]{0.74,0.48,0.00}{{#1}}}
    \newcommand{\AttributeTok}[1]{\textcolor[rgb]{0.49,0.56,0.16}{{#1}}}
    \newcommand{\InformationTok}[1]{\textcolor[rgb]{0.38,0.63,0.69}{\textbf{\textit{{#1}}}}}
    \newcommand{\WarningTok}[1]{\textcolor[rgb]{0.38,0.63,0.69}{\textbf{\textit{{#1}}}}}
    
    
    % Define a nice break command that doesn't care if a line doesn't already
    % exist.
    \def\br{\hspace*{\fill} \\* }
    % Math Jax compatability definitions
    \def\gt{>}
    \def\lt{<}
    % Document parameters
    \title{Asset Management HW3}
    \author{Ze Yang \\ Zhengyang Qi}
    
    
    

    % Pygments definitions
    
\makeatletter
\def\PY@reset{\let\PY@it=\relax \let\PY@bf=\relax%
    \let\PY@ul=\relax \let\PY@tc=\relax%
    \let\PY@bc=\relax \let\PY@ff=\relax}
\def\PY@tok#1{\csname PY@tok@#1\endcsname}
\def\PY@toks#1+{\ifx\relax#1\empty\else%
    \PY@tok{#1}\expandafter\PY@toks\fi}
\def\PY@do#1{\PY@bc{\PY@tc{\PY@ul{%
    \PY@it{\PY@bf{\PY@ff{#1}}}}}}}
\def\PY#1#2{\PY@reset\PY@toks#1+\relax+\PY@do{#2}}

\expandafter\def\csname PY@tok@w\endcsname{\def\PY@tc##1{\textcolor[rgb]{0.73,0.73,0.73}{##1}}}
\expandafter\def\csname PY@tok@c\endcsname{\let\PY@it=\textit\def\PY@tc##1{\textcolor[rgb]{0.25,0.50,0.50}{##1}}}
\expandafter\def\csname PY@tok@cp\endcsname{\def\PY@tc##1{\textcolor[rgb]{0.74,0.48,0.00}{##1}}}
\expandafter\def\csname PY@tok@k\endcsname{\let\PY@bf=\textbf\def\PY@tc##1{\textcolor[rgb]{0.00,0.50,0.00}{##1}}}
\expandafter\def\csname PY@tok@kp\endcsname{\def\PY@tc##1{\textcolor[rgb]{0.00,0.50,0.00}{##1}}}
\expandafter\def\csname PY@tok@kt\endcsname{\def\PY@tc##1{\textcolor[rgb]{0.69,0.00,0.25}{##1}}}
\expandafter\def\csname PY@tok@o\endcsname{\def\PY@tc##1{\textcolor[rgb]{0.40,0.40,0.40}{##1}}}
\expandafter\def\csname PY@tok@ow\endcsname{\let\PY@bf=\textbf\def\PY@tc##1{\textcolor[rgb]{0.67,0.13,1.00}{##1}}}
\expandafter\def\csname PY@tok@nb\endcsname{\def\PY@tc##1{\textcolor[rgb]{0.00,0.50,0.00}{##1}}}
\expandafter\def\csname PY@tok@nf\endcsname{\def\PY@tc##1{\textcolor[rgb]{0.00,0.00,1.00}{##1}}}
\expandafter\def\csname PY@tok@nc\endcsname{\let\PY@bf=\textbf\def\PY@tc##1{\textcolor[rgb]{0.00,0.00,1.00}{##1}}}
\expandafter\def\csname PY@tok@nn\endcsname{\let\PY@bf=\textbf\def\PY@tc##1{\textcolor[rgb]{0.00,0.00,1.00}{##1}}}
\expandafter\def\csname PY@tok@ne\endcsname{\let\PY@bf=\textbf\def\PY@tc##1{\textcolor[rgb]{0.82,0.25,0.23}{##1}}}
\expandafter\def\csname PY@tok@nv\endcsname{\def\PY@tc##1{\textcolor[rgb]{0.10,0.09,0.49}{##1}}}
\expandafter\def\csname PY@tok@no\endcsname{\def\PY@tc##1{\textcolor[rgb]{0.53,0.00,0.00}{##1}}}
\expandafter\def\csname PY@tok@nl\endcsname{\def\PY@tc##1{\textcolor[rgb]{0.63,0.63,0.00}{##1}}}
\expandafter\def\csname PY@tok@ni\endcsname{\let\PY@bf=\textbf\def\PY@tc##1{\textcolor[rgb]{0.60,0.60,0.60}{##1}}}
\expandafter\def\csname PY@tok@na\endcsname{\def\PY@tc##1{\textcolor[rgb]{0.49,0.56,0.16}{##1}}}
\expandafter\def\csname PY@tok@nt\endcsname{\let\PY@bf=\textbf\def\PY@tc##1{\textcolor[rgb]{0.00,0.50,0.00}{##1}}}
\expandafter\def\csname PY@tok@nd\endcsname{\def\PY@tc##1{\textcolor[rgb]{0.67,0.13,1.00}{##1}}}
\expandafter\def\csname PY@tok@s\endcsname{\def\PY@tc##1{\textcolor[rgb]{0.73,0.13,0.13}{##1}}}
\expandafter\def\csname PY@tok@sd\endcsname{\let\PY@it=\textit\def\PY@tc##1{\textcolor[rgb]{0.73,0.13,0.13}{##1}}}
\expandafter\def\csname PY@tok@si\endcsname{\let\PY@bf=\textbf\def\PY@tc##1{\textcolor[rgb]{0.73,0.40,0.53}{##1}}}
\expandafter\def\csname PY@tok@se\endcsname{\let\PY@bf=\textbf\def\PY@tc##1{\textcolor[rgb]{0.73,0.40,0.13}{##1}}}
\expandafter\def\csname PY@tok@sr\endcsname{\def\PY@tc##1{\textcolor[rgb]{0.73,0.40,0.53}{##1}}}
\expandafter\def\csname PY@tok@ss\endcsname{\def\PY@tc##1{\textcolor[rgb]{0.10,0.09,0.49}{##1}}}
\expandafter\def\csname PY@tok@sx\endcsname{\def\PY@tc##1{\textcolor[rgb]{0.00,0.50,0.00}{##1}}}
\expandafter\def\csname PY@tok@m\endcsname{\def\PY@tc##1{\textcolor[rgb]{0.40,0.40,0.40}{##1}}}
\expandafter\def\csname PY@tok@gh\endcsname{\let\PY@bf=\textbf\def\PY@tc##1{\textcolor[rgb]{0.00,0.00,0.50}{##1}}}
\expandafter\def\csname PY@tok@gu\endcsname{\let\PY@bf=\textbf\def\PY@tc##1{\textcolor[rgb]{0.50,0.00,0.50}{##1}}}
\expandafter\def\csname PY@tok@gd\endcsname{\def\PY@tc##1{\textcolor[rgb]{0.63,0.00,0.00}{##1}}}
\expandafter\def\csname PY@tok@gi\endcsname{\def\PY@tc##1{\textcolor[rgb]{0.00,0.63,0.00}{##1}}}
\expandafter\def\csname PY@tok@gr\endcsname{\def\PY@tc##1{\textcolor[rgb]{1.00,0.00,0.00}{##1}}}
\expandafter\def\csname PY@tok@ge\endcsname{\let\PY@it=\textit}
\expandafter\def\csname PY@tok@gs\endcsname{\let\PY@bf=\textbf}
\expandafter\def\csname PY@tok@gp\endcsname{\let\PY@bf=\textbf\def\PY@tc##1{\textcolor[rgb]{0.00,0.00,0.50}{##1}}}
\expandafter\def\csname PY@tok@go\endcsname{\def\PY@tc##1{\textcolor[rgb]{0.53,0.53,0.53}{##1}}}
\expandafter\def\csname PY@tok@gt\endcsname{\def\PY@tc##1{\textcolor[rgb]{0.00,0.27,0.87}{##1}}}
\expandafter\def\csname PY@tok@err\endcsname{\def\PY@bc##1{\setlength{\fboxsep}{0pt}\fcolorbox[rgb]{1.00,0.00,0.00}{1,1,1}{\strut ##1}}}
\expandafter\def\csname PY@tok@kc\endcsname{\let\PY@bf=\textbf\def\PY@tc##1{\textcolor[rgb]{0.00,0.50,0.00}{##1}}}
\expandafter\def\csname PY@tok@kd\endcsname{\let\PY@bf=\textbf\def\PY@tc##1{\textcolor[rgb]{0.00,0.50,0.00}{##1}}}
\expandafter\def\csname PY@tok@kn\endcsname{\let\PY@bf=\textbf\def\PY@tc##1{\textcolor[rgb]{0.00,0.50,0.00}{##1}}}
\expandafter\def\csname PY@tok@kr\endcsname{\let\PY@bf=\textbf\def\PY@tc##1{\textcolor[rgb]{0.00,0.50,0.00}{##1}}}
\expandafter\def\csname PY@tok@bp\endcsname{\def\PY@tc##1{\textcolor[rgb]{0.00,0.50,0.00}{##1}}}
\expandafter\def\csname PY@tok@fm\endcsname{\def\PY@tc##1{\textcolor[rgb]{0.00,0.00,1.00}{##1}}}
\expandafter\def\csname PY@tok@vc\endcsname{\def\PY@tc##1{\textcolor[rgb]{0.10,0.09,0.49}{##1}}}
\expandafter\def\csname PY@tok@vg\endcsname{\def\PY@tc##1{\textcolor[rgb]{0.10,0.09,0.49}{##1}}}
\expandafter\def\csname PY@tok@vi\endcsname{\def\PY@tc##1{\textcolor[rgb]{0.10,0.09,0.49}{##1}}}
\expandafter\def\csname PY@tok@vm\endcsname{\def\PY@tc##1{\textcolor[rgb]{0.10,0.09,0.49}{##1}}}
\expandafter\def\csname PY@tok@sa\endcsname{\def\PY@tc##1{\textcolor[rgb]{0.73,0.13,0.13}{##1}}}
\expandafter\def\csname PY@tok@sb\endcsname{\def\PY@tc##1{\textcolor[rgb]{0.73,0.13,0.13}{##1}}}
\expandafter\def\csname PY@tok@sc\endcsname{\def\PY@tc##1{\textcolor[rgb]{0.73,0.13,0.13}{##1}}}
\expandafter\def\csname PY@tok@dl\endcsname{\def\PY@tc##1{\textcolor[rgb]{0.73,0.13,0.13}{##1}}}
\expandafter\def\csname PY@tok@s2\endcsname{\def\PY@tc##1{\textcolor[rgb]{0.73,0.13,0.13}{##1}}}
\expandafter\def\csname PY@tok@sh\endcsname{\def\PY@tc##1{\textcolor[rgb]{0.73,0.13,0.13}{##1}}}
\expandafter\def\csname PY@tok@s1\endcsname{\def\PY@tc##1{\textcolor[rgb]{0.73,0.13,0.13}{##1}}}
\expandafter\def\csname PY@tok@mb\endcsname{\def\PY@tc##1{\textcolor[rgb]{0.40,0.40,0.40}{##1}}}
\expandafter\def\csname PY@tok@mf\endcsname{\def\PY@tc##1{\textcolor[rgb]{0.40,0.40,0.40}{##1}}}
\expandafter\def\csname PY@tok@mh\endcsname{\def\PY@tc##1{\textcolor[rgb]{0.40,0.40,0.40}{##1}}}
\expandafter\def\csname PY@tok@mi\endcsname{\def\PY@tc##1{\textcolor[rgb]{0.40,0.40,0.40}{##1}}}
\expandafter\def\csname PY@tok@il\endcsname{\def\PY@tc##1{\textcolor[rgb]{0.40,0.40,0.40}{##1}}}
\expandafter\def\csname PY@tok@mo\endcsname{\def\PY@tc##1{\textcolor[rgb]{0.40,0.40,0.40}{##1}}}
\expandafter\def\csname PY@tok@ch\endcsname{\let\PY@it=\textit\def\PY@tc##1{\textcolor[rgb]{0.25,0.50,0.50}{##1}}}
\expandafter\def\csname PY@tok@cm\endcsname{\let\PY@it=\textit\def\PY@tc##1{\textcolor[rgb]{0.25,0.50,0.50}{##1}}}
\expandafter\def\csname PY@tok@cpf\endcsname{\let\PY@it=\textit\def\PY@tc##1{\textcolor[rgb]{0.25,0.50,0.50}{##1}}}
\expandafter\def\csname PY@tok@c1\endcsname{\let\PY@it=\textit\def\PY@tc##1{\textcolor[rgb]{0.25,0.50,0.50}{##1}}}
\expandafter\def\csname PY@tok@cs\endcsname{\let\PY@it=\textit\def\PY@tc##1{\textcolor[rgb]{0.25,0.50,0.50}{##1}}}

\def\PYZbs{\char`\\}
\def\PYZus{\char`\_}
\def\PYZob{\char`\{}
\def\PYZcb{\char`\}}
\def\PYZca{\char`\^}
\def\PYZam{\char`\&}
\def\PYZlt{\char`\<}
\def\PYZgt{\char`\>}
\def\PYZsh{\char`\#}
\def\PYZpc{\char`\%}
\def\PYZdl{\char`\$}
\def\PYZhy{\char`\-}
\def\PYZsq{\char`\'}
\def\PYZdq{\char`\"}
\def\PYZti{\char`\~}
% for compatibility with earlier versions
\def\PYZat{@}
\def\PYZlb{[}
\def\PYZrb{]}
\makeatother


    % Exact colors from NB
    \definecolor{incolor}{rgb}{0.0, 0.0, 0.5}
    \definecolor{outcolor}{rgb}{0.545, 0.0, 0.0}



    
    % Prevent overflowing lines due to hard-to-break entities
    \sloppy 
    % Setup hyperref package
    \hypersetup{
      breaklinks=true,  % so long urls are correctly broken across lines
      colorlinks=true,
      urlcolor=urlcolor,
      linkcolor=linkcolor,
      citecolor=citecolor,
      }
    % Slightly bigger margins than the latex defaults
    
    \geometry{verbose,tmargin=1in,bmargin=1in,lmargin=1in,rmargin=1in}
    
    

    \begin{document}
    
    
    \maketitle
    

    \section{Problem 1}
\subsection{(a)} 
The Lagrangian of agent $i$'s problem is
\begin{equation}
  \mathcal{L}_i(\bm{x}, \lambda) = (\mathbb{E}\left[\bm{P}^1\right] - \bm{P}^0)^{\top} \bm{x} - \frac{\gamma_i}{2} \bm{x}^{\top} \bm{\Omega}\bm{x} - \lambda(m_i (\bm{P}^0)^{\top} \bm{x} - W_i)
\end{equation}
$\bm{x}^i, \lambda_i$ solves the KKT condition:
\begin{equation}
  \begin{cases}
    \nabla_{\bm{x}} \mathcal{L} = (\mathbb{E}\left[\bm{P}^1\right] - \bm{P}^0) - \gamma_i \bm{\Omega x} - \lambda_i m_i \bm{P}^0 = \bm{0} \\
    m_i (\bm{P}^0)^{\top} \bm{x} - W_i \leq 0 \\
    \lambda_i \geq 0 \\
    \lambda_i (m_i (\bm{P}^0)^{\top} \bm{x} - W_i) = 0
  \end{cases}
\end{equation}
The first equality yields $\bm{x}^i = \frac{1}{\gamma_i}\bm{\Omega}^{-1}\left(\mathbb{E}\left[\bm{P}^1\right] - (1+\lambda_i m_i)\bm{P}^0\right)$. Assume the constraint is binding, then
\begin{equation}
  \begin{split}
    &0 = m_i (\bm{P}^0)^{\top} \bm{x}^i - W_i = \frac{1}{\gamma_i}m_i (\bm{P}^0)^{\top}\bm{\Omega}^{-1}\left(\mathbb{E}\left[\bm{P}^1\right] - (1+\lambda_i m_i)\bm{P}^0\right) - W_i \\
    \Rightarrow \quad &  \frac{\lambda_i}{\gamma_i}m_i^2 (\bm{P}^0)^{\top}\bm{\Omega}^{-1} \bm{P}^0 =  \frac{1}{\gamma_i}m_i (\bm{P}^0)^{\top}\bm{\Omega}^{-1}\left(\mathbb{E}\left[\bm{P}^1\right] - \bm{P}^0\right) - W_i \\
    \Rightarrow \quad & \lambda_i = \frac{m_i (\bm{P}^0)^{\top}\bm{\Omega}^{-1}\left(\mathbb{E}\left[\bm{P}^1\right] - \bm{P}^0\right) - \gamma_i W_i}{m_i^2 (\bm{P}^0)^{\top}\bm{\Omega}^{-1} \bm{P}^0}
  \end{split}
\end{equation}
Dual feasibility requires $\lambda_i \geq 0 \Rightarrow$ $m_i (\bm{P}^0)^{\top}\bm{\Omega}^{-1}\left(\mathbb{E}\left[\bm{P}^1\right] - \bm{P}^0\right) - \gamma_i W_i \geq 0~~(\dagger)$. If instead we have $(\dagger)<0$, we have $\lambda_i = 0$. Then $\bm{x}^i = \frac{1}{\gamma_i}\bm{\Omega}^{-1}\left(\mathbb{E}\left[\bm{P}^1\right] - \bm{P}^0\right)$, hence $m_i (\bm{P}^0)^{\top} \bm{x}^i - W_i =\frac{1}{\gamma_i} m_i (\bm{P}^0)^{\top}\bm{\Omega}^{-1}\left(\mathbb{E}\left[\bm{P}^1\right] - \bm{P}^0\right) -  W_i < 0$ as the risk aversion coefficient $\gamma_i > 0$. The primal feasibility is satisfied. To conclude:
\begin{equation}
  \begin{split}\label{psii}
  &\bm{x}^i = \frac{1}{\gamma_i}\bm{\Omega}^{-1}\left(\mathbb{E}\left[\bm{P}^1\right] - (1+\psi_i)\bm{P}^0\right) \\
  &\text{where }\psi_i = 
  \begin{cases}
  \frac{m_i (\bm{P}^0)^{\top}\bm{\Omega}^{-1}\left(\mathbb{E}\left[\bm{P}^1\right] - \bm{P}^0\right) - \gamma_i W_i}{m_i (\bm{P}^0)^{\top}\bm{\Omega}^{-1} \bm{P}^0} & \text{if } m_i (\bm{P}^0)^{\top}\bm{\Omega}^{-1}\left(\mathbb{E}\left[\bm{P}^1\right] - \bm{P}^0\right) - \gamma_i W_i \geq 0 \\
  0 & \text{otherwise}
  \end{cases}
  \end{split}
\end{equation}
Consequently
\begin{equation}\label{xstar}
  \begin{split}
    \bm{x}^* &= \sum_{i=1}^I \bm{x}_i = \sum_{i=1}^I\frac{1}{\gamma_i}\bm{\Omega}^{-1}\left(\mathbb{E}\left[\bm{P}^1\right] - (1+\psi_i)\bm{P}^0\right) \\
    &=\frac{1}{\gamma}\bm{\Omega}^{-1}\left(\mathbb{E}\left[\bm{P}^1\right] - (1+\psi)\bm{P}^0\right)\\
    &\text{where } \gamma = \frac{1}{\sum_{i=1}^I 1/\gamma_i};~~~\psi = \frac{\sum_{i=1}^I \psi_i/\gamma_i}{\sum_{i=1}^I 1/\gamma_i}
  \end{split}
\end{equation}
\subsection{(b)} 
Premultiply both sides of (\ref{xstar}) by $\gamma \bm{\Omega}$:
\begin{equation}
  \begin{split}
    &\gamma \bm{\Omega}\bm{x}^* = \mathbb{E}\left[\bm{P}^1\right] - (1+\psi)\bm{P}^0 \\
    \Rightarrow\quad & \mathbb{E}\left[\bm{P}^1\right] = (1+\psi)\bm{P}^0 + \gamma \bm{\Omega}\bm{x}^* \\
    \Rightarrow \quad &\frac{\mathbb{E}\left[P^1_s\right]-P^0_s}{P^0_s} = \psi + \frac{\gamma}{P_s^0} (\bm{\Omega}\bm{x}^*)_s
  \end{split}
\end{equation}
Since $P_s^0$ is non-random, we conclude that $\mathbb{E}\left[r_s\right] = \frac{\mathbb{E}\left[P^1_s\right]-P^0_s}{P^0_s} = \psi + \frac{\gamma}{P_s^0} (\bm{\Omega}\bm{x}^*)_s$.

\subsection{(c)} 

By definition of $\bm{h}$, we know $\bm{h}^{\top} \bm{1} = 1$. Let $\mu_s := (\bm{\Omega} \bm{x}^*)_s /P_s^0 = \sum_{j=1}^S \Omega_{sj}x^{*}_j / P_s^0$ for $s=1,...,S$.
\begin{equation}
  \begin{split}
    \frac{\mathbb{E}\left[r_M\right] - \psi}{\mathrm{\mathbb{V}ar}\left[r_M\right]} &= \frac{\mathbb{E}\left[\bm{r}^{\top} \bm{h}\right] - \psi}{\mathrm{\mathbb{V}ar}\left[\bm{r}^{\top} \bm{h}\right]} = \frac{\bm{h}^{\top} \mathbb{E}\left[\bm{r}\right] - \psi}{\bm{h}^{\top} \mathrm{\mathbb{C}ov}\left[\bm{r}\right] \bm{h}}\\
    &= \frac{\bm{h}^{\top} (\psi \bm{1} + \gamma \bm{\mu})- \psi}{\bm{h}^{\top} \mathrm{\mathbb{C}ov}\left[\bm{r}\right] \bm{h}}=\frac{\gamma\bm{h}^{\top} \bm{\mu} + \psi\overbrace{(\bm{h}^{\top} \bm{1} - 1)}^{0}}{\bm{h}^{\top} \bm{V}\bm{h}}\\
    &=\frac{\gamma\sum_{i=1}^S h_i \mu_i}{\sum_{i=1}^S \sum_{j=1}^S h_i V_{ij} h_j} =\gamma\frac{\sum_{i=1}^S  \left(\frac{P_i^0 x_i^*}{(\bm{P}^0)^{\top} \bm{x}^*}\right) \left(\sum_{j=1}^S \Omega_{ij}x^{*}_j \frac{1}{P_i^0}\right) }{\sum_{i=1}^S \sum_{j=1}^S \frac{P_i^0 x_i^*}{(\bm{P}^0)^{\top} \bm{x}^*} V_{ij} \frac{P_j^0 x_j^*}{(\bm{P}^0)^{\top} \bm{x}^*}}\\
    &=\gamma(\bm{P}^0)^{\top} \bm{x}^*\frac{\sum_{i=1}^S   \sum_{j=1}^S \textcolor{red}{P_i^0} \textcolor{blue}{x_i^*P_i^0V_{ij}P_j^0x^{*}_j} \textcolor{red}{\frac{1}{P_i^0}} }{\sum_{i=1}^S \sum_{j=1}^S  \textcolor{blue}{x_i^* P_i^0 V_{ij} P_j^0 x_j^*}} \\
    &=\gamma(\bm{P}^0)^{\top} \bm{x}^*
  \end{split}
\end{equation}

\subsection{(d)}
Use the results of \textbf{(c)}:
\begin{equation}
  \begin{split}
    RHS &= \psi + \beta+s (\mathbb{E}\left[r_M\right] - \psi) =\psi + \mathrm{\mathbb{C}ov}\left[r_s, r_M\right]\frac{\mathbb{E}\left[r_M\right] - \psi}{\mathrm{\mathbb{V}ar}\left[r_M\right]}\\
    &=\psi + \mathrm{\mathbb{C}ov}[r_s, \Sigma_{j=1}^S h_j r_j] \gamma(\bm{P}^0)^{\top} \bm{x}^* =\psi + \gamma(\bm{P}^0)^{\top} \bm{x}^* \sum_{j=1}^S V_{sj}h_j \\
    &=\psi + \gamma(\bm{P}^0)^{\top} \bm{x}^* \sum_{j=1}^S \frac{\Omega_{sj}}{P_s^0 P_j^0} \frac{P_j^0 x_j^*}{(\bm{P}^0)^{\top} \bm{x}^*} \\
    &=\psi + \gamma \sum_{j=1}^S \frac{\Omega_{sj}x_j^*}{P_s^0} = \psi + \gamma \frac{1}{P_s^0}(\bm{\Omega} \bm{x}^*)_s = \mathbb{E}\left[r_s\right]\quad\text{(By the result of \textbf{(b)})}
  \end{split}
\end{equation}
Which completes the proof.
    

    


    \section{Problem 2}\label{problem-2}
\emph{Unit for returns: percent, the same as what was used in French’s Data Library}
\subsection{(a)}\label{a}


            \begin{Verbatim}[commandchars=\\\{\}]
{\color{outcolor}Out[{\color{outcolor}72}]:}                  BM1       BM2       BM3       BM4       BM5
         ME1 mean    0.138941  0.721294  0.725931  0.938094  1.013257
             se      0.322117  0.280844  0.242438  0.229816  0.243004
             t-stat  0.431337  2.568314  2.994295  4.081942  4.169705
         ME2 mean    0.451483  0.732897  0.801649  0.875069  0.916870
             se      0.293281  0.244021  0.220284  0.211684  0.244061
             t-stat  1.539423  3.003419  3.639155  4.133851  3.756730
         ME3 mean    0.481265  0.765048  0.701462  0.823439  0.993619
             se      0.269862  0.221977  0.203838  0.198656  0.229668
             t-stat  1.783373  3.446511  3.441274  4.145044  4.326334
         ME4 mean    0.614499  0.622239  0.681545  0.794719  0.791070
             se      0.242311  0.210575  0.202576  0.193848  0.230790
             t-stat  2.535990  2.954949  3.364389  4.099694  3.427656
         ME5 mean    0.501318  0.573863  0.571117  0.473859  0.679433
             se      0.189887  0.181751  0.175723  0.189212  0.219410
             t-stat  2.640088  3.157413  3.250092  2.504379  3.096636
\end{Verbatim}
        
    \subsection{(b)}\label{b}


    \begin{center}
    \adjustimage{max size={0.8\linewidth}{0.9\paperheight}}{output_5_0.png}
    \end{center}
    
    \begin{itemize}
\tightlist
\item
  For the same BM quantile group (above BM2), we can observe a size
  effect across different size groups, in the sense that the excess
  return increases as size group goes from BIG to SMALL. (Figure 1)
\item
  The size effect described above is not evident for LoBM(BM1) group:
  the excess returns do not increase monotonically.
\item
  For the same size quantile group, we can observe a value effect across
  different BM groups, in the sense that the excess return increases as
  BM group goes from LowBM to HighBM. (Figure 2)
\item
  The value effect is stronger in the SMALL(ME1) size group, as the blue
  line in figure 2 has the steepest slope.
\end{itemize}

    \subsection{(c)}\label{c}


            \begin{Verbatim}[commandchars=\\\{\}]
{\color{outcolor}Out[{\color{outcolor}69}]:}                   BM1        BM2       BM3        BM4       BM5
         ME1 alpha   -0.604325  0.0792367  0.151235   0.410014  0.462256
             t-stat   -3.08398   0.456519     1.079    2.91167   3.03706
             p<0.05        (*)                             (*)       (*)
         ME2 alpha    -0.28214   0.117712  0.252293   0.352649  0.336371
             t-stat    -1.8987   0.975059   2.24046     3.1656   2.40059
             p<0.05                             (*)        (*)       (*)
         ME3 alpha   -0.214433   0.180408   0.17511   0.326279  0.443554
             t-stat   -1.73615    1.94766   1.88901    3.24615   3.41398
             p<0.05                                        (*)       (*)
         ME4 alpha   -0.033038  0.0555156  0.153521   0.299895  0.225168
             t-stat  -0.352906    0.71254   1.73193    3.25954   1.82287
             p<0.05                                        (*)          
         ME5 alpha  -0.0130529  0.0802354  0.124674  0.0141237  0.179359
             t-stat  -0.194115    1.27065   1.47073   0.136355   1.31604
             p<0.05                                                     
\end{Verbatim}
        

    \begin{center}
    \adjustimage{max size={0.8\linewidth}{0.9\paperheight}}{output_9_0.png}
    \end{center}

    
    \subsection{(d)}\label{d}


            \begin{Verbatim}[commandchars=\\\{\}]
{\color{outcolor}Out[{\color{outcolor}68}]:}                   BM1         BM2         BM3        BM4        BM5
         ME1 alpha   -0.547419 -0.00188453  -0.0285962   0.169426   0.121075
             t-stat   -5.62311  -0.0260974   -0.523543    3.10497    2.11213
             p<0.05        (*)                                (*)        (*)
         ME2 alpha   -0.167721   0.0232988   0.0469446  0.0768339 -0.0407018
             t-stat   -2.46079    0.395727    0.793127    1.48421  -0.733885
             p<0.05        (*)                                              
         ME3 alpha  -0.0678117   0.0795249  -0.0269168  0.0524966   0.079584
             t-stat   -1.09019     1.19467   -0.405653   0.817183      1.015
             p<0.05                                                         
         ME4 alpha     0.11902  -0.0376912  -0.0348759  0.0584575  -0.119813
             t-stat    1.91318   -0.521096   -0.468031   0.859294   -1.39163
             p<0.05                                                         
         ME5 alpha    0.162147   0.0504822  0.00859773  -0.248215  -0.157442
             t-stat    3.50773    0.881661    0.123204   -3.80089   -1.57558
             p<0.05        (*)                                (*)           
\end{Verbatim}
\textbf{Comments:}
\begin{itemize}
    \item The alphas that are statistically significant in part (c), i.e. the CAPM single factor model becomes either
    \begin{itemize}
        \item Much smaller in scale, in terms of the absolute values. When we inspect the plots of alphas across different groups, we can see a much messy pattern as opposed to the CAPM alphas. OR
        \item Becomes not statistically significant. The $(*)$ in table marks those alphas with p-value < 0.05. We can observe that many alphas in the top-right corrner (small size, high B/M) that are previously significant now becomes not significant to the confidence level of 0.95.
    \end{itemize}
    \item These observations bespeaks the fact that a considerable amount of the CAPM alphas on the size$\times$value quantile portfolios can be explained by the SMB and HML risk factors.
\end{itemize}
        

    \begin{center}
    \adjustimage{max size={0.8\linewidth}{0.9\paperheight}}{output_12_1.png}
    \end{center}
    
    \subsection{(e)}\label{e}

    \begin{center}
    \adjustimage{max size={0.8\linewidth}{0.9\paperheight}}{output_14_1.png}
    \end{center}

    \begin{Verbatim}[commandchars=\\\{\}]
# Repeat Part (a)
                 BM1       BM2       BM3       BM4       BM5
ME1 mean    0.279520  0.911910  0.851436  1.074964  1.078133
    se      0.397315  0.346695  0.281501  0.270598  0.282844
    t-stat  0.703521  2.630298  3.024629  3.972553  3.811763
ME2 mean    0.673830  0.906238  0.946844  0.917159  0.928510
    se      0.350828  0.284240  0.255377  0.254190  0.302355
    t-stat  1.920685  3.188284  3.707637  3.608169  3.070930
ME3 mean    0.687848  0.906062  0.834394  0.920435  1.059032
    se      0.323769  0.259975  0.240980  0.243974  0.278055
    t-stat  2.124506  3.485189  3.462497  3.772677  3.808714
ME4 mean    0.880749  0.878720  0.767234  0.923185  0.851772
    se      0.291348  0.238622  0.247059  0.232234  0.283599
    t-stat  3.023010  3.682481  3.105468  3.975237  3.003433
ME5 mean    0.772807  0.751408  0.772299  0.505007  0.801949
    se      0.221474  0.210062  0.208792  0.244736  0.295633
    t-stat  3.489385  3.577085  3.698892  2.063480  2.712648


# Repeat Part (b): see the plots above.


# Repeat Part (c)
                  BM1        BM2        BM3       BM4        BM5
ME1 alpha   -0.693204  0.0743478   0.127975   0.41341   0.376434
    t-stat    -2.5771   0.311648   0.719195   2.25265    1.99988
    p<0.05        (*)                             (*)        (*)
ME2 alpha    -0.27519    0.12434   0.254925  0.235765   0.140363
    t-stat   -1.36331   0.789451    1.74087   1.58208   0.751792
    p<0.05                                                      
ME3 alpha   -0.214548   0.147307   0.146577  0.256493   0.327003
    t-stat   -1.23544    1.22032    1.21334    1.8511    1.93806
    p<0.05                                                      
ME4 alpha   0.0251955   0.170176  0.0642887  0.267753  0.0885353
    t-stat   0.190777    1.64981   0.514063   2.22401    0.53679
    p<0.05                                        (*)           
ME5 alpha   0.0923624    0.11796   0.190769 -0.144086  0.0422177
    t-stat    1.17026    1.40275    1.70035  -0.98358   0.225891
    p<0.05        


# Repeat Part (d)                                              
                 BM1        BM2        BM3        BM4        BM5
ME1 alpha  -0.634844   0.047471  0.0315884   0.270725   0.175225
    t-stat  -4.60346   0.448571   0.424996    3.60445    2.33627
    p<0.05       (*)                              (*)        (*)
ME2 alpha  -0.203053  0.0659427   0.119476  0.0594593 -0.0984468
    t-stat  -2.24812   0.837165    1.53178   0.879905    -1.3781
    p<0.05       (*)                                            
ME3 alpha  -0.108831  0.0794007  0.0143403   0.078121   0.101301
    t-stat   -1.3034   0.873981   0.162489   0.856674   0.945184
    p<0.05                                                      
ME4 alpha    0.11877  0.0891055 -0.0694418   0.124072  -0.125605
    t-stat   1.43847   0.965733  -0.680546      1.353   -1.07457
    p<0.05                                                      
ME5 alpha   0.189886   0.079747  0.0923893    -0.3241  -0.172125
    t-stat   3.71963    1.11945     1.1159   -3.70759   -1.28333
    p<0.05       (*)                              (*)           

    \end{Verbatim}
We used 1988 - 2018, i.e. recent 30 years for this section.
\textbf{Differences:}
\begin{itemize}
    \item In the recent period, the excess return profile across size$\times$ value double sorts becomes much more messy. As in the plots we can't find the downward/upward sloping average returns.
    \item Less portfolios from the double sorts have statisticallty significant CAPM alphas and FF3 alphas: the market becomes more efficient.
\end{itemize}
\textbf{Similarities:}
\begin{itemize}
    \item The (Small size $\times$ high BM) portfolios in the top-right corner still has significant CAPM alphas, and that alpha reduces in a similar fashion as we account for FF3 factors.
\end{itemize}

    
    \section{Problem 3}\label{problem-3}
    \textbf{Note: all the sharpe ratios are annualized, i.e. scaled by multiplying $\sqrt{12}$}

\subsection{(a)}\label{a}

    \begin{Verbatim}[commandchars=\\\{\}]
Sharpe ratio (1991/01 - 2018/10): 0.6705
Sharpe ratio (1991/01 - 2005/12): 0.7862
Sharpe ratio (2006/01 - 2018/10): 0.5523

    \end{Verbatim}

    \begin{center}
    \adjustimage{max size={0.7\linewidth}{0.9\paperheight}}{output_17_1.png}
    \end{center}
    { \hspace*{\fill} \\}
    
    \subsection{(b)}\label{b}
    \begin{Verbatim}[commandchars=\\\{\}]
Sharpe ratio (1991/01 - 2018/10): 0.6916
Sharpe ratio (1991/01 - 2005/12): 0.9773
Sharpe ratio (2006/01 - 2018/10): 0.4414

    \end{Verbatim}

    \begin{center}
    \adjustimage{max size={0.7\linewidth}{0.9\paperheight}}{output_19_1.png}
    \end{center}
    { \hspace*{\fill} \\}
    
    \subsection{(c)}\label{c}

    \begin{Verbatim}[commandchars=\\\{\}]
Sharpe ratio (1991/01 - 2018/10): 0.7013
Sharpe ratio (1991/01 - 2005/12): 1.0266
Sharpe ratio (2006/01 - 2018/10): 0.4248

    \end{Verbatim}

    \begin{center}
    \adjustimage{max size={0.7\linewidth}{0.5\paperheight}}{output_21_1.png}
    \end{center}
    { \hspace*{\fill} \\}
    
    \subsection{(d)}\label{d}

    \begin{Verbatim}[commandchars=\\\{\}]
Sharpe ratio (1991/01 - 2018/10): 0.7074
Sharpe ratio (1991/01 - 2005/12): 1.0659
Sharpe ratio (2006/01 - 2018/10): 0.4083

    \end{Verbatim}

    \begin{center}
    \adjustimage{max size={0.7\linewidth}{0.9\paperheight}}{output_23_1.png}
    \end{center}
    { \hspace*{\fill} \\}
    
    \subsection{(e)}\label{e}
    \textbf{Comments:}
    \begin{itemize}
        \item In terms of aggressiveness, we have (d) > (c) > (b) > (a). 
        \item More aggressive strategy has higher return and sharpe ratio in 1991-2018 (``overall''), and 1991-2005 (``good periods''); and outperforms the benchmark in those periods.
        \item More aggressive strategy has lower return and sharpe ratio in 2006-2018 (``bad/high volatility periods''); and underperforms the benchmark in those periods.
    \end{itemize}
    \textbf{Limitations:}
    \begin{itemize}
        \item The margin requirements ask for large amount of capital to implement more aggressive strategies.
        \item Short selling may not be easy for some countries/exchanges.
        \item The aggressive strategies have higher drawdowns in the ``bad'' periods, which may cause the fund to blow up.
    \end{itemize}

    % Add a bibliography block to the postdoc
    
    
    
    \end{document}


% Default to the notebook output style

    


% Inherit from the specified cell style.




    
\documentclass[11pt]{article}

    
    
    \usepackage[T1]{fontenc}
    % Nicer default font (+ math font) than Computer Modern for most use cases
    %\usepackage{mathpazo}

    % Basic figure setup, for now with no caption control since it's done
    % automatically by Pandoc (which extracts ![](path) syntax from Markdown).
    \usepackage{graphicx}
    % We will generate all images so they have a width \maxwidth. This means
    % that they will get their normal width if they fit onto the page, but
    % are scaled down if they would overflow the margins.
    \makeatletter
    \def\maxwidth{\ifdim\Gin@nat@width>\linewidth\linewidth
    \else\Gin@nat@width\fi}
    \makeatother
    \let\Oldincludegraphics\includegraphics
    % Set max figure width to be 80% of text width, for now hardcoded.
    \renewcommand{\includegraphics}[1]{\Oldincludegraphics[width=.8\maxwidth]{#1}}
    % Ensure that by default, figures have no caption (until we provide a
    % proper Figure object with a Caption API and a way to capture that
    % in the conversion process - todo).
    \usepackage{caption}
    \DeclareCaptionLabelFormat{nolabel}{}
    \captionsetup{labelformat=nolabel}

    \usepackage{adjustbox} % Used to constrain images to a maximum size 
    \usepackage{xcolor} % Allow colors to be defined
    \usepackage{enumerate} % Needed for markdown enumerations to work
    \usepackage{geometry} % Used to adjust the document margins
    \usepackage{amsmath} % Equations
    \usepackage{amssymb,bm,bbm} % Equations
    \usepackage{textcomp} % defines textquotesingle
    % Hack from http://tex.stackexchange.com/a/47451/13684:
    \AtBeginDocument{%
        \def\PYZsq{\textquotesingle}% Upright quotes in Pygmentized code
    }
    \usepackage{upquote} % Upright quotes for verbatim code
    \usepackage{eurosym} % defines \euro
    \usepackage[mathletters]{ucs} % Extended unicode (utf-8) support
    \usepackage[utf8x]{inputenc} % Allow utf-8 characters in the tex document
    \usepackage{fancyvrb} % verbatim replacement that allows latex
    \usepackage{grffile} % extends the file name processing of package graphics 
                         % to support a larger range 
    % The hyperref package gives us a pdf with properly built
    % internal navigation ('pdf bookmarks' for the table of contents,
    % internal cross-reference links, web links for URLs, etc.)
    \usepackage{hyperref}
    \usepackage{longtable} % longtable support required by pandoc >1.10
    \usepackage{booktabs}  % table support for pandoc > 1.12.2
    \usepackage[inline]{enumitem} % IRkernel/repr support (it uses the enumerate* environment)
    \usepackage[normalem]{ulem} % ulem is needed to support strikethroughs (\sout)
                                % normalem makes italics be italics, not underlines
    

    
    
    % Colors for the hyperref package
    \definecolor{urlcolor}{rgb}{0,.145,.698}
    \definecolor{linkcolor}{rgb}{.71,0.21,0.01}
    \definecolor{citecolor}{rgb}{.12,.54,.11}

    % ANSI colors
    \definecolor{ansi-black}{HTML}{3E424D}
    \definecolor{ansi-black-intense}{HTML}{282C36}
    \definecolor{ansi-red}{HTML}{E75C58}
    \definecolor{ansi-red-intense}{HTML}{B22B31}
    \definecolor{ansi-green}{HTML}{00A250}
    \definecolor{ansi-green-intense}{HTML}{007427}
    \definecolor{ansi-yellow}{HTML}{DDB62B}
    \definecolor{ansi-yellow-intense}{HTML}{B27D12}
    \definecolor{ansi-blue}{HTML}{208FFB}
    \definecolor{ansi-blue-intense}{HTML}{0065CA}
    \definecolor{ansi-magenta}{HTML}{D160C4}
    \definecolor{ansi-magenta-intense}{HTML}{A03196}
    \definecolor{ansi-cyan}{HTML}{60C6C8}
    \definecolor{ansi-cyan-intense}{HTML}{258F8F}
    \definecolor{ansi-white}{HTML}{C5C1B4}
    \definecolor{ansi-white-intense}{HTML}{A1A6B2}

    % commands and environments needed by pandoc snippets
    % extracted from the output of `pandoc -s`
    \providecommand{\tightlist}{%
      \setlength{\itemsep}{0pt}\setlength{\parskip}{0pt}}
    \DefineVerbatimEnvironment{Highlighting}{Verbatim}{commandchars=\\\{\}}
    % Add ',fontsize=\small' for more characters per line
    \newenvironment{Shaded}{}{}
    \newcommand{\KeywordTok}[1]{\textcolor[rgb]{0.00,0.44,0.13}{\textbf{{#1}}}}
    \newcommand{\DataTypeTok}[1]{\textcolor[rgb]{0.56,0.13,0.00}{{#1}}}
    \newcommand{\DecValTok}[1]{\textcolor[rgb]{0.25,0.63,0.44}{{#1}}}
    \newcommand{\BaseNTok}[1]{\textcolor[rgb]{0.25,0.63,0.44}{{#1}}}
    \newcommand{\FloatTok}[1]{\textcolor[rgb]{0.25,0.63,0.44}{{#1}}}
    \newcommand{\CharTok}[1]{\textcolor[rgb]{0.25,0.44,0.63}{{#1}}}
    \newcommand{\StringTok}[1]{\textcolor[rgb]{0.25,0.44,0.63}{{#1}}}
    \newcommand{\CommentTok}[1]{\textcolor[rgb]{0.38,0.63,0.69}{\textit{{#1}}}}
    \newcommand{\OtherTok}[1]{\textcolor[rgb]{0.00,0.44,0.13}{{#1}}}
    \newcommand{\AlertTok}[1]{\textcolor[rgb]{1.00,0.00,0.00}{\textbf{{#1}}}}
    \newcommand{\FunctionTok}[1]{\textcolor[rgb]{0.02,0.16,0.49}{{#1}}}
    \newcommand{\RegionMarkerTok}[1]{{#1}}
    \newcommand{\ErrorTok}[1]{\textcolor[rgb]{1.00,0.00,0.00}{\textbf{{#1}}}}
    \newcommand{\NormalTok}[1]{{#1}}
    
    % Additional commands for more recent versions of Pandoc
    \newcommand{\ConstantTok}[1]{\textcolor[rgb]{0.53,0.00,0.00}{{#1}}}
    \newcommand{\SpecialCharTok}[1]{\textcolor[rgb]{0.25,0.44,0.63}{{#1}}}
    \newcommand{\VerbatimStringTok}[1]{\textcolor[rgb]{0.25,0.44,0.63}{{#1}}}
    \newcommand{\SpecialStringTok}[1]{\textcolor[rgb]{0.73,0.40,0.53}{{#1}}}
    \newcommand{\ImportTok}[1]{{#1}}
    \newcommand{\DocumentationTok}[1]{\textcolor[rgb]{0.73,0.13,0.13}{\textit{{#1}}}}
    \newcommand{\AnnotationTok}[1]{\textcolor[rgb]{0.38,0.63,0.69}{\textbf{\textit{{#1}}}}}
    \newcommand{\CommentVarTok}[1]{\textcolor[rgb]{0.38,0.63,0.69}{\textbf{\textit{{#1}}}}}
    \newcommand{\VariableTok}[1]{\textcolor[rgb]{0.10,0.09,0.49}{{#1}}}
    \newcommand{\ControlFlowTok}[1]{\textcolor[rgb]{0.00,0.44,0.13}{\textbf{{#1}}}}
    \newcommand{\OperatorTok}[1]{\textcolor[rgb]{0.40,0.40,0.40}{{#1}}}
    \newcommand{\BuiltInTok}[1]{{#1}}
    \newcommand{\ExtensionTok}[1]{{#1}}
    \newcommand{\PreprocessorTok}[1]{\textcolor[rgb]{0.74,0.48,0.00}{{#1}}}
    \newcommand{\AttributeTok}[1]{\textcolor[rgb]{0.49,0.56,0.16}{{#1}}}
    \newcommand{\InformationTok}[1]{\textcolor[rgb]{0.38,0.63,0.69}{\textbf{\textit{{#1}}}}}
    \newcommand{\WarningTok}[1]{\textcolor[rgb]{0.38,0.63,0.69}{\textbf{\textit{{#1}}}}}
    
    
    % Define a nice break command that doesn't care if a line doesn't already
    % exist.
    \def\br{\hspace*{\fill} \\* }
    % Math Jax compatability definitions
    \def\gt{>}
    \def\lt{<}
    % Document parameters
    \title{Asset Management HW2}
    \author{Ze Yang}
    
    
    

    % Pygments definitions
    
\makeatletter
\def\PY@reset{\let\PY@it=\relax \let\PY@bf=\relax%
    \let\PY@ul=\relax \let\PY@tc=\relax%
    \let\PY@bc=\relax \let\PY@ff=\relax}
\def\PY@tok#1{\csname PY@tok@#1\endcsname}
\def\PY@toks#1+{\ifx\relax#1\empty\else%
    \PY@tok{#1}\expandafter\PY@toks\fi}
\def\PY@do#1{\PY@bc{\PY@tc{\PY@ul{%
    \PY@it{\PY@bf{\PY@ff{#1}}}}}}}
\def\PY#1#2{\PY@reset\PY@toks#1+\relax+\PY@do{#2}}

\expandafter\def\csname PY@tok@w\endcsname{\def\PY@tc##1{\textcolor[rgb]{0.73,0.73,0.73}{##1}}}
\expandafter\def\csname PY@tok@c\endcsname{\let\PY@it=\textit\def\PY@tc##1{\textcolor[rgb]{0.25,0.50,0.50}{##1}}}
\expandafter\def\csname PY@tok@cp\endcsname{\def\PY@tc##1{\textcolor[rgb]{0.74,0.48,0.00}{##1}}}
\expandafter\def\csname PY@tok@k\endcsname{\let\PY@bf=\textbf\def\PY@tc##1{\textcolor[rgb]{0.00,0.50,0.00}{##1}}}
\expandafter\def\csname PY@tok@kp\endcsname{\def\PY@tc##1{\textcolor[rgb]{0.00,0.50,0.00}{##1}}}
\expandafter\def\csname PY@tok@kt\endcsname{\def\PY@tc##1{\textcolor[rgb]{0.69,0.00,0.25}{##1}}}
\expandafter\def\csname PY@tok@o\endcsname{\def\PY@tc##1{\textcolor[rgb]{0.40,0.40,0.40}{##1}}}
\expandafter\def\csname PY@tok@ow\endcsname{\let\PY@bf=\textbf\def\PY@tc##1{\textcolor[rgb]{0.67,0.13,1.00}{##1}}}
\expandafter\def\csname PY@tok@nb\endcsname{\def\PY@tc##1{\textcolor[rgb]{0.00,0.50,0.00}{##1}}}
\expandafter\def\csname PY@tok@nf\endcsname{\def\PY@tc##1{\textcolor[rgb]{0.00,0.00,1.00}{##1}}}
\expandafter\def\csname PY@tok@nc\endcsname{\let\PY@bf=\textbf\def\PY@tc##1{\textcolor[rgb]{0.00,0.00,1.00}{##1}}}
\expandafter\def\csname PY@tok@nn\endcsname{\let\PY@bf=\textbf\def\PY@tc##1{\textcolor[rgb]{0.00,0.00,1.00}{##1}}}
\expandafter\def\csname PY@tok@ne\endcsname{\let\PY@bf=\textbf\def\PY@tc##1{\textcolor[rgb]{0.82,0.25,0.23}{##1}}}
\expandafter\def\csname PY@tok@nv\endcsname{\def\PY@tc##1{\textcolor[rgb]{0.10,0.09,0.49}{##1}}}
\expandafter\def\csname PY@tok@no\endcsname{\def\PY@tc##1{\textcolor[rgb]{0.53,0.00,0.00}{##1}}}
\expandafter\def\csname PY@tok@nl\endcsname{\def\PY@tc##1{\textcolor[rgb]{0.63,0.63,0.00}{##1}}}
\expandafter\def\csname PY@tok@ni\endcsname{\let\PY@bf=\textbf\def\PY@tc##1{\textcolor[rgb]{0.60,0.60,0.60}{##1}}}
\expandafter\def\csname PY@tok@na\endcsname{\def\PY@tc##1{\textcolor[rgb]{0.49,0.56,0.16}{##1}}}
\expandafter\def\csname PY@tok@nt\endcsname{\let\PY@bf=\textbf\def\PY@tc##1{\textcolor[rgb]{0.00,0.50,0.00}{##1}}}
\expandafter\def\csname PY@tok@nd\endcsname{\def\PY@tc##1{\textcolor[rgb]{0.67,0.13,1.00}{##1}}}
\expandafter\def\csname PY@tok@s\endcsname{\def\PY@tc##1{\textcolor[rgb]{0.73,0.13,0.13}{##1}}}
\expandafter\def\csname PY@tok@sd\endcsname{\let\PY@it=\textit\def\PY@tc##1{\textcolor[rgb]{0.73,0.13,0.13}{##1}}}
\expandafter\def\csname PY@tok@si\endcsname{\let\PY@bf=\textbf\def\PY@tc##1{\textcolor[rgb]{0.73,0.40,0.53}{##1}}}
\expandafter\def\csname PY@tok@se\endcsname{\let\PY@bf=\textbf\def\PY@tc##1{\textcolor[rgb]{0.73,0.40,0.13}{##1}}}
\expandafter\def\csname PY@tok@sr\endcsname{\def\PY@tc##1{\textcolor[rgb]{0.73,0.40,0.53}{##1}}}
\expandafter\def\csname PY@tok@ss\endcsname{\def\PY@tc##1{\textcolor[rgb]{0.10,0.09,0.49}{##1}}}
\expandafter\def\csname PY@tok@sx\endcsname{\def\PY@tc##1{\textcolor[rgb]{0.00,0.50,0.00}{##1}}}
\expandafter\def\csname PY@tok@m\endcsname{\def\PY@tc##1{\textcolor[rgb]{0.40,0.40,0.40}{##1}}}
\expandafter\def\csname PY@tok@gh\endcsname{\let\PY@bf=\textbf\def\PY@tc##1{\textcolor[rgb]{0.00,0.00,0.50}{##1}}}
\expandafter\def\csname PY@tok@gu\endcsname{\let\PY@bf=\textbf\def\PY@tc##1{\textcolor[rgb]{0.50,0.00,0.50}{##1}}}
\expandafter\def\csname PY@tok@gd\endcsname{\def\PY@tc##1{\textcolor[rgb]{0.63,0.00,0.00}{##1}}}
\expandafter\def\csname PY@tok@gi\endcsname{\def\PY@tc##1{\textcolor[rgb]{0.00,0.63,0.00}{##1}}}
\expandafter\def\csname PY@tok@gr\endcsname{\def\PY@tc##1{\textcolor[rgb]{1.00,0.00,0.00}{##1}}}
\expandafter\def\csname PY@tok@ge\endcsname{\let\PY@it=\textit}
\expandafter\def\csname PY@tok@gs\endcsname{\let\PY@bf=\textbf}
\expandafter\def\csname PY@tok@gp\endcsname{\let\PY@bf=\textbf\def\PY@tc##1{\textcolor[rgb]{0.00,0.00,0.50}{##1}}}
\expandafter\def\csname PY@tok@go\endcsname{\def\PY@tc##1{\textcolor[rgb]{0.53,0.53,0.53}{##1}}}
\expandafter\def\csname PY@tok@gt\endcsname{\def\PY@tc##1{\textcolor[rgb]{0.00,0.27,0.87}{##1}}}
\expandafter\def\csname PY@tok@err\endcsname{\def\PY@bc##1{\setlength{\fboxsep}{0pt}\fcolorbox[rgb]{1.00,0.00,0.00}{1,1,1}{\strut ##1}}}
\expandafter\def\csname PY@tok@kc\endcsname{\let\PY@bf=\textbf\def\PY@tc##1{\textcolor[rgb]{0.00,0.50,0.00}{##1}}}
\expandafter\def\csname PY@tok@kd\endcsname{\let\PY@bf=\textbf\def\PY@tc##1{\textcolor[rgb]{0.00,0.50,0.00}{##1}}}
\expandafter\def\csname PY@tok@kn\endcsname{\let\PY@bf=\textbf\def\PY@tc##1{\textcolor[rgb]{0.00,0.50,0.00}{##1}}}
\expandafter\def\csname PY@tok@kr\endcsname{\let\PY@bf=\textbf\def\PY@tc##1{\textcolor[rgb]{0.00,0.50,0.00}{##1}}}
\expandafter\def\csname PY@tok@bp\endcsname{\def\PY@tc##1{\textcolor[rgb]{0.00,0.50,0.00}{##1}}}
\expandafter\def\csname PY@tok@fm\endcsname{\def\PY@tc##1{\textcolor[rgb]{0.00,0.00,1.00}{##1}}}
\expandafter\def\csname PY@tok@vc\endcsname{\def\PY@tc##1{\textcolor[rgb]{0.10,0.09,0.49}{##1}}}
\expandafter\def\csname PY@tok@vg\endcsname{\def\PY@tc##1{\textcolor[rgb]{0.10,0.09,0.49}{##1}}}
\expandafter\def\csname PY@tok@vi\endcsname{\def\PY@tc##1{\textcolor[rgb]{0.10,0.09,0.49}{##1}}}
\expandafter\def\csname PY@tok@vm\endcsname{\def\PY@tc##1{\textcolor[rgb]{0.10,0.09,0.49}{##1}}}
\expandafter\def\csname PY@tok@sa\endcsname{\def\PY@tc##1{\textcolor[rgb]{0.73,0.13,0.13}{##1}}}
\expandafter\def\csname PY@tok@sb\endcsname{\def\PY@tc##1{\textcolor[rgb]{0.73,0.13,0.13}{##1}}}
\expandafter\def\csname PY@tok@sc\endcsname{\def\PY@tc##1{\textcolor[rgb]{0.73,0.13,0.13}{##1}}}
\expandafter\def\csname PY@tok@dl\endcsname{\def\PY@tc##1{\textcolor[rgb]{0.73,0.13,0.13}{##1}}}
\expandafter\def\csname PY@tok@s2\endcsname{\def\PY@tc##1{\textcolor[rgb]{0.73,0.13,0.13}{##1}}}
\expandafter\def\csname PY@tok@sh\endcsname{\def\PY@tc##1{\textcolor[rgb]{0.73,0.13,0.13}{##1}}}
\expandafter\def\csname PY@tok@s1\endcsname{\def\PY@tc##1{\textcolor[rgb]{0.73,0.13,0.13}{##1}}}
\expandafter\def\csname PY@tok@mb\endcsname{\def\PY@tc##1{\textcolor[rgb]{0.40,0.40,0.40}{##1}}}
\expandafter\def\csname PY@tok@mf\endcsname{\def\PY@tc##1{\textcolor[rgb]{0.40,0.40,0.40}{##1}}}
\expandafter\def\csname PY@tok@mh\endcsname{\def\PY@tc##1{\textcolor[rgb]{0.40,0.40,0.40}{##1}}}
\expandafter\def\csname PY@tok@mi\endcsname{\def\PY@tc##1{\textcolor[rgb]{0.40,0.40,0.40}{##1}}}
\expandafter\def\csname PY@tok@il\endcsname{\def\PY@tc##1{\textcolor[rgb]{0.40,0.40,0.40}{##1}}}
\expandafter\def\csname PY@tok@mo\endcsname{\def\PY@tc##1{\textcolor[rgb]{0.40,0.40,0.40}{##1}}}
\expandafter\def\csname PY@tok@ch\endcsname{\let\PY@it=\textit\def\PY@tc##1{\textcolor[rgb]{0.25,0.50,0.50}{##1}}}
\expandafter\def\csname PY@tok@cm\endcsname{\let\PY@it=\textit\def\PY@tc##1{\textcolor[rgb]{0.25,0.50,0.50}{##1}}}
\expandafter\def\csname PY@tok@cpf\endcsname{\let\PY@it=\textit\def\PY@tc##1{\textcolor[rgb]{0.25,0.50,0.50}{##1}}}
\expandafter\def\csname PY@tok@c1\endcsname{\let\PY@it=\textit\def\PY@tc##1{\textcolor[rgb]{0.25,0.50,0.50}{##1}}}
\expandafter\def\csname PY@tok@cs\endcsname{\let\PY@it=\textit\def\PY@tc##1{\textcolor[rgb]{0.25,0.50,0.50}{##1}}}

\def\PYZbs{\char`\\}
\def\PYZus{\char`\_}
\def\PYZob{\char`\{}
\def\PYZcb{\char`\}}
\def\PYZca{\char`\^}
\def\PYZam{\char`\&}
\def\PYZlt{\char`\<}
\def\PYZgt{\char`\>}
\def\PYZsh{\char`\#}
\def\PYZpc{\char`\%}
\def\PYZdl{\char`\$}
\def\PYZhy{\char`\-}
\def\PYZsq{\char`\'}
\def\PYZdq{\char`\"}
\def\PYZti{\char`\~}
% for compatibility with earlier versions
\def\PYZat{@}
\def\PYZlb{[}
\def\PYZrb{]}
\makeatother


    % Exact colors from NB
    \definecolor{incolor}{rgb}{0.0, 0.0, 0.5}
    \definecolor{outcolor}{rgb}{0.545, 0.0, 0.0}



    
    % Prevent overflowing lines due to hard-to-break entities
    \sloppy 
    % Setup hyperref package
    \hypersetup{
      breaklinks=true,  % so long urls are correctly broken across lines
      colorlinks=true,
      urlcolor=urlcolor,
      linkcolor=linkcolor,
      citecolor=citecolor,
      }
    % Slightly bigger margins than the latex defaults
    
    \geometry{verbose,tmargin=1in,bmargin=1in,lmargin=1in,rmargin=1in}
    
    

    \begin{document}
    
    
    \maketitle
    
   \section{Problem 1.} 
    \subsection{(a)} 
    By the definition of benchmark portfolio, clearly we have $\mathbb{E}\left[r_B\right] = 0$. To compute $\alpha$ and tracking error, we define $\bm{r} = (r_1, r_2, ..., r_N)^{\top}$, $\tilde{\bm{r}} = (\mathbbm{1}_{\{u_{11}=1\%\}}r_1, \mathbbm{1}_{\{u_{21}=1\%\}}r_2, ..., \mathbbm{1}_{\{u_{N1}=1\%\}}r_N)^{\top}$; $\mathbbm{1}_{\{u_{i1}=1\%\}}$ is the indicator function that equals 1 if $u_{i1} = 1\%$, otherwise 0.\\
Since $\{u_{ij}\}$ are i.i.d., we have $\mathbb{E}\left[\mathbbm{1}_{\{u_{ij}=1\%\}}r_i\right] = \mathbb{E}\left[\mathbbm{1}_{\{u_{ij}=1\%\}}\left(1\%+\sum_{j=2}^M u_{ij}\right)\right] = 1\%/2$. $\mathbb{E}\left[r_i\right] = 0$.\\
Besides, the independence of $u$ suggests $r_i$ are also i.i.d. across difference stock label $i$'s. Hence $\mathrm{\mathbb{V}ar}\left[\bm{r}\right] = 0.01^2 M \bm{I}$
\begin{equation}
  \begin{split}
    \alpha &= \mathbb{E}\left[\frac{2}{N}\bm{1}^{\top}\tilde{\bm{r}} - \frac{1}{N} \bm{1}^{\top} \bm{r}\right] = \frac{2}{N} \bm{1}^{\top} \mathbb{E}\left[\tilde{\bm{r}}\right] - \frac{1}{N}\bm{1}^{\top} \bm{0} \\
    &= \frac{2}{N}\sum_{i=1}^N \frac{1\%}{2} + 0 = 1\%
  \end{split}
 \end{equation} 
For the variance, we first compute the covariance matrix of $\bm{\tilde{\bm{r}}}$. By the independence across stocks, it's also a diagonal matrix. The second moment is
\begin{equation}
  \begin{split}
    \mathbb{E}\left[(\mathbbm{1}_{\{u_{ij}=1\%\}}r_i)^2\right] &= \mathbb{E}\left[\mathbbm{1}^2_{\{u_{ij}=1\%\}}(\sum_{j=1}^M u_{ij})^2\right] \\
  &=\mathbb{E}\left[\mathbbm{1}^2_{\{u_{ij}=1\%\}}\left(1\%^2 + \sum_{j=2}^M u_{ij}^2\right)\right]\\
  &=\left(\frac{1}{2} + \frac{M-1}{2}\right) 1\%^2 = \frac{M}{2}1\%^2
  \end{split}
\end{equation}
Therefore, $\mathrm{\mathbb{V}ar}\left[\tilde{\bm{r}}\right] = \mathbb{E}\left[(\mathbbm{1}_{\{u_{ij}=1\%\}}r_i)^2\right] - \mathbb{E}\left[\mathbbm{1}_{\{u_{ij}=1\%\}}r_i\right]^2 = \frac{0.01^2(2M-1)}{4} \bm{I}$. Moreover, $\mathbb{E}\left[\mathbbm{1}_{\{u_{ij}=1\%\}}r_i^2\right] = \frac{M}{2} \cdot 1\%^2$; so $\mathrm{\mathbb{C}ov}\left[\tilde{\bm{r}}, \bm{r}\right]=\frac{0.01^2M}{2} \bm{I}$
\begin{equation}
  \begin{split}
    \mathrm{\mathbb{V}ar}\left[r-r_B\right] &= \mathrm{\mathbb{V}ar}\left[\frac{2}{N}\bm{1}^{\top}\tilde{\bm{r}} - \frac{1}{N} \bm{1}^{\top} \bm{r}\right] \\
    &=\frac{4}{N^2} \bm{1}^{\top} \mathrm{\mathbb{V}ar}\left[\tilde{\bm{r}}\right] \bm{1} - 2\times\frac{2}{N^2}\bm{1}^{\top}\mathrm{\mathbb{C}ov}\left[\tilde{\bm{r}}, \bm{r}\right]\bm{1} + \frac{1}{N^2} \bm{1}^{\top} \mathrm{\mathbb{V}ar}\left[ \bm{r}\right]\bm{1} \\
    &= \frac{4}{N} \frac{2M-1}{4}0.01^2 - \frac{4}{N} \frac{M}{2}0.01^2 + \frac{1}{N} M0.01^2\\
    &= \frac{M-1}{N}0.01^2
  \end{split}
\end{equation}
Hence $\sigma(r-r_B) = 0.01\sqrt{\frac{M-1}{N}}$. The information ratio, by definition, is
\begin{equation}
  IR = \frac{\alpha}{\sigma(r-r_B)} = \sqrt{\frac{N}{M-1}}
\end{equation}

\subsection{(b)} 
\begin{equation}
  IC = \frac{\mathrm{\mathbb{C}ov}\left[u_{i1}, r_i\right]}{\sigma(u_{i1})\sigma(r_i)} = \frac{0.01^2}{0.01\sqrt{0.01^2M}} = \frac{1}{\sqrt{M}}
\end{equation}
The information ratio implied by fundamental law of AM is therefore $\sqrt{N/M}$, which does not equal to IR in (a) exactly. \\
~\\
There are a few things that can be said about this observation.
\begin{itemize}
  \item[1.] The fundamental law of AM is a rule of thumb. In Grinold and Kahn's derivation, many assumptions are made to prove the law, for example, the active holdings solve the mean-variance problem; the conditional expectation of residual returns (which they call \emph{refined forecast}) can be estiamted by
  $$
  \widehat{\mathbb{E}}\left[r\middle|g\right] = \mathbb{E}\left[r\right] + \mathrm{\mathbb{C}ov}\left[r, g\right] \cdot \mathrm{\mathbb{V}ar}^{-1}\left[g\right] \cdot (g - \mathbb{E}\left[g\right])
  $$
  for some signal $g$. These assumptions does not hold exactly for our case. 
  \item[2.] The information ratio derived in (1) becomes close to $\sqrt{N/M}$ when both $M$ and $N$ are $\gg 1$. 
\end{itemize}

    
\noindent\rule{16cm}{0.4pt}
%///////////////////////////////////////////////////////////////////////

    \section{Problem 2}\label{problem-2}

    \subsection{(a)}\label{a}

    \begin{Verbatim}[commandchars=\\\{\}]
{\color{incolor}In [{\color{incolor}80}]:} \PY{n}{N}\PY{p}{,} \PY{n}{M}\PY{p}{,} \PY{n}{runs} \PY{o}{=} \PY{l+m+mi}{100}\PY{p}{,} \PY{l+m+mi}{200}\PY{p}{,} \PY{l+m+mi}{20000}
         \PY{n}{sample} \PY{o}{=} \PY{n}{np}\PY{o}{.}\PY{n}{zeros}\PY{p}{(}\PY{p}{(}\PY{n}{runs}\PY{p}{,} \PY{l+m+mi}{2}\PY{p}{)}\PY{p}{)}
         \PY{n}{bar} \PY{o}{=} \PY{n}{ProgressBar}\PY{p}{(}\PY{p}{)}
         \PY{k}{for} \PY{n}{i} \PY{o+ow}{in} \PY{n}{bar}\PY{p}{(}\PY{n+nb}{range}\PY{p}{(}\PY{n}{runs}\PY{p}{)}\PY{p}{)}\PY{p}{:}
             \PY{n}{uu} \PY{o}{=} \PY{o}{.}\PY{l+m+mi}{01}\PY{o}{*}\PY{n}{np}\PY{o}{.}\PY{n}{random}\PY{o}{.}\PY{n}{choice}\PY{p}{(}\PY{p}{[}\PY{o}{\PYZhy{}}\PY{l+m+mi}{1}\PY{p}{,}\PY{l+m+mi}{1}\PY{p}{]}\PY{p}{,}\PY{p}{(}\PY{n}{M}\PY{p}{,} \PY{n}{N}\PY{p}{)}\PY{p}{)}
             \PY{n}{u1} \PY{o}{=} \PY{n}{uu}\PY{p}{[}\PY{l+m+mi}{0}\PY{p}{,}\PY{p}{:}\PY{p}{]}
             \PY{n}{r} \PY{o}{=} \PY{p}{(}\PY{l+m+mi}{2}\PY{o}{/}\PY{n}{N}\PY{p}{)}\PY{o}{*}\PY{p}{(}\PY{n}{u1}\PY{o}{\PYZgt{}}\PY{l+m+mi}{0}\PY{p}{)}\PY{o}{*}\PY{p}{(}\PY{n}{uu}\PY{o}{.}\PY{n}{sum}\PY{p}{(}\PY{l+m+mi}{0}\PY{p}{)}\PY{p}{)}
             \PY{n}{rb} \PY{o}{=} \PY{p}{(}\PY{l+m+mi}{1}\PY{o}{/}\PY{n}{N}\PY{p}{)}\PY{o}{*}\PY{p}{(}\PY{n}{uu}\PY{o}{.}\PY{n}{sum}\PY{p}{(}\PY{l+m+mi}{0}\PY{p}{)}\PY{p}{)}
             \PY{n}{sample}\PY{p}{[}\PY{n}{i}\PY{p}{,} \PY{p}{:}\PY{p}{]} \PY{o}{=} \PY{n}{np}\PY{o}{.}\PY{n}{array}\PY{p}{(}\PY{p}{[}\PY{n}{r}\PY{o}{.}\PY{n}{sum}\PY{p}{(}\PY{p}{)}\PY{p}{,} \PY{n}{rb}\PY{o}{.}\PY{n}{sum}\PY{p}{(}\PY{p}{)}\PY{p}{]}\PY{p}{)}
\end{Verbatim}

    \begin{Verbatim}[commandchars=\\\{\}]
100\% (20000 of 20000) |\#\#\#\#\#\#\#\#\#\#\#\#\#\#\#\#\#\#\#| Elapsed Time: 0:00:05 Time: 0:00:05

    \end{Verbatim}

    \begin{Verbatim}[commandchars=\\\{\}]
{\color{incolor}In [{\color{incolor}81}]:} \PY{n}{r\PYZus{}active} \PY{o}{=} \PY{n}{sample}\PY{p}{[}\PY{p}{:}\PY{p}{,} \PY{l+m+mi}{0}\PY{p}{]} \PY{o}{\PYZhy{}} \PY{n}{sample}\PY{p}{[}\PY{p}{:}\PY{p}{,} \PY{l+m+mi}{1}\PY{p}{]}
         \PY{n+nb}{print}\PY{p}{(}\PY{n}{f}\PY{l+s+s1}{\PYZsq{}}\PY{l+s+s1}{Simulated mean = }\PY{l+s+s1}{\PYZob{}}\PY{l+s+s1}{r\PYZus{}active.mean():.6f\PYZcb{}, std = }\PY{l+s+s1}{\PYZob{}}\PY{l+s+s1}{r\PYZus{}active.std():.6f\PYZcb{}}\PY{l+s+s1}{\PYZsq{}}\PY{p}{)}
         \PY{n+nb}{print}\PY{p}{(}\PY{n}{f}\PY{l+s+s1}{\PYZsq{}}\PY{l+s+s1}{Theoretical mean = }\PY{l+s+si}{\PYZob{}0.01:.6f\PYZcb{}}\PY{l+s+s1}{, std = }\PY{l+s+s1}{\PYZob{}}\PY{l+s+s1}{0.01*np.sqrt((M\PYZhy{}1)/N):.6f\PYZcb{}}\PY{l+s+s1}{\PYZsq{}}\PY{p}{)}
\end{Verbatim}

    \begin{Verbatim}[commandchars=\\\{\}]
Simulated mean = 0.009956, std = 0.014129
Theoretical mean = 0.010000, std = 0.014107

    \end{Verbatim}

    The simulated mean and standard deviation of \(r-r_B\) are very close to
the theoretical values, which comfirms the results from problem 1.

    \subsection{(b)}\label{b}

    \begin{Verbatim}[commandchars=\\\{\}]
{\color{incolor}In [{\color{incolor}105}]:} \PY{n}{N}\PY{p}{,} \PY{n}{M}\PY{p}{,} \PY{n}{runs} \PY{o}{=} \PY{l+m+mi}{100}\PY{p}{,} \PY{l+m+mi}{200}\PY{p}{,} \PY{l+m+mi}{100000}
          \PY{n}{sample} \PY{o}{=} \PY{n}{np}\PY{o}{.}\PY{n}{zeros}\PY{p}{(}\PY{p}{(}\PY{n}{runs}\PY{p}{,} \PY{l+m+mi}{2}\PY{p}{)}\PY{p}{)}
          \PY{n}{bar} \PY{o}{=} \PY{n}{ProgressBar}\PY{p}{(}\PY{p}{)}
          \PY{k}{for} \PY{n}{i} \PY{o+ow}{in} \PY{n}{bar}\PY{p}{(}\PY{n+nb}{range}\PY{p}{(}\PY{n}{runs}\PY{p}{)}\PY{p}{)}\PY{p}{:}
              \PY{n}{uu} \PY{o}{=} \PY{o}{.}\PY{l+m+mi}{01}\PY{o}{*}\PY{n}{np}\PY{o}{.}\PY{n}{random}\PY{o}{.}\PY{n}{choice}\PY{p}{(}\PY{p}{[}\PY{o}{\PYZhy{}}\PY{l+m+mi}{1}\PY{p}{,}\PY{l+m+mi}{1}\PY{p}{]}\PY{p}{,}\PY{p}{(}\PY{n}{M}\PY{p}{,} \PY{n}{N}\PY{p}{)}\PY{p}{)}
              \PY{n}{u1} \PY{o}{=} \PY{n}{uu}\PY{p}{[}\PY{l+m+mi}{0}\PY{p}{,}\PY{p}{:}\PY{p}{]}
              \PY{n}{L} \PY{o}{=} \PY{p}{(}\PY{n}{u1}\PY{o}{\PYZgt{}}\PY{l+m+mi}{0}\PY{p}{)}\PY{o}{.}\PY{n}{sum}\PY{p}{(}\PY{p}{)}
              \PY{n}{weight\PYZus{}scale} \PY{o}{=} \PY{l+m+mi}{1}\PY{o}{/}\PY{n}{L} \PY{k}{if} \PY{n}{L}\PY{o}{\PYZgt{}}\PY{l+m+mi}{0} \PY{k}{else} \PY{l+m+mi}{0}
              \PY{n}{r} \PY{o}{=} \PY{n}{weight\PYZus{}scale}\PY{o}{*}\PY{p}{(}\PY{n}{u1}\PY{o}{\PYZgt{}}\PY{l+m+mi}{0}\PY{p}{)}\PY{o}{*}\PY{p}{(}\PY{n}{uu}\PY{o}{.}\PY{n}{sum}\PY{p}{(}\PY{l+m+mi}{0}\PY{p}{)}\PY{p}{)}
              \PY{n}{rb} \PY{o}{=} \PY{p}{(}\PY{l+m+mi}{1}\PY{o}{/}\PY{n}{N}\PY{p}{)}\PY{o}{*}\PY{p}{(}\PY{n}{uu}\PY{o}{.}\PY{n}{sum}\PY{p}{(}\PY{l+m+mi}{0}\PY{p}{)}\PY{p}{)}
              \PY{n}{sample}\PY{p}{[}\PY{n}{i}\PY{p}{,} \PY{p}{:}\PY{p}{]} \PY{o}{=} \PY{n}{np}\PY{o}{.}\PY{n}{array}\PY{p}{(}\PY{p}{[}\PY{n}{r}\PY{o}{.}\PY{n}{sum}\PY{p}{(}\PY{p}{)}\PY{p}{,} \PY{n}{rb}\PY{o}{.}\PY{n}{sum}\PY{p}{(}\PY{p}{)}\PY{p}{]}\PY{p}{)}
\end{Verbatim}

    \begin{Verbatim}[commandchars=\\\{\}]
100\% (100000 of 100000) |\#\#\#\#\#\#\#\#\#\#\#\#\#\#\#\#\#| Elapsed Time: 0:00:29 Time: 0:00:29

    \end{Verbatim}

    \begin{Verbatim}[commandchars=\\\{\}]
{\color{incolor}In [{\color{incolor}106}]:} \PY{n}{r\PYZus{}active} \PY{o}{=} \PY{n}{sample}\PY{p}{[}\PY{p}{:}\PY{p}{,} \PY{l+m+mi}{0}\PY{p}{]} \PY{o}{\PYZhy{}} \PY{n}{sample}\PY{p}{[}\PY{p}{:}\PY{p}{,} \PY{l+m+mi}{1}\PY{p}{]}
          \PY{n+nb}{print}\PY{p}{(}\PY{n}{f}\PY{l+s+s1}{\PYZsq{}}\PY{l+s+s1}{Simulated mean = }\PY{l+s+s1}{\PYZob{}}\PY{l+s+s1}{r\PYZus{}active.mean():.6f\PYZcb{}, std = }\PY{l+s+s1}{\PYZob{}}\PY{l+s+s1}{r\PYZus{}active.std():.6f\PYZcb{}}\PY{l+s+s1}{\PYZsq{}}\PY{p}{)}
\end{Verbatim}

    \begin{Verbatim}[commandchars=\\\{\}]
Simulated mean = 0.009940, std = 0.014216

    \end{Verbatim}

\noindent\rule{16cm}{0.4pt}
%///////////////////////////////////////////////////////////////////////

\section{Problem 3.} 
\subsection{(a)} 
Let $g(\bm{f})$ be the objective function.
\begin{equation}
  \begin{split}
    g(\bm{f}) &= (\bm{r}-\bm{Bf})^{\top} \bm{\Delta}^{-1} (\bm{r}-\bm{Bf}) \\
  &= \bm{r}^{\top} \bm{\Delta}^{-1}\bm{r} - 2 \bm{f}^{\top} \bm{B}^{\top}\bm{\Delta}^{-1} \bm{r} + \bm{f}^{\top} \bm{B}^{\top}\bm{\Delta}^{-1}  \bm{Bf}\\
  \Rightarrow\quad \nabla_{\bm{f}} g &= -2\bm{B}^{\top}\bm{\Delta}^{-1} \bm{r} + 2\bm{B}^{\top}\bm{\Delta}^{-1}  \bm{Bf}\\
  \end{split}
\end{equation}
\subsection{(b)}  
The first order condition yields:
\begin{equation}
  \begin{split}
    &-2\bm{B}^{\top}\bm{\Delta}^{-1} \bm{r} + 2\bm{B}^{\top}\bm{\Delta}^{-1}  \bm{Bf} = 0 \\
    \Rightarrow \quad &\bm{f} = (\bm{B}^{\top}\bm{\Delta}^{-1} \bm{B})^{-1}\bm{B}^{\top}\bm{\Delta}^{-1} \bm{r}
  \end{split}
\end{equation}
\subsection{(c)}  
In this case, both $\bm{b}^{\top}\bm{\Delta}^{-1} \bm{b}$ and $\bm{b}^{\top} \bm{\Delta}^{-1}\bm{r}$ are scalars. So does $f = \bm{b}^{\top} \bm{\Delta}^{-1}\bm{r}/\bm{b}^{\top}\bm{\Delta}^{-1} \bm{b}$. The return of characteristic portfolio is
\begin{equation}
  r_p = \bm{r}^{\top}\bm{x} = \frac{1}{\bm{b}^{\top}\bm{\Delta}^{-1} \bm{b}} \bm{r}^{\top} \bm{\Delta}^{-1}\bm{b} = f^{\top} = f
\end{equation}
\subsection{(d)} 
By assumption, let $\bm{\Delta} = \text{diag}\{\delta_i\}$, in this case
\begin{equation}
  \bm{x} = \frac{1}{\bm{b}^{\top}\bm{\Delta}^{-1} \bm{b}} \bm{\Delta}^{-1}\bm{b} = \frac{1}{\sum_{i=1}^n \delta_i^{-1}} \begin{pmatrix}
    \delta_1^{-1} b_1\\
    \vdots \\
    \delta_n^{-1} b_n
  \end{pmatrix}
\end{equation}
So the result holdings are just $+\delta_i^{-1}/\sum \delta_i^{-1}$ for stocks in long list $(b_i = 1)$, and $-\delta_i^{-1}/\sum \delta_i^{-1}$ for stocks in short list $(b_i = -1)$. If $\Delta = \delta \bm{I}$, the holdings become $\text{sgn}(b_i) \frac{1}{n}$, i.e. $\bm{x} = \frac{1}{n} \bm{b}$, which is just the signs of $\bm{b}$ times a constant scaler $1/n$.

    % Add a bibliography block to the postdoc
    
    
    
    \end{document}

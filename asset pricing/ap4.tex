\documentclass[a4paper, 10pt]{article}    
\usepackage{geometry}       
\geometry{a4paper}
\geometry{margin=1in} 
\usepackage{paralist}
  \let\itemize\compactitem
  \let\enditemize\endcompactitem
  \let\enumerate\compactenum
  \let\endenumerate\endcompactenum
  \let\description\compactdesc
  \let\enddescription\endcompactdesc
  \pltopsep=\medskipamount
  \plitemsep=1pt
  \plparsep=1pt
\usepackage[english]{babel}
\usepackage[utf8]{inputenc}

\usepackage{bbm, bm}
\usepackage{amsmath, amssymb, amsthm, mathrsfs}
\usepackage{booktabs, tikz, array, eurosym}
\usepackage{float}
\renewcommand{\arraystretch}{1.4}
\newcolumntype{L}{>{\arraybackslash}m{10cm}}
\newcommand\indep{\protect\mathpalette{\protect\indeP}{\perp}}
\def\indeP#1#2{\mathrel{\rlap{$#1#2$}\mkern2mu{#1#2}}}

\pagestyle{headings}
\newcommand{\boxwidth}{430pt}

\theoremstyle{definition}
\newtheorem{problem}{Problem}

\newtheoremstyle{hSol}
  {1.0pt}% Space above
  {1.0pt}% Space below
  {}% bodyfont
  {}% indent
  {\bfseries}% thm head font
  {.}% punctuation after thm head
  { }% Space after thm head
  {}% thm head spec

\theoremstyle{hSol}
\newtheorem*{solution}{Solution}



\title{\textbf{Asset Pricing Assignment IV}}
\author{Ze Yang~~~~(zey@andrew.cmu.edu)}

\begin{document}
\maketitle

\begin{table}[h]
\vspace{-10pt}
\caption{\textit{Nomenclatures}}
\vspace{3pt}
\centering
\def\arraystretch{1.15}
\begin{tabular}{lL}
\hline
Notation & \hspace{4.6cm} Description \\ 
\hline
$\bm{w}$ & The $N$-periods coin toss path, $\bm{w} = (w_1~~w_2~~...~~w_N)^{\top}$, where $w_n$ is the outcome for single period $n$.\\
$\psi_n$ & The partial path till $i$-th period. I.e. the event of all $\bm{w}$ that has same trajectory until $i$: $\psi_n=\{\bm{w}: w_k=\hat{w}_k, k =1,2,...,n\}$\\
$S_n \in m \mathcal{F}_n$ & Adaptive process to filtration $\{\mathcal{F}_n\}$. If $\{\mathcal{F}_n\}$ is coin toss filtration, then $S_n = S_n(\bm{w}^{[n]})$.\\
$\tilde{p}(w), \tilde{\mathbb{P}}\left(w\right)$ & Risk neutral probabilities. In the coin toss space, we refer to $\tilde{p}(H)$ as $\tilde{p}$, $\tilde{p}(T)$ as $\tilde{q}$.\\
$\#H(\bm{w})$, $\#T(\bm{w})$ & The number of heads (tails) in path $\bm{w}$. \\
$d(i, j)$ & The discount factor from period $i$ to $j$. \\
$\tilde{\mathbb{E}}_j\left[S_{i}\right](\psi_i)$ & The expectation of security payoff at period $i$ conditional on period $i<j$; it's a random variable with respect to partial path $\psi_i$. Tilde means the risk neutral probability measure.\\
\hline 
\end{tabular}
\label{tab:Nomen}
\end{table}


\noindent\rule{16cm}{0.4pt}
%///////////////////////////////////////////////////////////////////////
\begin{problem} 
\end{problem}
\begin{solution} \textbf{(1)} For the boundary condition $n=N$ we have:
\begin{equation}
  v_N(S_N) = S_N \left(\mathbbm{1}_{\{S_N \leq L\}} + \mathbbm{1}_{\{S_N \geq U\}}\right)
\end{equation}
For $0\leq n<N$:
\begin{equation}
  \begin{split}
      v_n(S_n) &= S_n \left(\mathbbm{1}_{\{S_n \leq L\}} + \mathbbm{1}_{\{S_n \geq U\}}\right) +\frac{\mathbbm{1}_{\{L<S_n<U\}}}{1+r} \tilde{\mathbb{E}}_n\left[v_{n+1}\right] \\
      &= S_n \left(\mathbbm{1}_{\{S_n \leq L\}} + \mathbbm{1}_{\{S_n \geq U\}}\right) +\frac{\mathbbm{1}_{\{L<S_n<U\}}}{1+r} \left(\tilde{p}v_{n+1}(uS_n)+\tilde{q}v_{n+1}(dS_n)\right)
  \end{split}
\end{equation}
So in summary,
\begin{equation}
  v_n(s) = \begin{cases}
    s \left(\mathbbm{1}_{\{s \leq L\}} + \mathbbm{1}_{\{s \geq U\}}\right) & n=N\\
    s \left(\mathbbm{1}_{\{s \leq L\}} + \mathbbm{1}_{\{s \geq U\}}\right) +\frac{\mathbbm{1}_{\{L<s<U\}}}{1+r} \left(\tilde{p}v_{n+1}(us)+\tilde{q}v_{n+1}(ds)\right) & 0\leq n< N
  \end{cases}
\end{equation}
where $\tilde{p}, \tilde{q}$ are risk neutral probabilities, $\tilde{p}=\frac{1+r-d}{u-d}, \tilde{q}=1- \tilde{p}$. \\
~\\
\textbf{(2)} Plug in the numbers we have $\tilde{p}=\frac{1+0.25-0.5}{2-0.5}=0.5$, $\tilde{q}=0.5$.\\
At $n=2$ we have:
$$
  v_2(16) = 16;~~~~v_2(4) = 0;~~~~v_2(1) = 1
$$
At $n=1$:
$$
v_1(8) = 8\cdot(0) +\frac{1}{1.25} \left(0.5\times16+0.5\times0\right) = 6.4
$$
$$
v_1(2) = 2\cdot(1) +0 = 2
$$
At $n=0$:
$$
v_0(4) = 4\cdot(0) +\frac{1}{1.25} \left(0.5\times6.4+0.5\times 2\right) = 3.36
$$
Hence the arbitrage free price of the option is $V_0 = v_0(4)=3.36$.
\end{solution}


\noindent\rule{16cm}{0.4pt}
%///////////////////////////////////////////////////////////////////////
\begin{problem}
\end{problem}
\begin{solution} \textbf{(1)} Since $f_n$ stands for the price of option at $n$ assuming the stock price was inside the band for all $k=1,2,...,n-1$; for the boundary condition $n=N$, $f_N$ must be at least $F\frac{N-1}{N}$, so we have
\begin{equation}
  f_N(S_N) = \frac{F}{N} \mathbbm{1}_{\{L\leq S_N \leq U\}} + F\frac{N-1}{N}
\end{equation}
For $0\leq n<N$, since the stock was inside the band before, the option will pay $F\frac{n-1}{N}$ if the stock price break through the band at this period.
\begin{equation}
  \begin{split}
    f_n(S_n) &= F\frac{n-1}{N} \mathbbm{1}_{\{L\leq S_n \leq U\}} + \left(\mathbbm{1}_{\{S_n \leq L\}} + \mathbbm{1}_{\{S_n \geq U\}}\right) \frac{1}{1+r} \tilde{\mathbb{E}}_n\left[f_n(S_{n+1})\right] \\
    &=F\frac{n-1}{N} \mathbbm{1}_{\{L\leq S_n \leq U\}} + \left(\mathbbm{1}_{\{S_n \leq L\}} + \mathbbm{1}_{\{S_n \geq U\}}\right) \frac{1}{1+r} \left(\tilde{p}f_{n+1}(uS_n)+\tilde{q}f_{n+1}(dS_n)\right)
  \end{split}
\end{equation}
So in summary,
\begin{equation}
  f_n(s) = \begin{cases}
    \frac{F}{N} \mathbbm{1}_{\{L\leq s \leq U\}} + F\frac{N-1}{N} & n=N\\
    F\frac{n-1}{N} \mathbbm{1}_{\{L\leq s \leq U\}} + \left(\mathbbm{1}_{\{s \leq L\}} + \mathbbm{1}_{\{s \geq U\}}\right) \frac{1}{1+r} \left(\tilde{p}f_{n+1}(us)+\tilde{q}f_{n+1}(ds)\right) & 0\leq n< N
  \end{cases}
\end{equation}
where $\tilde{p}, \tilde{q}$ are risk neutral probabilities, $\tilde{p}=\frac{1+r-d}{u-d}, \tilde{q}=1- \tilde{p}$. \\
~\\
\textbf{(2)} Plug in the numbers we have $\tilde{p}=\frac{1+0.25-0.5}{2-0.5}=0.5$, $\tilde{q}=0.5$.\\
At $n=3$ we have:
\begin{equation}
  \begin{split}
    f_3(32) &= 150\times \frac{2}{3} = 100,~~~~~f_3(8) = 150 \\
    f_3(2) &= 150\times \frac{2}{3} = 100,~~~~~f_3(0.5) = 150\times \frac{2}{3} = 100 \\
  \end{split}
\end{equation}
At $n=2$:
$$
f_2(16) = 150\times \frac{1}{3} = 50,~~~~f_2(1) = 150\times \frac{1}{3} = 50
$$
$$
f_2(4) = \frac{1}{1.25}(0.5\times f_3(8)+0.5\times f_3(2)) = 100
$$
At $n=1$:
$$
f_1(8) = \frac{1}{1.25}(0.5\times f_2(16)+0.5\times f_2(4)) = 60,~~~~f_1(2)=0
$$
At $n=0$:
$$
f_0(4) =  \frac{1}{1.25}(0.5\times f_1(8)+0.5\times f_1(2)) = 24
$$
$$
\Delta_0 = \frac{f_1(uS_0)-f_1(dS_0)}{uS_0-dS_0} = \frac{60-0}{8-2} = 10
$$
Hence, the arbitrage-free price at time 0 is $V_0=24$, and the number of shares in replicating strategy is $\Delta_0=10$.
\end{solution}

\noindent\rule{16cm}{0.4pt}
%///////////////////////////////////////////////////////////////////////
\begin{problem} 
\end{problem}
\begin{solution} Denote $G_n=g_n(S_n)$ the payment received at time $n$ if the option is exercised at that period. Denote $f_n$ the price of american option at time $n$, then by the example we've learned in the class:
\begin{equation}
  f_N(S_N) = g_N(S_N)
\end{equation}
and for $0\leq n<N$:
\begin{equation}
  \begin{split}
    f_n(S_n) &= \max\left\{G_n, \tfrac{1}{1+r}\tilde{\mathbb{E}}_n\left[f_{n+1}(S_{n+1})\right]\right\}\\
    &= \max\left\{g_n(S_n), \tfrac{1}{1+r}\left(\tilde{p}f_{n+1}(uS_n)+\tilde{q}f_{n+1}(dS_n)\right)\right\}\\
  \end{split}
\end{equation}
where $\tilde{p}, \tilde{q}$ are risk neutral probabilities, $\tilde{p}=\frac{1+r-d}{u-d}, \tilde{q}=1- \tilde{p}$. In this problem we have $\tilde{p}=\tilde{q}=0.5$. \\
~\\
\textbf{(1)} For American Put
$$
G_n = (K - S_n)^+
$$
The backward induction calculation is shown in the table, where $*$ denotes the node where the option should be (early) executed.
\begin{table}[h]
\vspace{-10pt}
\caption{\textit{American Put}}
\vspace{3pt}
\centering
\def\arraystretch{1.15}
\begin{tabular}{|r|ccc|}
\hline
$n$ & $S_n$ & $G_n=(4-S_n)^+$ & $f_n$\\ 
\hline
3 & 32 & 0 & 0\\
3 & 8 & 0& 0\\
3 & 2 & 2 & $2^*$\\
3 & 0.5 & 3.5 & $3.5^*$\\
\hline
2 & 16 & 0 & 0\\
2 & 4 & 0& 0.8\\
2 & 1 & 3 & $3^*$\\
\hline
1 & 8 & 0 & 0.32\\
1 & 2 & 2 & $2^*$\\
\hline 
0 & 4 & 0 & 0.928\\
\hline
\end{tabular}
\label{tab:put}
\end{table}
It follows that price of American Put is $f_0 = 0.928$. \\
~\\
\textbf{(2)} For the American Call
$$
G_n = (S_n-K)^+
$$
The backward induction calculation is shown in the table, where $*$ denotes the node where the option should be (early) executed.
\begin{table}[h]
\vspace{-10pt}
\caption{\textit{American Call}}
\vspace{3pt}
\centering
\def\arraystretch{1.15}
\begin{tabular}{|r|ccc|}
\hline
$n$ & $S_n$ & $G_n=(S_n-4)^+$ & $f_n$\\ 
\hline
3 & 32 & 28 & $28^*$\\
3 & 8 & 4& $4^*$\\
3 & 2 & 0 & 0\\
3 & 0.5 & 0 & 0\\
\hline
2 & 16 & 12 & $12.8$\\
2 & 4 & 0& $1.6$\\
2 & 1 & 0 & 0\\
\hline
1 & 8 & 4 & 5.76\\
1 & 2 & 0 & 0.64\\
\hline 
0 & 4 & 0 & 2.56\\
\hline
\end{tabular}
\label{tab:call}
\end{table}
It follows that price of American Call is $f_0 = 2.56$. \\
~\\
\textbf{(3)} For the American Straddle
$$
G_n = |S_n-K|
$$
The backward induction calculation is shown in the table, where $*$ denotes the node where the option should be (early) executed.
\begin{table}[H]
\vspace{-10pt}
\caption{\textit{American Straddle}}
\vspace{3pt}
\centering
\def\arraystretch{1.15}
\begin{tabular}{|r|ccc|}
\hline
$n$ & $S_n$ & $G_n=|S_n-4|$ & $f_n$\\ 
\hline
3 & 32 & 28 & $28^*$\\
3 & 8 & 4& $4^*$\\
3 & 2 & 2 & $2^*$\\
3 & 0.5 & 3.5 & $3.5^*$\\
\hline
2 & 16 & 12 & $12.8$\\
2 & 4 & 0& $2.4$\\
2 & 1 & 3 & $3^*$\\
\hline
1 & 8 & 4 & 6.08\\
1 & 2 & 2 & 2.16\\
\hline 
0 & 4 & 0 & 3.296\\
\hline
\end{tabular}
\label{tab:call}
\end{table}
\end{solution}
It follows that price of American Call is $f_0 = 3.296$. \\
~\\
\textbf{(4)} We observe that 
$$
f_0^{Straddle} = 3.296 < 3.488 = f_0^{Put} + f_0^{Call}
$$
Although
$$
G_n^{Straddle} = G_n^{Put} + G_n^{Call}
$$
for any $n$. This does not implies that their price should be equal, because the price is path-dependent and involves early execution. From the calculation we can see that American Straddle has different optimal execution time from American Put. Namely, the put should be executed at $\psi_1 = T$, whereas the straddle should not. As a result, 
$$
f_1^{Straddle}(S_1(T)) = 2.16 < 2 + 0.64 = f_1^{Put}(S_1(T)) + f_1^{Call}(S_1(T))
$$
And hence has strictly lower price at $n=0$.

\noindent\rule{16cm}{0.4pt}
%///////////////////////////////////////////////////////////////////////
\begin{problem} 
\end{problem}
\begin{solution} By same construction as problem 3, denote $G_n=g_n(S_n)$ the payment received at time $n$ if the option is exercised at that period. Denote $f_n$ the price of american option at time $n$, we have, for boundary condition:
\begin{equation}
  f_N(S_N) = g_N(S_N) = \sqrt{S_N}
\end{equation}
and for $1\leq n<N$:
\begin{equation}
  \begin{split}
    f_n(S_n) &= \max\left\{G_n, \tfrac{1}{1+r}\tilde{\mathbb{E}}_n\left[f_{n+1}(S_{n+1})\right]\right\}\\
    &= \max\left\{\sqrt{S_n}, \tfrac{1}{1+r}\left(\tilde{p}f_{n+1}(uS_n)+\tilde{q}f_{n+1}(dS_n)\right)\right\}\\
  \end{split}
\end{equation}
and $n=0$:
\begin{equation}
  f_0(S_0) =\tfrac{1}{1+r}\left(\tilde{p}f_{1}(uS_0)+\tilde{q}f_{1}(dS_0)\right)
\end{equation}
where $\tilde{p}, \tilde{q}$ are risk neutral probabilities, $\tilde{p}=\frac{1+r-d}{u-d}, \tilde{q}=1- \tilde{p}$. In this setting we have $\tilde{p}=\tilde{q}=0.5$.\\
~\\
By Jensen's inequality:
\begin{equation}
  \tilde{p}\sqrt{us} + \tilde{q}\sqrt{ds} \leq \sqrt{\tilde{p}us + \tilde{q}ds}~~(\dag)
\end{equation}
By the definition of risk neutral probability:
$\tilde{p}u + \tilde{q}d = 1+r$. So
\begin{equation}
  (\dag) = \sqrt{\tilde{p}us + \tilde{q}ds} = \sqrt{(1+r)s} < \sqrt{s}
\end{equation}
It follows that 
\begin{equation}
  f_{N-1}(S_{N-1}) = \max\left\{\sqrt{S_{N-1}}, \tfrac{1}{1+r}\left(\tilde{p}\sqrt{uS_{N-1}} + \tilde{q}\sqrt{dS_{N-1}}\right)\right\} = \sqrt{S_{N-1}}
\end{equation}
Then by induction, we obtain that $f_n(S_n) = \sqrt{S_n}$ for each $n=1,2,...,N-1$, i.e. the optimal exercise time is $\hat{\tau}=1$.
$$
f_1(S_1(H)) = \sqrt{4} = 2;~~~~f_1(S_1(T)) = \sqrt{1} = 1
$$
So the arbitrage free price is
$$
f_0 = \frac{1}{1+0.25}(0.5\times 2 + 0.5\times 1) = 1.2
$$
\end{solution}



\end{document}
\documentclass[a4paper, 10pt]{article}    
\usepackage{geometry}       
\geometry{a4paper}
\geometry{margin=1in} 
\usepackage{paralist}
  \let\itemize\compactitem
  \let\enditemize\endcompactitem
  \let\enumerate\compactenum
  \let\endenumerate\endcompactenum
  \let\description\compactdesc
  \let\enddescription\endcompactdesc
  \pltopsep=\medskipamount
  \plitemsep=1pt
  \plparsep=1pt
\usepackage[english]{babel}
\usepackage[utf8]{inputenc}

\usepackage{bbm, bm}
\usepackage{amsmath, amssymb, amsthm, mathrsfs}
\usepackage{booktabs, tikz, array, eurosym}
\usepackage{float}
\renewcommand{\arraystretch}{1.4}
\newcolumntype{L}{>{\arraybackslash}m{10cm}}
\newcommand\indep{\protect\mathpalette{\protect\indeP}{\perp}}
\def\indeP#1#2{\mathrel{\rlap{$#1#2$}\mkern2mu{#1#2}}}

\pagestyle{headings}
\newcommand{\boxwidth}{430pt}

\theoremstyle{definition}
\newtheorem{problem}{Problem}

\newtheoremstyle{hSol}
  {1.0pt}% Space above
  {1.0pt}% Space below
  {}% bodyfont
  {}% indent
  {\bfseries}% thm head font
  {.}% punctuation after thm head
  { }% Space after thm head
  {}% thm head spec

\theoremstyle{hSol}
\newtheorem*{solution}{Solution}



\title{\textbf{Asset Pricing Assignment II}}
\author{Ze Yang~~~~(zey@andrew.cmu.edu)}

\begin{document}
\maketitle

\begin{table}[h]
\vspace{-10pt}
\caption{\textit{Nomenclatures}}
\vspace{3pt}
\centering
\def\arraystretch{1.15}
\begin{tabular}{lL}
\hline
Notation & \hspace{4.6cm} Description \\ 
\hline
$\bm{x}\succeq \bm{y}$ & Componentwise inequality between vectors.\\
& Means: $x_i \geq y_i$ for $i=1,2,...,n$, $n$ is number of components of $\bm{x}, \bm{y}$.\\
$\bm{x}\succ \bm{y}$ & Componentwise strict inequality between vectors.\\
$\tilde{p}(w), \tilde{\mathbb{P}}\left(w\right)$ & Risk neutral probabilities. We refer to $\tilde{p}(w_j)$ as $\tilde{p}_j$.\\
$\tilde{\bm{p}}$ & The $n \times 1$ probability vector such that $\tilde{\bm{p}}\succeq \bm{0}$; $\tilde{\bm{p}}^{\top} \bm{1}=1$.\\
$\bm{A}_i$ & The $n\times m$ payoff matrix at period $i$, where $n=|\Omega|$, $m$ is number of traded securities.\\
$\bm{\Delta}_i$ & The $m\times 1$ vector of positions holdings at period $i$.\\
$\mathbb{E}_{\mathbb{Q}}\left[S_{i+1}\middle|\mathcal{F}_{i}\right](\bm{w})$ & The expectation of security payoff at period $i$ conditional on period $i-1$; it's a random variable with respect to realized path $\bm{w} = (w^{[1]}~~w^{[2]}~~...~~w^{[i]})^{\top}$, where $w^{[i]}$ is the outcome for single period $i$. $\mathbb{Q}$ denotes the probability measure.\\
\hline 
\end{tabular}
\label{tab:Nomen}
\end{table}


\noindent\rule{16cm}{0.4pt}
%///////////////////////////////////////////////////////////////////////
\begin{problem} 
\end{problem}
\begin{solution} \textbf{(1) \& (2).} The payoff matrix for two traded security and 3 states is:
$$
\bm{A}_1 = \begin{pmatrix}
	1+r & S_1(w_1) \\[2pt]
	1+r & S_1(w_2) \\[2pt]
	1+r & S_1(w_3) \\[2pt]
\end{pmatrix} = 
\begin{pmatrix}
	1.25 & 1 \\[2pt]
	1.25 & 2 \\[2pt]
	1.25 & 4 \\[2pt]
\end{pmatrix}
$$
The risk neutral probability is given by: $\frac{1}{1+r}\bm{A}_1^{\top} \tilde{\bm{p}} = \begin{pmatrix}1 & S_0\end{pmatrix}^{\top}$. The system is underdetermined, the solution satisfies: 
\begin{equation}
	\tilde{p}_1=2\tilde{p}_3-0.5 ~~~;~~~ \tilde{p}_2 = 1.5 - 3\tilde{p}_3
\end{equation}
which implies $\tilde{\mathcal{P}} \ne \emptyset$ and $|\tilde{\mathcal{P}}| = \infty$. By FTAPs: the model is arbitrage free but not complete. \\
\textbf{(3).} By the fact $\tilde{\bm{p}}\succeq \bm{0}$ we can deduce that:
\begin{equation}
	\tilde{p}_1=2\tilde{p}_3-0.5 \geq 0 ~~~;~~~ \tilde{p}_2 = 1.5 - 3\tilde{p}_3 \geq 0~~~ \Rightarrow ~~~ \tilde{p}_3 \in \left[\frac{1}{4}, \frac{1}{2}\right]
\end{equation}
The extended payoff matrix (add the call option) becomes
$$
\overline{\bm{A}}_1 = \begin{pmatrix}
	1+r & S_1(w_1) & (S_1(w_1)-K)^{+}\\[2pt]
	1+r & S_1(w_2) & (S_1(w_2)-K)^{+}\\[2pt]
	1+r & S_1(w_3) & (S_1(w_3)-K)^{+}\\[2pt]
\end{pmatrix} = 
\begin{pmatrix}
	1.25 & 1 & 0\\[2pt]
	1.25 & 2 & 0\\[2pt]
	1.25 & 4 & 2\\[2pt]
\end{pmatrix}
$$
If $v$ is the arbitrage free price of call option, the extended market should be arbitrage free, i.e.:
\begin{equation}
	\frac{1}{1+r}\overline{\bm{A}}_1^{\top} \tilde{\bm{p}} = \begin{pmatrix}1 \\ S_0 \\ v\end{pmatrix}
\end{equation}
The third linear equation is:
$$
v = \frac{1}{1+r}(0\cdot\tilde{p}_1 + 0\cdot\tilde{p}_2 + 2 \tilde{p}_3) = 1.6 \tilde{p}_3
$$
since $\tilde{p}_3 \in [\frac{1}{4}, \frac{1}{2}]$, we obtain upper and lower bound for $v$, $v_* < v < v^*$, where $v_*=0.4$ and $v^* = 0.8$. \\
However, neither $v_*$ nor $v^*$ is arbitrage free.\\
\begin{itemize}
	\item[$\cdot$] If $v=v_* = 0.4$, consider the portfolio $\bm{\Delta}_{0*} = \begin{pmatrix}
		1.6 & -1 & 1
	\end{pmatrix}^{\top}$. I.e. long 1 share of call option, short 1 share of stock and save $1.6$ dollars to bank account. The portfolio payoff is
	$$
	\bm{\pi}_{1*} = \overline{\bm{A}}_1 \bm{\Delta}_{0*} = \begin{pmatrix}
		1.6 \times 1.25 - 1 + 0\\
		1.6 \times 1.25 - 2 + 0\\
		1.6 \times 1.25 - 4 + 2\\
	\end{pmatrix} = \begin{pmatrix}
		1 \\
		0 \\
		0
	\end{pmatrix} = \begin{pmatrix}
		\pi_{1*}(w_1) \\
		\pi_{1*}(w_2) \\
		\pi_{1*}(w_3)
	\end{pmatrix}
	$$
	The payoff satisfies $\bm{\pi}_{1*}\succeq \bm{0}$, $\pi_{1*}(w) > 0$ for some $w$. But the initial capital $V_{0*} = \bm{A}_0 \bm{\Delta}_{0*} = 0$. So there exists arbitrage opportunity.
	\item[$\cdot$] If $v=v^* = 0.8$, consider the portfolio $\bm{\Delta}_{0}^* = \begin{pmatrix}
		-0.8 & 1 & -1.5
	\end{pmatrix}^{\top}$. I.e. short $1.5$ share of call option, long 1 share of stock and borrow $0.8$ dollars from bank account. The portfolio payoff is
	$$
	\bm{\pi}_{1}^* = \overline{\bm{A}}_1 \bm{\Delta}_{0}^* = \begin{pmatrix}
		-0.8 \times 1.25 + 1 - 0\\
		-0.8 \times 1.25 + 2 - 0\\
		-0.8 \times 1.25 + 4 - 2 \times 1.5\\
	\end{pmatrix} = \begin{pmatrix}
		0 \\
		1 \\
		0
	\end{pmatrix} = \begin{pmatrix}
		\pi_{1}^*(w_1) \\
		\pi_{1}^*(w_2) \\
		\pi_{1}^*(w_3)
	\end{pmatrix}
	$$
	The payoff satisfies $\bm{\pi}_{1}^*\succeq \bm{0}$, $\pi_{1}^*(w) > 0$ for some $w$. But the initial capital $V_0^* = \bm{A}_0 \bm{\Delta}_{0}^* = 0$. So there exists arbitrage opportunity.
\end{itemize}
\end{solution}




\noindent\rule{16cm}{0.4pt}
%///////////////////////////////////////////////////////////////////////
\begin{problem} 
\end{problem}
\begin{solution} Denote stock price $S_i=S_i(\bm{w})$. First off, we analyze the forward price. We know that:
\begin{itemize}
	\item[$\cdot$] The forward price is determined such that the arbitrage-free price of the forward contract itself is zero at current time. At time $t$, define the value of forward contract entered at $\tau$, expried at time $T$ as $f_{\tau, t, T}$, we have $f_{t, t, T} = 0$.
	\item[$\cdot$] The payoff of forward contract at expiry time $T$ is $V_{\tau, T} = f_{\tau, T, T} = S_T -F$.
\end{itemize}
Assume there exists risk-neutral measure $\tilde{\mathbb{P}}$. For the forward contract entered at $t=1$,
\begin{equation}
	0= f_{1,1,2}(w^{[1]}) = \frac{1}{1+r}\mathbb{E}_{\tilde{\mathbb{P}}}\left[S_2 - F_1\middle|\mathcal{F}_1\right](w^{[1]}) = \frac{1}{1+r}\left(\mathbb{E}_{\tilde{\mathbb{P}}}\left[S_2|\mathcal{F}_1\right](w^{[1]}) - F_1(w^{[1]})\right)
\end{equation}
We have $\mathbb{E}_{\tilde{\mathbb{P}}}\left[S_2|\mathcal{F}_1\right](w^{[1]}) = F_1(w^{[1]})$. However, the forward contract entered at $t=0$ has non-zero value at $t=1$:
\begin{equation}
	\begin{split}
		f_{0,1,2}(w^{[1]}) &= \frac{1}{1+r}\mathbb{E}_{\tilde{\mathbb{P}}}\left[S_2 - F_0\middle|\mathcal{F}_1\right](w^{[1]}) = \frac{1}{1+r}\left(\mathbb{E}_{\tilde{\mathbb{P}}}\left[S_2|\mathcal{F}_1\right](w^{[1]}) - F_0\right) \\
		&=\frac{1}{1+r}\left(F_1(w^{[1]}) - F_0\right) 
	\end{split}
\end{equation}
Hence
$$
f_{0,1,2}(w^{[1]}_1) =\frac{1}{1.25}(2-4)= -1.6,~~~~f_{0,1,2}(w^{[1]}_2)=\frac{1}{1.25}(4-4) = 0, ~~~~f_{0,1,2}(w^{[1]}_3) =\frac{1}{1.25}(9-4)= 4
$$
Secondly, consider the put option on forward. 
\begin{itemize}
	\item[$\cdot$] The put option, if executed, becomes a short forward contract with strike $K$. As stated above, this forward contract has non-negative value at time $t=1$. Therefore, the option is executed $\iff$ this forward contract has positive value.
\end{itemize}
Denote put price as $v(\cdot)$. At time $t=1$, the arbitrage-free value of put option is
\begin{equation}
	v(w^{[1]}) = \frac{1}{1+r}\left(\mathbb{E}_{\tilde{\mathbb{P}}}\left[K-S_2\middle|\mathcal{F}_1\right](w^{[1]})\right)^{+} = \frac{1}{1+r}\left(K-F_1(w^{[1]})\right)^+
\end{equation}
Hence
$$
v(w^{[1]}_1) =\frac{1}{1.25}(4-2)^+= 1.6,~~~~v(w^{[1]}_2)=\frac{1}{1.25}(4-4)^+ = 0, ~~~~v(w^{[1]}_3) =\frac{1}{1.25}(4-9)^+= 0
$$
Now we can write down the payoff matrix at period $1$:
\begin{equation}
	\bm{A}_1 = \begin{pmatrix}
	1+r & f_{0,1,2}(w^{[1]}_1) & v(w^{[1]}_1)\\[2pt]
	1+r & f_{0,1,2}(w^{[1]}_2) & v(w^{[1]}_2)\\[2pt]
	1+r & f_{0,1,2}(w^{[1]}_3) & v(w^{[1]}_3)\\[2pt]
\end{pmatrix} = 
\begin{pmatrix}
	1.25 & -1.6 & 1.6\\[2pt]
	1.25 & 0 & 0\\[2pt]
	1.25 & 4 & 0\\[2pt]
\end{pmatrix}
\end{equation}
$\bm{A}_1$ has full rank, so $\exists 1$ risk neutral measure for period 1. Which is calculted by
\begin{equation}
	\frac{1}{1+r}\bm{A}_1^{\top} \tilde{\bm{p}} = \bm{V}_0=\begin{pmatrix}1 \\ 0 \\ P_0\end{pmatrix} ~~\Rightarrow ~~\tilde{\bm{p}} = (1+r) \bm{A}_1^{-\top}\bm{V}_0 = \begin{pmatrix}
		\frac{5}{8} \\[2pt] \frac{1}{8} \\[2pt] \frac{1}{4} \\[2pt]
	\end{pmatrix}
\end{equation}
\end{solution}



\noindent\rule{16cm}{0.4pt}
%///////////////////////////////////////////////////////////////////////
\begin{problem} 
\end{problem}
\begin{solution} By the FTOI:
\begin{equation}
	\frac{1}{\hat{X}_1(w)}= U'(\hat{X}_1(w)) = \lambda \frac{\tilde{\mathbb{P}}(w)}{\mathbb{P}(w)} ~~\Rightarrow~~\mathbb{P}\left(w\right) = \lambda \hat{X}_1(w) \tilde{\mathbb{P}}(w)
\end{equation}
Take summation over all dimensions of the state space:
\begin{equation}
	1 = \langle \mathbb{P}, \bm{1} \rangle = \lambda \langle \tilde{\mathbb{P}}, \hat{\bm{X}}_1 \rangle
\end{equation}
By the construction of risk-neutral measure we have $\langle \tilde{\mathbb{P}}, \hat{\bm{X}}_1 \rangle = (1+r)X_0$. Plug into the FTOI equation yields: $1 = (1+r)X_0 \lambda$, so $\lambda = \tfrac{1}{(1+r)X_0}$. \\
Look at the FTOI equation again:
\begin{equation}
	\mathbb{P}\left(w\right) = \frac{\hat{X}_1(w) \tilde{\mathbb{P}}(w)}{X_0(1+r)}~~\Rightarrow~~\frac{\mathbb{P}(w)X_0(1+r)}{\hat{X}_1(w)} = \tilde{\mathbb{P}}(w)
\end{equation}
Hence
\begin{equation}
	X_0(1+r)\left\langle \mathbb{P}, \begin{pmatrix}
		\frac{1}{\hat{X}_1(w_1)} & \frac{1}{\hat{X}_1(w_2)} & ...
	\end{pmatrix}\right\rangle = \langle \tilde{\mathbb{P}}, \bm{1} \rangle = 1
\end{equation}
which is true for every optimal strategy $\hat{X}_1$. Threfore:
\begin{equation}
	\begin{cases}
	2\left(p_1 + \frac{1}{2}p_2 + \frac{1}{4}p_3\right) = 1\\
	2\left(\frac{1}{2}p_1 + \frac{1}{4}p_2 + p_1\right) = 1 \\
	\langle \bm{p}, \bm{1} \rangle = 1
	\end{cases}
\end{equation}
We obtain: $\bm{p} = (p_1~~p_2~~p_3)^{\top} = (\tfrac{1}{7}~~\tfrac{4}{7}~~\tfrac{2}{7})^{\top}$.
\end{solution}


\noindent\rule{16cm}{0.4pt}
%///////////////////////////////////////////////////////////////////////
\begin{problem} 
\end{problem}
\begin{solution} \textbf{(1).} Same as problem (3), we have
\begin{equation}
	\frac{1}{\hat{X}_1(w)}= U'(\hat{X}_1(w)) = \lambda \frac{\tilde{\mathbb{P}}(w)}{\mathbb{P}(w)} ~~\Rightarrow~~\mathbb{P}\left(w\right) = \lambda \hat{X}_1(w) \tilde{\mathbb{P}}(w)
\end{equation}
Take summation over all dimensions of the state space:
\begin{equation}
	1 = \langle \mathbb{P}, \bm{1} \rangle = \lambda \langle \tilde{\mathbb{P}}, \hat{\bm{X}}_1 \rangle
\end{equation}
By the construction of risk-neutral measure we have $\langle \tilde{\mathbb{P}}, \hat{\bm{X}}_1 \rangle = (1+r)X_0$. Plug into the FTOI equation yields: $1 = (1+r)X_0 \lambda$, so $\lambda = \tfrac{1}{(1+r)X_0} = 1$. Plug into the FTOI formula we have $\mathbb{P}\left(w\right) = \hat{X}_1(w) \tilde{\mathbb{P}}(w)$, where $\hat{X}_1(w) = \hat{\Delta}_0 S_1(w) + (1-S_0\hat{\Delta}_0 )$. I.e. $p_i =  \hat{\Delta}_0 S_1(w_i) + (1-S_0\hat{\Delta}_0 )\tilde{p}_i$ for $i=1,2,3$.
\begin{equation}
	\begin{cases}
	0.1/\tilde{p}_1 = \hat{\Delta}_0 + (1-2\hat{\Delta}_0) = 1 - \hat{\Delta}_0\\
	0.4/\tilde{p}_2 = 2\hat{\Delta}_0 + (1-2\hat{\Delta}_0) = 1\\
	0.5/\tilde{p}_3 = 3\hat{\Delta}_0 + (1-2\hat{\Delta}_0) = 1+\hat{\Delta}_0\\
	\end{cases}
\end{equation}
Additional equations are provided by the calculation of risk neutural probabilities:
\begin{equation}
	\frac{1}{1+r}\bm{A}_1^{\top} \tilde{\bm{p}} = \bm{V}_0=\begin{pmatrix}1 \\ 2 \end{pmatrix} ~~~~\Rightarrow~~~~ \begin{cases}
	\tilde{p}_1 + \tilde{p}_2 + \tilde{p}_3 = 1\\
	\tilde{p}_1 + 2\tilde{p}_2 + 3\tilde{p}_3 = 2
	\end{cases}
\end{equation}
Combine all equations: $\tilde{p}_2 = 0.4, \tilde{p}_3 = 0.3, \tilde{p}_1 = 0.3$; $\hat{\Delta}_0=\frac{2}{3}$.\\
\textbf{(2).} We have $U'(\hat{X}_1(w)) = \frac{\tilde{\mathbb{P}}(w)}{\mathbb{P}(w)}$. Since $\mathbb{P}$ is given, the first order condition depend only on $\tilde{\mathbb{P}}$. \\
The current wealth at period 1, i.e. $\hat{X}_1(w)$ maximizes the expected utility. If $\tilde{P}$ as we calculated in question 1 remains to be a risk-neutral probability, $\hat{X}_1(w)$ will still be optimal. \\
Therefore, the investor won't alter her strategy as long as $\tilde{\mathbb{P}}$ is still a risk-neutral probability $\iff$ the call option is compatible with $\tilde{\mathbb{P}}$.\\
Hence
\begin{equation}
	C_{indifferent} = \frac{1}{1+r} \mathbb{E}_{\tilde{\mathbb{P}}}\left[(S_1-K)^+\right] = (3-2)\times \tilde{p}_3 = 0.3
\end{equation}
\end{solution}


\end{document}